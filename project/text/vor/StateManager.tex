\subsection{Менеджери стану}

Redux — це інструмент управління як станом даних, так і станом інтерфейсу в JavaScript-додатках. Він підходить для односторінкових додатків, в яких управління станом може з часом стає складним. Redux не пов'язаний з якимось певним фреймворком, і хоча розроблявся для React, може використовуватися з Angular або jQuery.

C Redux всі компоненти отримують свої дані зі сховища. Також зрозуміло, куди компонент повинен відправити інформацію про зміну стану — знову ж в сховище. Компонент тільки ініціює зміну і не піклується про інших компонентах, які повинні отримати цю зміну. Таким чином, Redux робить потік даних більш зрозумілим.

Загальна концепція використання сховищ для координації стану програми — це шаблон, відомий як Flux. Цей шаблон проектування доповнює односпрямований потік даних як в React.

Redux використовує тільки одне сховище для всього стану програми. Оскільки стан знаходиться в одному місці, його називає єдиним джерелом істини. Структура даних сховища повністю залежить від вас, але для реального застосування це, як правило, об'єкт з декількома рівнями укладення.

Такий підхід єдиного сховища є основною відмінністю між Redux і Flux з його численними сховищами.

Згідно з документацією Redux, «Єдиний спосіб змінити стан - передати action — об'єкт, що описує, що сталося». Це означає, що програма не може безпосередньо змінити стан. Замість цього, необхідно передати «action», щоб висловити намір змінити стан в сховищі (рис.~\ref{fig:ReactReduxCommunicate}).
