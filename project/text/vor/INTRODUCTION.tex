\anonsection{ВСТУП}

Якість підготовки спеціалістів у закладах освіти і особливо ефективність використання науково-педагогічного потенціалу залежать певною мірою від рівня організації навчального процесу.

Одна з основних складових цього процесу~--- розклад занять~--- регламентує трудовий ритм, впливає на творчу віддачу викладачів, тому його можна вважати фактором оптимізації використання обмежених ресурсів~--- викладацького складу і аудиторного фонду.

Проблему складання розкладу слід розглядати не тільки як трудомісткий процес, об'єкт автоматизації з використанням комп’ютера, але і як проблему оптимального керування. 

Оскільки всі фактори, що впливають на розклад, практично неможливо врахувати, а інтереси учасників навчального процесу різноманітні, задача складання розкладу є багатокритеріальною з нечіткою множиною факторів.

Незалежно від алгоритму побудови розкладу, виникає прикладна проблема з інструментів різних рівнів, що використовуються в процесі. Саме ним і буде присвячено проведену роботу.

\textbf{Актуальність дослідження} полягає в необхідності забезпечення всіх учасників освітнього процесу доступом до актуальної версії розкладу занять у будь-який час, а також можливості спрощення процесу формування розкладу та подальшої інформатизації освітнього процесу.

\textbf{Об'єкт дослідження}~--- інтерфейси системи для планування та підтримки планування розкладу. \textbf{Предмет дослідження}~--- мобільний та веб-додатки для підтримки планування розкладу в закладах освіти з поділом учнів (вихованців, здобувачів освіти тощо) на стабільні академічні групи.

\textbf{Метою роботи} є проектування розширюваного веб-додатку редагування розкладу та мобільного додатку для перегляду розкладу в закладах освіти з можливістю використання всіма учасниками освітнього процесу та розробка їх робочих прототипу.

Для реалізації мети поставлено наступні \textbf{завдання роботи}:
\begin{enumerate}
	\item проаналізувати характеристики існуючих систем планування, зокрема обсяг їх можливостей;
	\item проаналізувати окремі частини процесу підготовки розкладу на прикладі факультету комп’ютерних наук, фізики та математики ХДУ;
	\item на основі проведеного аналізу розробити вимоги щодо можливостей додатків та їх інтерфейсів;
	\item відповідно до створених вимог розробити проект додатку;
	\item розробити робочий прототип і інтерфейс додатку;
	\item використовувати публічне API розробленого сервісу системи пыдтримки редагування розкладу для збережання та отримання даних;
	\item обґрунтувати використані технології при проектуванні клієнтської частини.
\end{enumerate}

Очікується, що спроектований продукт буде придатний до використання всіма учасниками освітнього процесу в ЗВО.
Робота складається з 2 розділів, містить \totalfigures\ рисунків.
