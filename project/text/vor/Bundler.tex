\subsubsection{Webpack}

При створенні сайту досить стандартною практикою є мати певний процес збірки на місці, щоб полегшити розробку і підготовку файлів до роботи.

Можливо використовувати Grunt або Gulp, побудувавши ланцюжки перетворень, які дадуть можливість подати код в один кінець ланцюжка і отримати мінімізовані CSS та JavaScript на іншому.

Подібні інструменти розробки досить популярні і корисні в наші дні. Проте, є й інший метод полегшення розробки~--- Webpack.

Webpack є так званим «збиральником модулів». Він приймає модулі JavaScript, аналізує їх залежності один від одного, а потім з'єднує їх найефективнішим способом, випускаючи у кінці лише один JavaScript файл~\cite{juhovepsalainen2016}.

З Webpack, модулі не обмежені тільки файлами JavaScript. Завдяки частині loaders, Webpack розуміє, що модуль JavaScript може потребувати CSS файл, а цей CSS файл може потребувати зображення. Результат роботи Webpack буде містити тільки те, що потрібно у проекті.

\subsubsection{Babel}

Нажаль, при постійному розвитку мов програмування невід'ємною є ситуація, що реалізація часто відстає від специфікації. Більш того, різні реалізації по-різному відстають від специфікації. Написавши код, ми не можемо гарантувати, де він буде запускатися, а де~--- ні.

Виходячи з цього можна зробити висновок, що потрібно писати код, дотримуючись старих стандартів. На щастя, є інший шлях: ми можемо писати код з використанням всіх найновіших можливостей, але перед публікацією автоматично транслювати його (тобто переводити з одного виду в інший) в стару версію. 

Сама природа JS і його способи використання готують нас до того, що ніколи не настане моменту, коли у всіх користувачів буде остання версія інтерпретатору. Люди використовували і продовжать використовувати різні браузери і різні версії браузерів, різні версії Node.js і так далі. Використання нових синтаксичних конструкцій в такій ситуації практично неможливо. Запуск коду на платформі що не підтримує новий синтаксис призведе до синтаксичної помилки. 

Закономірним вирішенням цієї проблеми стала поява Babel~--- програми, яка бере вказаний код і повертає той же код, але трансльований в стару версію JS. Фактично, в сучасному світі Babel став невід'ємною частиною JS. Всі нові проекти так чи інакше розробляють з його використанням~\cite{davidgeary2019}.
