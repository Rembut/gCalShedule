\label{subsubs:KIS}

Корпоративна інформаційна система — це інформаційна система, яка підтримує автоматизацію функцій управління на підприємстві і постачає інформацію для прийняття управлінських рішень. У ній реалізована управлінська ідеологія, яка об'єднує бізнес-стратегію підприємства і прогресивні інформаційні технології~\cite{гужва2001інформаційні}.

У загальному визначенні «автоматизована система» — сукупність керованого об’єкта й автоматичних керувальних пристроїв, у якій частину функцій керування виконує людина. Вона представляє собою організаційно-технічну систему, що забезпечує вироблення рішень на основі автоматизації інформаційних процесів у різних сферах діяльності~\cite{гужва2001інформаційні}. 

Сучасні автоматизовані системи управління навчальним процесом у  закладах вищої освіти здатні вирішувати велику кількість функцій, а саме:
\begin{itemize}
	\item планування, контроль та аналіз навчальної діяльності;
	\item оперативний доступ до інформації про навчальний процес;
	\item єдину систему звітів, як внутрішніх, так і за вимогами МОН України;
	\item системи безпеки даних з урахуванням вимог законодавства;
	\item облік контингенту студентів та співробітників;
	\item проведення вступної кампанії;
	\item формування пакетів даних з метою виготовлення тих чи інших документів.
\end{itemize}

Функціонування будь-якої автоматизованої системи можна швидко адаптувати до особливостей навчального процесу конкретного навчального закладу, до локальних мереж різного рівня, що допомагає розширити коло користувачів (адміністрації, викладачів і студентів) для оперативного забезпечення їх необхідною інформацією. 

Отже, використання таких систем дає змогу не тільки удосконалити якість планування навчального процесу, а й оперативність управління ним.

Не зважаючи на всі переваги, які надає використання автоматизованих систем, досі далеко не в кожному закладі вони впроваджені чи використовуються в повній мірі з тих чи інших причин — інерційності поглядів адміністрації, супротив працівників або «саботаж» на місцях, відсутність фінансової або організаційної можливості.

\subsubsection{Інформаційно-аналітична система} \label{subs:ias}

В ХДУ використовується корпоративна інтегрована система «Інформаційно-аналітична система (IAS)». Вона дозволяє вести облік працівників і студентів, бухгалтерський облік, контроль за матеріальними цінностями тощо. 
		
Програма IAS орієнтована на платформу Windows з використанням MS SQL Server. Вона має багаторівневу архітектуру, що складається з бази даних, бізнес-логіки та клієнтського інтерфейсу.

Відсутність компонентів, пов’язаних з формуванням розкладу занять, та відсутність у використанні сторонніх рішень ставить задачу з проектування власного додатку для забезпечення всіх учасників освітнього процесу доступом до актуальної версії розкладу занять у будь-який час, а також можливості спрощення процесу формування розкладу та подальшої інформатизації освітнього процесу.

\subsubsection{Єдине інформаційно-освітнє серидовище ХДУ}

Разом з осучасненням освітнього процесу у цілому актуальною є заадча з об'єднання існуючих інформаційних систем університету.

В результаті проведеної інтеграції планується отримати так званий <<Особистий кабінет студента>>, в котрому студент, як основний учасник освітнього процесу, матиме доступ до всієї необхідної інформації. Наразі інформаційне серидовище включає в себе низку в цілому незалежних один від одного проектів, а саме:
\begin{itemize}
	\item web-портал університету \cite{KspuEdu};
	\item система дистанційного навчання <<KSU Online>> \cite{KsuOnline};
	\item система дистанційної освіти <<Херсонський Віртуальний Університет>> \cite{KsuDis}.
	\item програмний комплекс <<ST-Абітурієнт>>, що використовується для підтримки процесу прийому документів абітурієнтів та обробки заяв про зарахування, результатів вступних іспитів тощо;
	\item програмний комплекс <<Інформаційно-аналітична система (ІАС)>>, що дозволяє вести облік співробітників і студентів, бухгалтерський облік, контроль за матеріальними цінностями (розділ~\ref{subs:ias}), проте лише частково охоплює навчальний процес;
	\item сервіс <<KSU Feedback>> призначений для проведення анонімного або звичайного голосування за визначеними критеріями серед строго респондентів \cite{KsuFeedback};
	\item сервіс  <<Пошук книг в електронному каталозі бібліотеки>> надає доступ до каталогу в будь який момент \cite{eLibrary};
	\item web-портал <<Збірник наукових праць <<Інформаційні технології в освіті>> (ІТО)>> є каналом поширення та передачі знань, де вчені, практики та дослідники можуть обговорювати, аналізувати, критикувати, синтезувати, спілкуватися та підтримувати розробку та впровадження ІТ і пов'язаних з ними наслідків у всіх аспектах їх використання у сфері освіти \cite{ITO};
	\item web-портал <<Чорноморський ботанiчний журнал>>, но котрому у відкритому доступі знаходяться електронні версії всіх статей у форматі pdf, опублікованих у журналі з 2005 року.
\end{itemize}

При цьому наразі відсутня будь-яка інтеграція між сервісами та сайтами, крім посилань на певну частину з нах на головній сторінці web-порталу університету.
