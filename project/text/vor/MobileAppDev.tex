\subsubsection{Вимоги до додатку та системи виконання}

Одним із завдань, поставлених для реалізації мети є розроблення вимог щодо можливостей мобільного додатку та його інтерфейсу.

Після проведення аналізу предметної області, додатків аналогів та технологій було сформульовано наступні вимоги:

\begin{enumerate}
    \item Мобільний додаток повинен виконуватися на платформі $ios$.
    \item Можливість вибору та збереження групи користувача.
    \item Перегляд поточного розкладу.
    \item Перегляд розкладу в конкретний день тижня в минулому або майбутньому. 
    \item Функція сповіщення про зміни у розклади користувача.
    \item Можливість пошуку та перегляду розкладу занять викладачів.
    \item Можливість редагування збереженої інформації про користувача додатку.
\end{enumerate}

\subsubsection{React Native}

\addimg{ReactNative.png}{0.7}{Структура React Naive}{fig:ReactNative}

React Native~--- це JS-фреймворк для створення нативних iOS і Android додатків. В його основі лежить розроблена в Facebook JS-бібліотека React (проаналізовано та описано в пункті \ref{subs:React}), призначена для створення користувацьких інтерфейсів. Але замість браузерів вона орієнтована на мобільні платформи. Іншими словами, якщо ви веб-розробник, то можете використовувати React Native для написання  швидких мобільних додатків, з комфортом використання у розробці звичного фреймворка і єдиної кодової бази JavaScript~\cite{jeffgothelf2016}.

Популярність React має ряд причин. Вона компактна і має високу продуктивність, особливо при роботі з швидкоплинними даними. Завдяки компонентної структурі, React заохочує до написання модульного коду и пропагує повторне його використання.

Спроектовано та розроблено серію компонентів для різних частин додатку. На рис.~\ref{fig:interphace} зображено сукупність компонентів, яка формує стартовове вікна додатку, у якому користувач може брати номер своєї групи для отримання розкладу (зліва) та сукупність компонентів, яка формує вікно відображення поточного розкладу з можливістю обрання конкретної дати для відображення (справа). Перший компонент зберігає номер групи локально на мобільному пристрої завдяки використанню функції $AsynkStorage()$, котра надається платформою React Native (код компоненту на рис.~\ref{fig:ReactNativeCode}).

\addtwoimghere{Mobile1.png}{Mobile1.png}{0.45}{Інтерфейс мобільного додатку}{fig:interphace}

React Native~--- це той же React, але для мобільних платформ. У нього є ряд відмінностей: замість тега $div$ використовується компонент $View$, а замість тега $img$~--- $Image$. Вам може стати в нагоді знання Objective-C, Swift або Java~\cite{9781787282537}. 

Нативну (native) розробку можна назвати <<рідною>> для операційних систем~--- Android, IOS, Win Phone і т.д. Такі мобільні додатки пишуться на мовах програмування, затверджених розробниками програмного забезпечення під кожну конкретну платформу, а тому органічно вбудовуються в самі операційні системи.

У React компонент описує власне відображення, а потім бібліотека обробляє для вас рендеринг (деталі у пункті \ref{fig:ReactNative}). Ці дві функції розділені ясним рівнем абстракції. Якщо потрібно відобразити компоненти для вебу, то React використовує стандартні HTML-теги. Завдяки тому ж рівню абстракції~--- <<мосту>>~--- для рендеринга в iOS і Android React Native викликаються відповідні API.

\addCodeAsImg{\lstinputlisting[numbers=left]{code/ReactNativeCode.tex}}{React Native компонент стартового вікна}{fig:ReactNativeCode}

Замість компіляції в нативний код, React Native запускає його за допомогою JS-движка хост-платформи, без блокування основного UI-потоку. Ви отримуєте переваги нативних продуктивності, анімації і поведінки без необхідності писати на Objective-C або Java. Інші методи розробки кроссплатформенних додатків, на кшталт Cordova або Titanium, ніколи не досягнуть такого рівня нативної продуктивності або відображення.

Головна перевага нативних додатків~--- то, що вони оптимізовані під конкретні операційні системи, а значить працюють коректно і швидко. Також вони мають доступ до апаратної частини пристроїв, тобто можуть використовувати в своєму функціоналі камеру смартфона, мікрофон, акселерометр, геолокацію, адресну книгу, плеєр і т.д. Можна налаштувати отримання push-повідомлень. Ще один плюс - економну витрату ресурсів телефону (батарея, пам'ять).

Головна особливість React Native~--- він «дійсно» нативний. Інші рішення JavaScript для мобільних платформ просто обертають ваш JS-код в веб-відображення. Вони можуть перереалізовать яке-небудь нативное поведінку інтерфейсу, наприклад, анімацію, але все ж  залишаються веб-додатоком.

\paragraph{Переваги для розробника}

У порівнянні зі стандартною розробкою під iOS і Android, React Native має набагато більше переваг. Оскільки додаток здебільшого складається з JavaScript, можливо користуватися численними перевагами веб-розробки. Наприклад, щоб побачити внесені в код зміни, можна миттєво «оновити» додаток замість тривалого очікування завершення традиційної компіляції~\cite{robinwieruch2018}. 

Крім того, React Native надає «розумну» систему повідомлень про помилки і стандартні інструменти налагодження JavaScript, що сильно полегшує процес мобільного розробки.

\paragraph{Обробка декількох платформ}

React Native витончено обробляє різні платформи. Переважна більшість API у фреймворку~--- кросплатформені, так що досить просто написати компонент React Native, і він буде без проблем працювати на iOS і Android платформах.

Якщо вам потрібно писати залежний від платформи код - в зв'язку з різними правилами взаємодії в iOS і Android, або через переваг платформозалежного API~--- то з цим не буде труднощів. React Native дозволяє призначати платформозалежні версії кожного компонента, які ви можете потім інтегрувати в свій додаток.
