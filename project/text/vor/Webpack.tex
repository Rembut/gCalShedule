\subsubsection{Webpack}

При створенні сайту досить стандартною практикою є мати певний процес збірки на місці, щоб полегшити розробку і підготовку файлів до роботи.

Можливо використовувати Grunt або Gulp, побудувавши ланцюжки перетворень, які дадуть можливість подати код в один кінець ланцюжка і отримати мінімізовані CSS та JavaScript на іншому.

Подібні інструменти розробки досить популярні і корисні в наші дні. Проте, є й інший метод полегшення розробки — Webpack.

Webpack є так званим «збиральником модулів». Він приймає модулі JavaScript, аналізує їх залежності один від одного, а потім з'єднує їх найефективнішим способом, випускаючи у кінці лише один JavaScript файл.

З Webpack, модулі не обмежені тільки файлами JavaScript. Завдяки частині loaders, Webpack розуміє, що модуль JavaScript може потребувати CSS файл, а цей CSS файл може потребувати зображення. Результат роботи Webpack буде містити тільки те, що потрібно у проекті.
