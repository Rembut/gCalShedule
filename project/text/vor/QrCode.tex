\subsubsection{Використання QR-кодів}
\addimg{QRcode.png}{0.25}{Приклад QR-коду з посиланням}{fig:QRcode}

Було проаналізовано перспективи при використанні QR-кодів (рис.~\ref{fig:QRcode}) з метою супроводження традиційного паперового розкладу (та інших документів), що публікується на стендах.

Хоча термін «QR code» є зареєстрованим товарним знаком японської корпорації «DENSO Corporation», їх використання не обкладається ніякими ліцензійними відрахуваннями, коди описані та опубліковані як стандарти ISO~\cite{воронкін2014можливості}. Основна перевага QR-коду – легке розпізнавання скануючим обладнанням (за допомогою мобільного телефону, планшета або ноутбука з камерою, на яких встановлена програма для зчитування кодів, тощо).

Одним з способів використання QR-кодів в навчальному процесі, крім запропонованих (зокрема, задля забезпечення швидкого доступу до навчально-методичного забезпечення, довідкової літератури, веб-сервісів навчального закладу) \cite[146]{воронкін2014можливості},  можна назвати надання доступу до електронної версії розкладу.
