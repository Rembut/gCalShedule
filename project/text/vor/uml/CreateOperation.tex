\begin{umlstyle}

\begin{umlseqdiag}
	\umlactor[no ddots, x=1]{User}
	\umlboundary[no ddots, x=5]{App}
	\umldatabase[no ddots, x=14, fill=blue!20]{DB}
	
	\begin{umlcall}[op=post request, type=synchron, return=response, padding=3]{User}{App}
		\begin{umlcall}[op=auth procedure, type=synchron]{App}{App}\end{umlcall}
		
		\begin{umlfragment}[type=create, fill=green!20, label=OK]
				\umlcreatecall[no ddots, x=11]{App}{Object}
				\begin{umlcall}[op=init, type=synchron, return=object]{App}{Object}
					\begin{umlfragment}[type=Store, name=Store, fill=cyan!20, label=OK]
						\begin{umlcall}[op=insert query, type=synchron, return=result]{Object}{DB}
							\begin{umlcall}[op=run query, type=synchron]{DB}{DB}\end{umlcall}				
						\end{umlcall}
					\end{umlfragment}
				\end{umlcall}	
				
			\umlnote[x=15.1, y=-8.8, fill=cyan!20]{Store}{Check permissions for run query, validate and run it (store object or raise an exception)}
			\umlfpart[Error]
			\begin{umlcall}[op=undo creation, type=synchron,]{App}{Object}\end{umlcall}
			\begin{umlcall}[op=error, type=synchron]{App}{App}\end{umlcall}
		
		\end{umlfragment}
	\end{umlcall}
		
	
\end{umlseqdiag}

\end{umlstyle}

