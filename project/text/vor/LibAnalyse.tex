\subsubsection{Аналіз існуючих бібліотек для розробки SPA}

На сьогодні, одними з розповсюджених фреймворків для розробки SPA (single page applications) є React, Angular та Vue (рис.~\ref{fig:ReactAngularVue}).

\addimg{ReactAngularVue.png}{0.35}{React, Angular та Vue}{fig:ReactAngularVue}

Angular -- Javascript-фреймворк, створений на основі TypeScript. Розроблений і підтримуваний компанією Google, він описується як JavaScript MVW-фреймворк. На даний момент останньою версією є 4. Фреймворк Angular використовується такими компаніями, як Google, Wix, weather.com, healthcare.gov і Forbes.

\label{subs:vue}
Vue -- ще один JS-фреймворк. Творці Vue описують його як «інтуїтивно зрозумілий та швидкий, призначений для створення інтерактивних інтерфейсів». Вперше він був представлений колишнім співробітником компанії Google Еваном Ю (Evan You) в лютому 2014 року. На даний момент фреймворк використовується такими компаніями, як Alibaba, Baidu, Expedia, Nintendo, GitLab~\cite{emmitscott2015}.

\subsubsection{Вимоги до додатку}

Одним із завдань, поставлених для реалізації мети є розроблення вимог щодо можливостей веб-додатку та його інтерфейсу.

Після проведення аналізу предметної області, додатків аналогів та технологій було сформульовано наступні вимоги:

\begin{enumerate}
    \item Веб-додаток повинен коректно відражатися у останніх версіях популярних веб-браузерів(Google Chrome, Mozilla Firefox, Opera, Safari).
    \item Адаптивність інтерфейсу до розмірів вікна браузера або пристрою.
    \item Можливість зміни типу відображення поточного розкладу.
    \item Зміна інтерфейсу відповідно до прав доступу поточного користувача.
    \item Створення нових користувачів у системі адміністратором.
    \item Авторизація користувачів у системі.
    \item Редагування даних користувачами системи у відповідності до їх прав доступу.
    \item Редагування даних адміністраторами системи у відповідності до їх прав доступу.
    \item Інтеграція з Google сервісами, зокрема Google calendar та Google Sheets.
    \item Інтерфейс для імпорту та експорту даних між системою та сервісами Google або специфікованими форматами даних.
    \item Забезпечення цілісності даних.
\end{enumerate}


SPA (Single Page Application) -- односторінковий JavaScript додаток, що запускається і працює в браузері. На відміну від «традиційного» сайту (рис.~\ref{fig:nespa}), архітектура на SPA-сайтах побудована так, що рендеринг сторінки повністю відбувається на стороні клієнта, а не на стороні сервера (рис.~\ref{fig:spa}).

У браузері користувача запускається JavaScript-додаток, а весь необхідний вміст сторінок динамічно завантажується за допомогою технології AJAX. Навігація по сайту відбувається без перезавантаження сторінок. За рахунок такої архітектури, SPA-сайти працюють швидше, ніж «традиційні» сайти.

\addimg{spa.png}{0.7}{Виконання запитів у SPA}{fig:spa}

\addimg{nespa.png}{0.7}{Виконання запитів у веб-додатку з звичайною організацією}{fig:nespa}

Розглянемо детальніше, як відбувається завантаження і рендеринг вмісту на SPA-сайтах (рис.~\ref{fig:spa}):

\begin{enumerate}
  \item Користувач ініціює запит HTML вміст сайту.
  \item У відповідь приходить JavaScript-додаток.
  \item Додаток визначає, на якій сторінці знаходиться користувач, і який вміст йому потрібно відобразити.
  \item За допомогою AJAX користувач отримує необхідний контент: CSS, JS, HTML і текстовий контент.
  \item JavaScript-додаток обробляє отримані дані і відображає їх в браузері.
  \item При навігації по сайту оновлюється не вся сторінка, а тільки необхідне вміст.
\end{enumerate}

Плюси SPA-сайтів:

\begin{enumerate}
  \item Висока швидкість роботи.
  \item Швидка розробка.
  \item Створення версій для різних платформ на основі готового коду (desktop і mobile додатки).
\end{enumerate}

Мінуси SPA-сайтів:

\begin{enumerate}
  \item JavaScript не обробляється більшістю пошукових систем.
  \item SPA-сайти не працюють без включеного JS в браузері.
  \item Їх не можна аналізувати на предмет помилок популярними програмами та інструментами (наприклад, Netpeak Spider).
\end{enumerate}