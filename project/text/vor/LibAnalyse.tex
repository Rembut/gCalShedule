\subsubsection{Аналіз існуючих бібліотек для розробки SPA}

На сьогодні, одними з росповсюджених фреймворків для розробки SPA (single page applications) є React, Angular та Vue (рис.~\ref{fig:ReactAngularVue}).

\addimg{ReactAngularVue.png}{0.35}{React, Angular та Vue}{fig:ReactAngularVue}

Angular -- Javascript-фреймворк, створений на основі TypeScript. Розроблений і підтримуваний компанією Google, він описується як JavaScript MVW-фреймворк. На даний момент останньою версією є 4. Фреймворк Angular використовується такими компаніями, як Google, Wix, weather.com, healthcare.gov і Forbes.

\label{subs:vue}
Vue -- ще один JS-фреймворк. Творці Vue описують його як «інтуїтивно зрозумілий та швидкий, призначений для створення інтерактивних інтерфейсів». Вперше він був представлений колишнім співробітником компанії Google Еваном Ю (Evan You) в лютому 2014 року. На даний момент фреймворк використовується такими компаніями, як Alibaba, Baidu, Expedia, Nintendo, GitLab.

\subsubsection{Вимоги до додатку}

Одним із завдань, поставлених для реалізації мети є розроблення вимог щодо можливостей веб-додатку та його інтерфейсу.

Після проведення аналізу предметної області, додатків аналогів та технологій було сформульовано наступні вимоги:

\begin{enumerate}
    \item Веб-додаток повинен коректно відражатися у останніх версіях популярних веб-браузерів(Google Chrome, Mozilla Firefox, Opera, Safari).
    \item Адаптивність інтерфейсу до розмірів вікна браузера або пристрою.
    \item Можливість зміни типу відображення поточного розкладу.
    \item Зміна інтерфейсу відповідно до прав доступу поточного користувача.
    \item Створення нових користувачів у системі адміністратором.
    \item Авторизація користувачів у системі.
    \item Редагування даних користувачами системи у відповідності до їх прав доступу.
    \item Редагування даних адміністраторами системи у відповідності до їх прав доступу.
    \item Інтеграція з Google сервісами, зокрема Google calendar та Google Sheets.
    \item Інтерфейс для імпорту та експорту даних між системою та сервісами Google або специфікованими форматами даних.
    \item Забезпечення цілісності даних.
\end{enumerate}


