\anonsection{ВИСНОВКИ}

Для виконання поставлених завдань було проведено аналіз характеристик існуючих систем планування, зокрема обсяг їх можливостей.

При підготовці до проектування було приділено увагу окремим частини процесу підготовки розкладу на прикладі факультету комп’ютерних наук, фізики та математики ХДУ.

На основі проведеного аналізу розроблено базові вимоги щодо можливостей додатку та його інтерфейсу.

Суттєву частку роботи приділено аналізу існуючих технологій всіх рівнів для створення веб-додатків. Детально досліджено роботу клієнт-серверних додатків та взаємодію через API (прикладний програмний інтерфейс). 

Відповідно до створених вимог розроблено проект додатку у відповідності до загальноприйнятих принципів побудови веб-додатків.

При реалізації веб-додатку використано бібліотеку React, завдяки якій було розроблено набір компонентів, що використовувались як в веб-додатку, так і в мобільному додатку. Останнє було досягнуто завдяки вибору платформи React Native~--- фреймворку для розробки мобільних додатків з відкритим кодом.

Розроблені додатки використовують публічний API для реалізації бізнес-логіки та взамодії з серверною частиною.


При розробці проекту використовується система контролю версій git з публічним репозиторієм на сервісі GitHub (github.com/ Rembut/gCalShedule), що дозволяє використовувати сучасні методи сумісної роботи та, одночасно з тим, дозволяє використовувати результати проведеного дослідження всім охочим під ліцензією MIT.
