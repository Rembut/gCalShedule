% За основу взят github.com/Amet13/bachelor-diploma

\documentclass[a4paper,14pt]{extarticle} % 14й шрифт
%%% Преамбула %%%

\usepackage{fontspec} % XeTeX
\usepackage{xunicode} % Unicode для XeTeX
\usepackage{xltxtra}  % Верхние и нижние индексы
\usepackage{pdfpages} % Вставка PDF

\usepackage{listings} % Оформление исходного кода
\lstset{
    basicstyle=\small\ttfamily, % Размер и тип шрифта
    breaklines=true, % Перенос строк
    tabsize=2, % Размер табуляции
    literate={--}{{-{}-}}2 % Корректно отображать двойной дефис
}

% Шрифты, xelatex
\defaultfontfeatures{Ligatures=TeX}
\setmainfont{Times New Roman} % Нормоконтроллеры хотят именно его
\newfontfamily\cyrillicfont{Times New Roman}
%\setsansfont{Liberation Sans} % Тут я его не использую, но если пригодится
\setmonofont{FreeMono} % Моноширинный шрифт для оформления кода

% Украинский язык
\usepackage{polyglossia}
\setdefaultlanguage{ukrainian}
\setotherlanguage{russian}
\setotherlanguage{english}

\usepackage[autostyle]{csquotes} % Стиль кавычек
\DeclareQuoteAlias{russian}{ukrainian} % csquotes не знает об украинском языке. Примем, что кавычки аналогичные русскому

\usepackage[
backend=biber, %подключение пакета biber (тоже нужен)
bibstyle=gost-numeric, %подключение одного из четырех главных стилей biblatex-gost 
citestyle=numeric-comp, %подключение стиля стиля (а вот!) 
language=auto, %указание сортировки языков
babel=other, %указание языков
sorting=ntvy, %тип сортировки в библиографии
doi=false, 
eprint=false, 
isbn=false, 
dashed=false, 
url=false %все false выключают отображение полей, заполненных в библиографической базе, но не актуальных для печатного листа
]{biblatex} % Библиография

\usepackage{import} % Разбиение текста проекта на файлы

\usepackage{import} % Разбиение текста проекта на файлы

\usepackage{amssymb,amsfonts,amsmath} % Математика
\usepackage{tikz-uml} %  UML диаграммы
\numberwithin{equation}{section} % Формула вида секция.номер

\usepackage{enumerate} % Тонкая настройка списков
\usepackage{indentfirst} % Красная строка после заголовка
\usepackage{float} % Расширенное управление плавающими объектами
\usepackage{multirow} % Сложные таблицы

% Пути к каталогам с изображениями
\usepackage{graphicx} % Вставка картинок и дополнений
\graphicspath{{img/}}

% Формат подрисуночных записей
\usepackage{chngcntr}
\counterwithin{figure}{section}

% Гиперссылки
\usepackage{hyperref}

% Оформление библиографии и подрисуночных записей через точку
\makeatletter
\renewcommand*{\@biblabel}[1]{\hfill#1.}
\renewcommand*\l@section{\@dottedtocline{1}{1em}{1em}}
\renewcommand{\thefigure}{\thesection.\arabic{figure}} % Формат рисунка секция.номер
\renewcommand{\thetable}{\thesection.\arabic{table}} % Формат таблицы секция.номер
\def\redeflsection{\def\l@section{\@dottedtocline{1}{0em}{10em}}}
\makeatother

\renewcommand{\baselinestretch}{1.4} % Полуторный межстрочный интервал
\parindent 1.27cm % Абзацный отступ

\sloppy             % Избавляемся от переполнений
\hyphenpenalty=1000 % Частота переносов
\clubpenalty=10000  % Запрещаем разрыв страницы после первой строки абзаца
\widowpenalty=10000 % Запрещаем разрыв страницы после последней строки абзаца

% Отступы у страниц
\usepackage{geometry}
\geometry{left=3cm}
\geometry{right=1cm}
\geometry{top=2cm}
\geometry{bottom=2cm}

% Списки
\usepackage{enumitem}
\setlist[enumerate,itemize]{leftmargin=12.7mm} % Отступы в списках

\makeatletter
    \AddEnumerateCounter{\asbuk}{\@asbuk}{м)}
\makeatother
\setlist{nolistsep} % Нет отступов между пунктами списка
\renewcommand{\labelitemi}{--} % Маркет списка --
\renewcommand{\labelenumi}{\asbuk{enumi})} % Список второго уровня
\renewcommand{\labelenumii}{\arabic{enumii})} % Список третьего уровня

% Содержание
\usepackage{tocloft}
\renewcommand{\cfttoctitlefont}{\hspace{0.38\textwidth}\MakeTextUppercase} % СОДЕРЖАНИЕ
\renewcommand{\cftsecfont}{\hspace{0pt}}            % Имена секций в содержании не жирным шрифтом
\renewcommand\cftsecleader{\cftdotfill{\cftdotsep}} % Точки для секций в содержании
\renewcommand\cftsecpagefont{\mdseries}             % Номера страниц не жирные
\setcounter{tocdepth}{3}                            % Глубина оглавления, до subsection
\setcounter{secnumdepth}{4}                         % Глубина нумерации, до paragraph

% Нумерация страниц справа сверху
\usepackage{fancyhdr}
\pagestyle{fancy}
\fancyhf{}
\fancyhead[R]{\textrm{\thepage}}
\fancyheadoffset{0mm}
\fancyfootoffset{0mm}
\setlength{\headheight}{17pt}
\renewcommand{\headrulewidth}{0pt}
\renewcommand{\footrulewidth}{0pt}
\fancypagestyle{plain}{ 
    \fancyhf{}
    \rhead{\thepage}
}

% Формат подрисуночных надписей
\RequirePackage{caption}
\DeclareCaptionLabelSeparator{defffis}{ -- } % Разделитель
\captionsetup[figure]{justification=centering, labelsep=defffis, format=plain} % Подпись рисунка по центру
\captionsetup[table]{justification=raggedright, labelsep=defffis, format=plain, singlelinecheck=false} % Подпись таблицы слева
\addto\captionsrussian{\renewcommand{\figurename}{Рис.}} % Имя фигуры

% Пользовательские функции
\usepackage{flafter} % рисунок появится не раньше первой ссылки на него
\renewcommand{\floatpagefraction}{.8}
\renewcommand{\topfraction}{.8}
\renewcommand{\bottomfraction}{.4}

\newcommand{\addimg}[4]{ % Добавление одного рисунка
    \begin{figure}[tb]
        \centering
        \includegraphics[width=#2\linewidth]{#1}
        \caption{#3} \label{#4}
    \end{figure}
}
\newcommand{\addCodeAsImg}[3]{ % Добавление tikz рисунка
    \begin{figure}[tb]
        \centering
        {#1}
        \caption{#2} \label{#3}
    \end{figure}
}
\newcommand{\addimghere}[4]{ % Добавить рисунок непосредственно в это место
    \begin{figure}[H]
        \centering
        \includegraphics[width=#2\linewidth]{#1}
        \caption{#3} \label{#4}
    \end{figure}
}
\newcommand{\addtwoimghere}[5]{ % Вставка двух рисунков
    \begin{figure}[H]
        \centering
        \includegraphics[width=#3\linewidth]{#1}
        \hfill
        \includegraphics[width=#3\linewidth]{#2}
        \caption{#4} \label{#5}
    \end{figure}
}
\newcommand{\addthreeimghere}[6]{ % Вставка трех рисунков
    \begin{figure}[H]
        \centering
        \includegraphics[width=#4\linewidth]{#1}
        \hfill
        \includegraphics[width=#4\linewidth]{#2}
        \hfill
        \includegraphics[width=#4\linewidth]{#3}
        \caption{#5} \label{#6}
    \end{figure}
}
\newcommand{\addimgapp}[2]{ % Это костыль для приложения Б
    \begin{figure}[H]
        \centering
        \includegraphics[width=1\linewidth]{#1}
        \caption*{#2}
    \end{figure}
}

% Заголовки секций в оглавлении в верхнем регистре
\usepackage{textcase}
\makeatletter
\let\oldcontentsline\contentsline
\def\contentsline#1#2{
    \expandafter\ifx\csname l@#1\endcsname\l@section
        \expandafter\@firstoftwo
    \else
        \expandafter\@secondoftwo
    \fi
    {\oldcontentsline{#1}{\MakeTextUppercase{#2}}}
    {\oldcontentsline{#1}{#2}}
}
\makeatother

% Оформление заголовков
\usepackage[compact,explicit]{titlesec}
\titleformat{\section}{}{}{0mm}{\clearpage\centering\bfseries{\thesection\quad\MakeTextUppercase{#1}}}
\titleformat{\subsection}[block]{\vspace{1em}}{}{12.5mm}{\thesubsection\quad#1}
\titleformat{\subsubsection}[block]{\vspace{1em}\normalsize}{}{12.5mm}{\thesubsubsection\quad#1}
\titleformat{\paragraph}[block]{\vspace{1em}\normalfont\normalsize}{}{12.5mm}{\theparagraph\quad#1}

\pdfstringdefDisableCommands{\let\uppercase\relax} % \uppercase не поддерживается в закладках pdf

% Секции без номеров (введение, заключение...), вместо section*{}
\newcommand{\anonsection}[1]{
    {\centering\bfseries{#1}\par}
    \phantomsection % Корректный переход по ссылкам в содержании
    \addcontentsline{toc}{section}{\uppercase{#1}}
}

% Секции для приложений
\newcommand{\appsection}[1]{
    {\centering\bfseries{#1}\par}
    \phantomsection
    \addcontentsline{toc}{section}{\uppercase{#1}}
}

% Библиография: отступы и межстрочный интервал
\makeatletter
\renewenvironment{thebibliography}[1]
    {\section*{\refname}
        \list{\@biblabel{\@arabic\c@enumiv}}
           {\settowidth\labelwidth{\@biblabel{#1}}
            \leftmargin\labelsep
            \itemindent 16.7mm
            \@openbib@code
            \usecounter{enumiv}
            \let\p@enumiv\@empty
            \renewcommand\theenumiv{\@arabic\c@enumiv}
        }
        \setlength{\itemsep}{0pt}
    }
\makeatother

\newenvironment{umlstyle}
{\hyphenpenalty 10000000
\footnotesize
\tikzumlset{font=\footnotesize}
\renewcommand{\baselinestretch}{1}
\begin{center}
\begin{tikzpicture}}
{\end{tikzpicture}
\end{center}}

\setcounter{page}{1} % Начало нумерации страниц
 % Подключаем преамбулу

\hypersetup{
    colorlinks, urlcolor={black}, % Все ссылки черного цвета, кликабельные
    linkcolor={black}, citecolor={black}, filecolor={black},
    pdfauthor={Воробйов Євгеній Андрійович},
    pdftitle={Проектування та розробка UI веб-додатку редагування розкладу}
}

\addbibresource{bibliography.bib} % Библиографический справочник


%%% Начало документа
\begin{document}
\thispagestyle{empty}

{\centering
МІНІСТЕРСТВО ОСВІТИ І НАУКИ УКРАЇНИ

ХЕРСОНСЬКИЙ ДЕРЖАВНИЙ УНІВЕРСИТЕТ

ФАКУЛЬТЕТ КОМП'ЮТЕРНИХ НАУК, ФІЗИКИ ТА ІНФОРМАТИКИ

КАФЕДРА ІНФОРМАТИКИ, ПРОГРАМНОЇ ІНЖЕНЕРІЇ ТА 

ЕКОНОМІЧНОЇ КІБЕРНЕТИКИ

\vfill

ПРОЕКТУВАННЯ ТА РОЗРОБКА UI ВЕБ-ДОДАТКУ РЕДАГУВАННЯ РОЗКЛАДУ

Дипломна робота

на здобуття ступеня вищої освіти бакалавр

}

\vfill

\hfill\begin{minipage}[t]{0.6\textwidth}
Виконав: 

студент 4 курсу 431 групи \\ спеціальності 6.040302  Інформатика

Воробйов Євгеній Андрійович

Керівник:

кандидат педагогічних наук, доцент

Круглик Владислав Сергійович

\end{minipage}

\vfill

{\centering
Херсон --- 2019

}

\tableofcontents % Содержание 
\clearpage

\anonsection{ВСТУП}

Якість підготовки спеціалістів у закладах освіти і особливо ефективність використання науково-педагогічного потенціалу залежать певною мірою від рівня організації навчального процесу.

Одна з головних складових цього процесу -- розклад занять -- регламентує трудовий ритм, впливає на творчу віддачу викладачів, тому його можна вважати фактором оптимізації використання обмежених ресурсів -- викладацького складу і аудиторного фонду.

Проблему складання розкладу слід розглядати не тільки як трудомісткий процес, об'єкт автоматизації з використанням комп’ютера, але і як проблему оптимального керування. 

Оскільки всі фактори, що впливають на розклад, практично неможливо врахувати, а інтереси учасників навчального процесу різноманітні, задача складання розкладу є багатокритеріальною з нечіткою множиною факторів.

Незалежно від алгоритму побудови розкладу, виникає прикладна проблема з інструментів різних рівнів, що використовуються в процесі. Саме ним і буде присвячено проведену роботу.

\textbf{Актуальність дослідження} полягає в необхідності забезпечення всіх учасників освітнього процесу доступом до актуальної версії розкладу занять у будь-який час, а також можливості спрощення процесу формування розкладу та подальшої інформатизації освітнього процесу.

\textbf{Об’єкт дослідження}~--- системи для планування та підтримки планування розкладу. \textbf{Предмет дослідження}~--- система для підтримки планування розкладу в закладах освіти з поділом учнів (вихованців, здобувачів освіти тощо) на стабільні академічні групи.

\textbf{Метою роботи} є проектування та розробка розширюваної системи підтримки редагування розкладу в закладах освіти з можливістю використання всіма учасниками освітнього процесу та реалізація відкритого API для взаємодії з системою.

Для реалізації мети поставлено наступні \textbf{завдання роботи}:
\begin{enumerate}
	\item Проаналізувати характеристики існуючих систем планування, зокрема обсяг їх можливостей.
	\item Проаналізувати окремі частини процесу підготовки розкладу на прикладі факультету комп'ютерних наук, фізики та математики ХДУ.
	\item На основі проведеного аналізу розробити вимоги щодо можливостей системи.
	\item Відповідно до створених вимог розробити серверну частину, зокрема реалізувати структуру бази даних та API.
	\item Розробити документацію до публічного API.
	\item Обґрунтувати використані технології при проектуванні серверної частини.
\end{enumerate}

Очікується, що спроектований продукт буде придатний до використання всіма учасниками освітнього процесу в ЗВО.

Робота складається з 2 розділів, містить \totalfigures\ рисунків. 
 % Введение

\section{АНАЛІЗ СИСТЕМ ПЛАНУВАННЯ ТА ІНФОРМУВАННЯ}

\subsection{Задачі систем планування} \subsection{Задачі систем планування}

Ідея планування робіт існує стільки, скільки існує людська цивілізація, адже ще в неоліті, з переходом до тваринництва і землеробства, постають задачі з контролем циклічних процесів, що і викликало у подальшому створення календаря і писемності для фіксування задач.


\subsection{Порівняння сервісів та додатків планування} На сьогодні, кожна людина так чи інакше стикається у повсякденному житті з системами, пов'язаними з контролем часу та завдань. Крім цього, такі технології знаходять застосування в освітньому процесі~\cite{ліщина2014проблеми}.

\subsubsection{Microsoft Outlook}

Microsoft Outlook — додаток-органайзер, входить до пакету офісних програм Microsoft Office. Дозволяє працювати з електронною поштою, надає функції календаря, планувальника завдань, записника і менеджера контактів. Крім того, Outlook дозволяє відстежувати роботу з документами пакету Microsoft Office для автоматичного складання щоденника роботи~\cite{франчук2016використання}.

Outlook може використовуватися  і як окремий додаток, так і виступати в ролі клієнта для Microsoft Exchange Server, що надає додаткові функції для спільної роботи всіх користувачів організації: загальні поштові скриньки, папки завдань, календарі, планування часу загальних зустрічей, узгодження документів тощо.

Крім цього, дозволяє підключати через протоколи POP3/IMAP інші поштові сервіси та додатки, що надаються ними. Зокрема, нижче буде розглянуто синхронізацію MS Outlook з сервісами Google.

\subsubsection{Lightning}

Lightning — проект Mozilla Foundation, що додає функції календаря і планувальника в Mozilla Thunderbird — безкоштовну кросплатформну програму для роботи з електронною поштою і новинами, що може вважатися відкритим аналогом для відповідних продуктів з пакету Microsoft Office.


\subsubsection{Google Calendar}

Google Calendar — безкоштовний веб-додаток для тайм-менеджменту розроблений Google. Інтерфейс подібний до аналогічних календарних додатків, таких як Microsoft Outlook. Має різні режими перегляду, зокрема денний, тижневий та місячний. Події зберігаються онлайн, а тому календар можна переглядати з будь-якого пристрою, обладнаного доступом до мережі Інтернет. Додаток може імпортувати та експортувати файли календаря різних форматів, а для існуючих — задавати різні права доступу~\cite{олексюк2013деякі}. 

Слід зазначити, що Google Calendar, як і інші сервіси Google, має відкрите API, що дозволяє взаємодіяти з ним через власні додатки після відповідних налаштувань.

Окремо слід звернути увагу на розвинені технології вбудовування документів Google (зокрема календарів Google Calendar) у власні веб додатки. 

Одним з прикладів такого використання в контексті розвитку інформаційної інфраструктури університету можна навести інтеграцію календаря подій факультету комп'ютерних наук, фізики та математики ХДУ в відповідну сторінку (kspu.edu/About/Faculty/FPhysMathemInformatics.aspx) на офіційному веб-сайті (рис.~\ref{fig:CalendarKspuEdu}).

\addimg{CalendarKspuEdu.png}{1}{Календар подій факультету}{fig:CalendarKspuEdu}

В наведеному прикладі події різних календарів, об'єднаних для відображення відображаються різними кольорами, в назву події включено час початку, а при натисканні на неї - відображаються деталі, зокрема опис, місце та посилання на подію в Google Calendar, де, крім іншого, можливо додати її для відслідковування та нагадування у власний календар, за умови, якщо користувач попередньо авторизувався в свій акаунт.

Документи, для яких встановлені публічні права для перегляду, можна включати в вихідний код сайту у вигляді фрейму. Фрейм — окремий HTML-документ, який сам чи разом з іншими документами відображений у вікні веб-переглядача. При цьому, всю відповідальність за відображуване в фреймі несе сервіс-власник, тобто Google Calendar в наведеному прикладі, а в місці відображення знаходиться лише код інтеграції з посиланням та супутніми параметрами~\cite{ліщина2014проблеми}.

Інший приклад використання мікросервісів (детальніше про мікросервіси в розділі~\ref{subsec:microservices}) та фреймів у контексті розвитку інформаційної інфраструктури  -- сервіс замовлення довідки про навчання в університеті (працює для студентів факультету комп'ютерних наук, фізики та математики, рис.~\ref{fig:RequestCertificateFromDeansOffice}).

\addimg{RequestCertificateFromDeansOffice.png}{0.75}{Форма замовлення довідки}{fig:RequestCertificateFromDeansOffice}

Після заповнення відповідної форми (не є справжньою формою на основі тегу <form> в розумінні HTML у зв'язку з обмеженнями сайту, проте реалізована з використанням його компонентів; поведінку цілком відтворено з допомогою javascript) відбувається запит до сервісу EmailJS.com, котрий, у свою чергу, надсилає лист за шаблоном на основі запиту з сайту. Сервіс має певні обмеження щодо кількості листів в проміжок часу, проце цілком задовільняє поставлені вимоги. Після отримання відповідного листа на корпоративну робочу пошту, відповідальний за підготовку довідок працівник деканату формує її в інформаційно-аналітичній системі університету  (детальніше в розділі~\ref{subsubs:KIS}) та виконує інші необхідні операції. В результаті зменшується навантаження на працівника та виключається необхідність  для студента у попередньому зверненні для запису; зменшується кількість <<паперових>> процесів що сприятливо впливає на подальшій інформатизції освітнього процесу.

При генерації коду фрейму для інтеграції у адміністратора є можливість налаштувати колірне оформлення фрейму, його розміри, регіональні стандарти (мову відображення, день початку тиждня, часовий пояс), обсяг за замовчуванням (тиждень, місяць), додати або приховати елементи керування. В процесі редагування налаштувань отримується невеликий за обсягом код (HTML тег <iframe>, рис.~\ref{fig:CalendarIframe}) для розміщення в коді власної веб-сторінки. 

\addCodeAsImg{\lstinputlisting[numbers=left]{code/CalendarIframe.tex}}{Код інтеграції календаря факультету}{fig:CalendarIframe}

Аналогічним чином інтегруються інші сервіси Google, що вже знайшло використання при розміщенні матеріалів на сайті, як то презентації, текстові документи, таблиці, карти, що одночасно підтверджує, по-перше, перспективність використання хмарних сервісів для поступового осучаснення інформаційної інфраструктури та, по друге, можливість переходу до використання їх замість звичних офісних пакетів (Microsoft Office, Open Office, Libre Office тощо).



\subsubsection{Корпоративні рішення}

\paragraph{Microsoft Outlook}

Microsoft Outlook — додаток-органайзер, входить в пакет офісних програм Microsoft Office. Дозволяє працювати з електронною поштою, надає функції календаря, планувальника завдань, записника і менеджера контактів. Крім того, Outlook дозволяє відстежувати роботу з документами пакету Microsoft Office для автоматичного складання щоденника роботи.

Outlook може використовуватися  і як окремий додаток, так і виступати в ролі клієнта для Microsoft Exchange Server, що надає додаткові функції для спільної роботи всіх користувачів організації: загальні поштові скриньки, папки завдань, календарі, планування часу загальних зустрічей, узгодження документів тощо.

Крім цього, дозволяє підключати через протоколи POP3/IMAP інші поштові сервіси та додатки, що надаються ними. Зокрема, нижче буде розглянуто синхронізацію MS Outlook з сервісами Google.

\paragraph{Lightning}

Lightning — проект Mozilla Foundation, що додає функції календаря і планувальника в Mozilla Thunderbird — безкоштовну кросплатформну програму для роботи з електронною поштою і новинами, що може вважатися відкритим аналогом для відповідних продуктів з пакету Microsoft Office.

\paragraph{Google Calendar}

Google Calendar — безкоштовний веб-додаток для тайм-менеджменту розроблений Google. Інтерфейс подібний до аналогічних календарних додатків, таких як Microsoft Outlook. Має різні режими перегляду, зокрема денний, тижневий та місячний. Події зберігаються онлайн, а тому календар можна переглядати з будь-якого пристрою, обладнаного доступом до мережі Інтернет. Додаток може імпортувати та експортувати файли календаря різних форматів, а для існуючих — задавати різні права доступу. 

Слід зазначити, що Google Calendar, як і інші сервіси Google, має відкрите API, що дозволяє взаємодіяти з ним через власні додатки після відповідних налаштувань.

Окремо слід звернути увагу на розвинені технології вбудовування документів Google (зокрема календарів Google Calendar) в власні веб додатки. 

Одним з прикладів такого використання в контексті розвитку інформаційної інфраструктури університету можна навести інтеграцію календаря подій факультету комп'ютерних наук, фізики та математики ХДУ в відповідну сторінку (kspu.edu/About/Faculty/FPhysMathemInformatics.aspx) на офіційному веб-сайті (рис.~\ref{fig:CalendarKspuEdu}).

\addimg{CalendarKspuEdu.png}{1}{Календар подій факультету}{fig:CalendarKspuEdu}

В наведеному прикладі події різних календарів, об'єднаних для відображення відображаються різними кольорами, в назву події включено час початку, а при натисканні на неї - відображаються деталі, зокрема опис, місце та посилання на подію в Google Calendar, де, крім іншого, можливо додати її для відслідковування та нагадування у власний календар, за умови, якщо користувач попередньо аворизувався в свій акаунт.

Документи, для яких встановлені публічні права для перегляду, можна включати в вихідний код сайту у вигляді фрейму. Фрейм — окремий HTML-документ, який сам чи разом з іншими документами відображений у вікні веб-переглядача. При цьому, всю відповідальність за відображуване в фреймі несе сервіс-власник, тобто Google Calendar в наведеному прикладі, а в місці відображення знаходиться лише код інтеграції з посиланням та супутніми параметрами.

Інший приклад використання мікросервісів та фреймів у контексті розвитку інформаційної інфраструктури  -- сервіс замовлення довідки про навчання в університеті (працює для студентів факультету комп'ютерних наук, фізики та математики, рис.~\ref{fig:RequestCertificateFromDeansOffice}).

\addimg{RequestCertificateFromDeansOffice.png}{0.75}{Форма замовлення довідки}{fig:RequestCertificateFromDeansOffice}

Після заповнення відповідної форми (не є справжньою формою на основі тегу <form> в розумінні HTML у зв'язку з обмеженнями сайту, проте реалізована з використанням його компонентів; поведінку цілком відтворено з допомогою javascript) відбувається запит до сервісу EmailJS.com, котрий, у свою чергу, надсилає лист за шаблоном на основі запиту з сайту. Сервіс має певні обмеження щодо кількості листів в проміжок часу, проце цілком задовільняє поставлені вимоги. Після отримання відповідного листа на корпоративну робочу пошту, відповідальний за підготовку довідок працівник деканату формує її в інформаційно-аналітичній системі університету  (детальніше в розділі~\ref{subsubs:KIS}) та виконує інші необхідні операції. В результаті зменшується навантаження на працівника та виключається необхідність  для студента у попередньому зверненні для запису; зменшується кількість <<паперових>> процесів що сприятливо впливає на подальшій інформатизції освітнього процесу.

При генерації коду фрейму для інтеграції у адміністратора є можливість налаштувати колірне оформлення фрейму, його розміри, регіональні стандарти (мову відображення, день початку тиждня, часовий пояс), обсяг за замовчуванням (тиждень, місяць), додати або приховати елементи керування. В процесі редагування налаштувань отримується невеликий за обсягом код (HTML тег <iframe>, рис.~\ref{fig:CalendarIframe}) для розміщення в коді власної веб-сторінки. 

\addCodeAsImg{\lstinputlisting[numbers=left]{code/CalendarIframe.html}}{Код інтеграції календаря факультету}{fig:CalendarIframe}

Аналогічним чином інтегруються інші сервіси Google, що вже знайшло використання при розміщенні матеріалів на сайті, як то презентації, текстові документи, таблиці, карти, що одночасно підтверджує, по перше, перспективність використання хмарних сервісів для поступового осучаснення інформаційної інфраструктури та, по друге, можливість переходу до використання їх замість звичних офісних пакетів (Microsoft Office, Open Office, Libre Office тощо).

TODO о мобильном приложении гугл календаря и виджете на телефоне, удобно, красиво, возможности, скрин сюда его!


\subsubsection{Мобільні додатки}

TODO посмотреть на каждый и по абзацу два ключевых вещей что он делает === перевод описания на украинский


\paragraph{Calendar+ Schedule Planner App}

https://play.google.com/store/apps/details?id=com.joshy21.vera.free.calendarplus

\paragraph{School Timetable - Class, University Plan Schedule}

https://play.google.com/store/apps/details?id=com.pranapps.schooltimetable

\paragraph{My Class Schedule: Timetable}

https://play.google.com/store/apps/details?id=de.rakuun.MyClassSchedule.free

\paragraph{Class Timetable}

https://play.google.com/store/apps/details?id=com.icemediacreative.timetable

\paragraph{Class Schedule – super broker of work}

https://play.google.com/store/apps/details?id=com.lixiangdong.classschedule

\paragraph{Class Planner}

https://play.google.com/store/apps/details?id=com.apps.ips.classplanner2

\paragraph{College Schedule Builder}

https://play.google.com/store/apps/details?id=com.kwidil.collegeschedulebuilder


\subsection{Корпоративні інформаційні системи} \label{subsubs:KIS}

Корпоративна інформаційна система — це інформаційна система, яка підтримує автоматизацію функцій управління на підприємстві і постачає інформацію для прийняття управлінських рішень. У ній реалізована управлінська ідеологія, яка об'єднує бізнес-стратегію підприємства і прогресивні інформаційні технології~\cite{hansvanderhoeven2011}.

У загальному визначенні «автоматизована система» — сукупність керованого об'єкта й автоматичних керувальних пристроїв, у якій частину функцій керування виконує людина. Вона представляє собою організаційно-технічну систему, що забезпечує вироблення рішень на основі автоматизації інформаційних процесів у різних сферах діяльності. 

Сучасні автоматизовані системи управління навчальним процесом у  закладах вищої освіти здатні вирішувати велику кількість функцій~\cite{співаковський2014побудова}, а саме:
\begin{itemize}
	\item планування, контроль та аналіз навчальної діяльності;
	\item оперативний доступ до інформації про навчальний процес;
	\item єдину систему звітів, як внутрішніх, так і за вимогами МОН України;
	\item системи безпеки даних з урахуванням вимог законодавства;
	\item облік контингенту студентів та співробітників;
	\item проведення вступної кампанії;
	\item формування пакетів даних з метою виготовлення тих чи інших документів.
\end{itemize}

Функціонування будь-якої автоматизованої системи можна швидко адаптувати до особливостей навчального процесу конкретного навчального закладу, до локальних мереж різного рівня, що допомагає розширити коло користувачів (адміністрації, викладачів і студентів) для оперативного забезпечення їх необхідною інформацією. 

Отже, використання таких систем дає змогу не тільки удосконалити якість планування навчального процесу, а й оперативність управління ним.

Не зважаючи на всі переваги, які надає використання автоматизованих систем, досі далеко не в кожному закладі вони впроваджені чи використовуються в повній мірі з тих чи інших причин — інерційності поглядів адміністрації, супротив працівників або «саботаж» на місцях, відсутність фінансової або організаційної можливості.

\subsubsection{Інформаційно-аналітична система} \label{subs:ias}

В ХДУ використовується корпоративна інтегрована система «Інформаційно-аналітична система (IAS)». Вона дозволяє вести облік працівників і студентів, бухгалтерський облік, контроль за матеріальними цінностями тощо (рис.~\ref{fig:IasSubsustem}). 

\addimg{IasSubsustem.png}{0.7}{Структура ІАС}{fig:IasSubsustem}
		
Система дозволяє вносити і ефективно стежити за будь-якими змінами. В основі системи лежить ядро, на основі ядра виконується розширення системи до будь-якої кількості компонентів. При цьому основна функціональність може бути розширена за рахунок додаткових компонентів~\cite{львов2007інформаційна}. 

Програма IAS орієнтована на платформу Windows з використанням MS SQL Server. Вона має багаторівневу архітектуру, що складається з бази даних, бізнес-логіки та клієнтського інтерфейсу. Внутрішній журнал реєстрації подій дозволяє вести та слідкувати за записами, що стосуються усіх подій.

Відсутність компонентів, пов’язаних з формуванням розкладу занять, та відсутність у використанні сторонніх рішень ставить задачу з проектування власного додатку для забезпечення всіх учасників освітнього процесу доступом до актуальної версії розкладу занять у будь-який час, а також можливості спрощення процесу формування розкладу та подальшої інформатизації освітнього процесу.

\subsection{Обґрунтування використаних технологій} У розробці веб-додатків на сьогодняшній день використувується велика кількість мов програмування, таких як: C\#, Java, JavaScript, Python, PHP. Кожен розробник обирає для себе ту мову, яка йому здається найбільш відповідною для певної задачі. В загалом усі вище перераховані мови, окрім, може JavaScript, традиційно використовуються для розробки бекенд-частини веб-ресурсів, або для генерації фронтенду на серверній стороні та відправки сгенерованої сторінки клієнту. Для кожної конкретної мови є певні фреймворки, що спрощують створення веб-додатків.

Фреймворк - це інфраструктура програмних рішень, що полегшує розробку складних систем.

Взагалом фреймворки можуть включати в себе бібліотеки, що створені для вирішення певних задач та найкращі сталі практики з вирішення тих чи інших питань. Головне завдання фреймворків - зменшити об'єм однотипної праці, що розробник додатку виконує у кожному своєму проекті. Хоча кожна мова програмування й має свої фреймворки, їх сутність у цілому залишається для певних задач доволі близькою: використання певних шаблонів проектування, що допомагають зробити програмний код більш зрозумілим та простим для подальшої підтримки та ускладнення.

Найбільш поширеними фреймворками є: для C\# є .NET, для Java - Spring, Hibernate, для JavaScript - Node, React, Vue, Angular, для Python - Django, для PHP - Laravel. Кожен фреймворк та мова програмування мають свої переваги та недоліки. Тому, коли перед нами постало завдання вибрати, на якому саме стеку технологій ми будемо розробляти веб-додаток, нами було проаналізовано кожну із вищепредставлених мов.

Наш вібір зупинився на JavaScript як для серверної, так і для клієнтської частини. Адже, якщо серверна та клієнтська частина написани з використанням однієї мови програмування, це сильно спрощує розробку та підримку веб-додатку. Для серверної частини нами було вирішено використовувати фреймворк Node.js, тому що він добре поєднується із будь-яким фронтенд фреймворком, має велике коло шанувальників, постійно оновлюється, що підвищує його безпечність, та хорошу документацію.  

\subsubsection{Семантичне версіювання}

В процесі розробки програмного забезпечення можливе виникнення проблеми під назвою <<пекло залежностей>>. 

Суть полягає в тому, що при збільшенні розмірів програмної системи, збільшується кількість бібліотек та пакетів, що використовуються в ній. При цьому, кожен з них, зазвичай, вимагає для своєї роботи деякі інші бібліотеки певних версій. У разі, якщо документація програмного забезпечення надто вільна, то рано чи піздно виникає проблема невідповідності між фактично необхідною версією, вказаною в документації та встановленою, що негативно позначається на всьому процесі розробки програмного забезпечення.

Для вирішення цієї проблеми пропонується простий набір правил і вимог, що визначають як встановлюються і збільнуються номери версій. Для роботи системи необхідно створити і описати публічне API програмного продукту. Після цього будь-які зміни в версії визначаються певною зміної її номера.

Розглянемо формат версій X.Y.Z (мажорна, мінорна, патч).

Зміни, що не впливають на API, збільнують патч-версію. Зворотньо-сумістні зміни та розширення API збільшують мінорну версію. І, нарешті, несумістні зміни API збільшують мажорну версію.

Ця система називатиметься <<Семантичне версіювання>>.

Мажорна версія <<0>> (0.Y.Z) призначена для початкової розробки, публічний API не має розглядатися як стабільний. Версія 1.0.0 визначає публічний API, після цього релізу вона змінюватиметься відповідно до змін в API. Після чергової зміни мінорної версії патч-версія змінюється на <<0>>, аналогічні зміни відбуваються зі зміною мажорної версії.

Крім зазначених правил, специфікація семантичного версіонування~\cite{semver} визначає додатково певні деталі та поради щодо його практичного використання, зокрема для продуктів, що мають складну систему релізів та передрелізних версій.

\subsubsection{Latex} \label{subsub:latex}

\TeX -- це створена чудовим американським математиком і програмістом Дональдом Кнутом система для верстки текстів з формулами. Сам по собі TEX є спеціалізованою мовою програмування (Кнут не тільки придумав мову, а й написав для нього транслятор, причому таким чином, що він працює абсолютно однаково на самих різних комп'ютерах), на якому пишуться видавничі системи, що використовуються на практиці. Точніше кажучи, кожна видавнича система на базі TEXа є пакетом макросів (макропакет) цієї мови. LATEX -- це створена Леслі Лампортом видавнича система на базі TEXа~\cite{львовский2003latex}.

Всі видавничі системи на базі TEXа володіють перевагами, закладеними в самому TEXе. Для новачка їх можна описати однією фразою: надрукований текст виглядає «зовсім як у книзі». LATEX, як видавнича система, надає зручні і гнучкі засоби досягти цього книжкового якості. Зокрема, вказавши за допомогою простих засобів структуру тексту, автор може не вникати в деталі оформлення, причому ці деталі при необхідності неважко змінити (щоб, скажімо, змінити шрифт, яким друкуються заголовки, не треба нишпорити по всьому тексту, змінюючи все заголовки , а досить замінити одну сходинку в «стильовому файлі»). Такі речі, як нумерація розділів, посилання, зміст і т. П. Виходять майже що «самі собою». Величезним плюсом систем на базі TEXа є висока якість та гнучкість форматування абзаців і математичних формул (в останньому відношенні краще TEXа цю задачу не вирішує жодна програма).

TEX (і всі видавничі системи на його базі) невибагливий до використовуваної техніки. З іншого сторони, TEXовські файли мають високий ступінь переносимості: Ви можете підготувати LATEXовський вихідний текст на своєму IBM PC, переслати його до видавництва, і бути впевненими, що там Ваш текст буде правильно оброблений і на друку вийде в точності те ж, що вийшло у Вас при пробному друку на Вашому улюбленому матричному принтері (з тією єдиною різницею, що фотоскладальний автомат дасть текст більш високої якості). Завдяки цій обставині TEX став дуже популярний як мова міжнародного обміну статтями з математики та фізики.

LaTeX - це високоякісна набірна система; він включає функції, призначені для виготовлення технічної та наукової документації. LaTeX є фактичним стандартом для комунікації та публікації наукових документів \cite{lamport1994latex}. LaTeX доступний як вільне програмне забезпечення.

При роботі над звітом також використано сервіс Overleaf -- сучасний інструмент, розроблений у 2012 році. Він був створений щоб допомогти редагувати свої наукові статті, технічні звіти, тези, презентації, блок-схеми та інші документи, написані на мові розмітки LaTeX. 

При цьому, було використано всі переваги хмарних технологій, в тому числі можливість миттєвого початку роботи на практично будь-якому комп'ютері, збереження версій та одночасної роботи над проектом кількох користувачів.

Також нівелюється необхідність у встановленні на комп'ютері додаткового програмного забезпечення, що може бути названим одним із недоліків використання окремої системи, як LaTeX.

\subsubsection{Система контролю версій Git}

Активну популярність мають розподілені системи контролю версій (SCM).

Найбільш поширеними з таких є Subversion (SVN), Microsoft Visual Source Safe (VSS), Revision Control System (RCS), Concurrent Versions System (CVS), Gіt та Mercurіal. Знання подібних систем підвищує затребуваність ІT фахівців на ринку праці, покращує продуктивність розробників та полегшує рішення щоденних завдань. Саме передача знань є вирішальною у процесі експорту-імпорту технологій~\cite{киричек2012модель}.

В процесі роботи використано систему контролю версій Git з віддаленим репозиторієм на сервісі GitHub~\cite{gCalShedule}.

Система контролю дозволяє зберігати попередні версії файлів та завантажувати їх за потребою. Вона зберігає повну інформацію про версію кожного з файлів, а також повну структуру проекту на всіх стадіях розробки. Місце зберігання даних файлів називають репозиторієм. В середині кожного з репозиторіїв можуть бути створені паралельні лінії розробки — гілки.

Git підтримує швидке розділення та злиття версій, містить можливості для візуалізації і навігації за нелінійною історією розробки. 

\subsubsection{Sublime Text}

Sublime Text - прорієтарний текстовий редактор. Підтримує плагіни на мові програмування Python.

Розробник дає можливість безкоштовно і без обмежень ознайомитися з редактором, однак програма періодично буде повідомляти про необхідність придбання ліцензії.

Редактор містить різні візуальні теми, а також можливість завантаження додаткових тем.

Коли користувач набере код, Sublime Text, в залежності від використовуваного мови, буде пропонувати різні варіанти для завершення запису. Також редактор може автоматично додавати розділові знаки (<<\{>>, <<\}>>, <<;>>).

Sublime Text дозволяє збирати програми у готовий проект і запускати їх без необхідності використання зовнішньої командної строки. Користувач також може налаштувати свою систему компіляції і включити автоматичну збірку програм кожного разу при збереженні коду. Ця система схожа з відповідним плагіном для зйомки тексту від віддаленого LaTeX (розділ~\ref{subsub:latex}).

Використовується плагін LaTeXTools. Плагін LaTeXTools надає кілька функцій, які спрощують роботу з файлами LaTeX.

Команда ST збирає компіляцію джерела LaTeX у PDF за допомогою texify (Windows / MikTeX) або latexmk (OSX / MacTeX, Windows / TeXlive, Linux / TeXlive). Потім він розбирає файл журналу і перераховує помилки та попередження. Нарешті, він запускає програму перегляду PDF і, на підтримуваних переглядачах (Sumatra PDF на Windows, Skim на OSX і Evince на Linux за замовчуванням) переходить до поточної позиції курсора.

Додатково реалізована функція автозбереження, що допомагає користувачам не втратити пророблену роботу.


\subsubsection{Бібліотеки JS}

\paragraph{bcrypt.js}

Optimized bcrypt in JavaScript with zero dependencies. Compatible to the C++ bcrypt binding on node.js and also working in the browser.

Security considerations
Besides incorporating a salt to protect against rainbow table attacks, bcrypt is an adaptive function: over time, the iteration count can be increased to make it slower, so it remains resistant to brute-force search attacks even with increasing computation power. (see)

While bcrypt.js is compatible to the C++ bcrypt binding, it is written in pure JavaScript and thus slower (about 30\%), effectively reducing the number of iterations that can be processed in an equal time span.

The maximum input length is 72 bytes (note that UTF8 encoded characters use up to 4 bytes) and the length of generated hashes is 60 characters.

\paragraph{Express}

При розробці використано бібліотеку Express — гнучкий фреймворк для веб-застосунків, побудованих на Node.js, що надає широкий набір функціональності, полегшуючи створення надійних API.

Express забезпечує тонкий прошарок базової функціональності для веб-застосунків, що не спотворює звичну та зручну функціональність Node.js., при отриманні запиту він оброблюватиметься відповідно до визначення маршруту (рис.~\ref{fig:Route}), де app є екземпляром express, METHOD є методом HTTP-запиту, PATH є шляхом на сервері, HANDLER є функцією-обробником, що спрацьовує, коли даний маршрут затверджено як співпадаючий.

\paragraph{JSON Web Token} \label{subsubsection:jwt}

Для забезпечення конфіденційності при обміні даними використовується JSON Web Token. Деталі роботи з ним розглянуто в розділі \ref{subsubsection:jwt}. Для роботи з JSON Web Token використовується бібліотека jsonwebtoken.

\paragraph{pg-hstore}

A node package for serializing and deserializing JSON data to hstore format



\clearpage
\section{ПРОЕКТУВАННЯ ТА РОЗРОБКА КЛІЄНТСЬКОЇ ЧАСТИНИ}

\subsection{Розробка SPA використовуючи бібліотеку React}
\subsubsection{Аналіз існуючих бібліотек для розробки SPA}

На сьогодні, одними з росповсюджених фреймворків для розробки SPA (single page applications) є React, Angular та Vue (рис.~\ref{fig:ReactAngularVue}).

\addimg{ReactAngularVue.png}{0.35}{React, Angular та Vue}{fig:ReactAngularVue}

Angular - Javascript-фреймворк, створений на основі TypeScript. Розроблений і підтримуваний компанією Google, він описується як JavaScript MVW-фреймворк. На даний момент останньою версією є 4. Фреймворк Angular використовується такими компаніями, як Google, Wix, weather.com, healthcare.gov і Forbes.

Vue - ще один JS-фреймворк. Творці Vue описують його як «інтуїтивно зрозумілий та швидкий, призначений для створення інтерактивних інтерфейсів». Вперше він був представлений колишнім співробітником компанії Google Еваном Ю (Evan You) в лютому 2014 року. На даний момент фреймворк використовується такими компаніями, як Alibaba, Baidu, Expedia, Nintendo, GitLab.

При розробці мобільного додатку використано фреймворк React Native, в основі якого знаходиться бібліотека React, призначена для створення користувацьких інтерфейсів. На відміну від React, призначеного для розробки односторінкових веб-додатків, React Native спрямований на мобільні платформи.

React — відкрита JavaScript бібліотека для створення інтерфейсів користувача, яка покликана вирішувати проблеми часткового оновлення вмісту веб-сторінки, з якими стикаються в розробці односторінкових застосунків. React використовують Facrbook, Airbnb, Uber, Netflix, Twitter, Pinterest, Reddit, Udemy, Wix, Paypal, Imgur, Feedly, Stripe, Tumblr, Walmart та інші.

React дозволяє розробникам створювати великі веб-додатки, які використовують дані, котрі змінюються з часом, без перезавантаження сторінки. React обробляє тільки користувацький інтерфейс у застосунках. Це відповідає видові у шаблоні модель-вид-контролер (MVC) і може бути використане у поєднанні з іншими JavaScript бібліотеками або в великих фреймворках MVC, таких як AngularJS. Він також може бути використаний з React на основі надбудов, щоб піклуватися про частини без користувацького інтерфейсу побудови веб-застосунків.

React підтримує віртуальний DOM, а не покладається виключно на DOM браузера. Це дозволяє бібліотеці визначити, які частини DOM змінилися, порівняно зі збереженою версією віртуального DOM, і таким чином визначити, як найефективніше оновити DOM браузера. Як бібліотека інтерфейсу користувача React часто використовується разом з іншими бібліотеками, такими як Redux, проте у його використанні при розробці проекту не було необхідності.

React надає розробникам безліч методів, які викликаються під час життєвого циклу компонента (рис.~\ref{fig:ReactReduxCommunicate}), вони дозволяють нам оновлювати UI і стан додатку. Коли необхідно використовувати кожен з них, що необхідно робити і в яких методах, а від чого краще відмовитися, є ключовим моментом до розуміння як працювати з React.

\addimg{ReactReduxCommunicate.png}{0.85}{Взаємодія Redux та React}{fig:ReactReduxCommunicate}

Конструктори є основною ООП — це спеціальна функція, яка буде викликатися щоразу, коли створюється новий об'єкт. Важливо викликати функцію super в випадках, коли наш клас розширює поведінку іншого класу, який має конструктор. Виконання цієї спеціальної функції буде викликати конструктор нашого батьківського класу і дозволяти йому проініціаліззувати себе. 
Конструктори —  це відмінне місце для ініціалізації компонента —  створення будь-яких полів (змінні, що починаються з this.).

Це також єдине місце де слід встановлювати стан безпосередньо перезаписуючи поле this.state. У всіх інших випадках необхідно використовувати this.setState.

За замовчуванням, всі компоненти будуть перемальовувати себе всякий раз, коли їх стан змінюється, змінюється контекст або вони приймають props від батьківського компонента. Якщо перерисовка компонента досить важка (наприклад генерація графіка), то у розробників є доступ до спеціальної функції, яка дозволяє контролювати цей процес.

\subsubsection{Webpack}

При створенні сайту досить стандартною практикою є мати певний процес збірки на місці, щоб полегшити розробку і підготовку файлів до роботи.

Можливо використовувати Grunt або Gulp, побудувавши ланцюжки перетворень, які дадуть можливість подати код в один кінець ланцюжка і отримати мінімізовані CSS та JavaScript на іншому.

Подібні інструменти розробки досить популярні і корисні в наші дні. Проте, є й інший метод полегшення розробки~--- Webpack.

Webpack є так званим «збиральником модулів». Він приймає модулі JavaScript, аналізує їх залежності один від одного, а потім з'єднує їх найефективнішим способом, випускаючи у кінці лише один JavaScript файл~\cite{juhovepsalainen2016}.

З Webpack, модулі не обмежені тільки файлами JavaScript. Завдяки частині loaders, Webpack розуміє, що модуль JavaScript може потребувати CSS файл, а цей CSS файл може потребувати зображення. Результат роботи Webpack буде містити тільки те, що потрібно у проекті.

\subsubsection{Babel}

Нажаль, при постійному розвитку мов програмування невід'ємною є ситуація, що реалізація часто відстає від специфікації. Більш того, різні реалізації по-різному відстають від специфікації. Написавши код, ми не можемо гарантувати, де він буде запускатися, а де~--- ні.

Виходячи з цього можна зробити висновок, що потрібно писати код, дотримуючись старих стандартів. На щастя, є інший шлях: ми можемо писати код з використанням всіх найновіших можливостей, але перед публікацією автоматично транслювати його (тобто переводити з одного виду в інший) в стару версію. 

Сама природа JS і його способи використання готують нас до того, що ніколи не настане моменту, коли у всіх користувачів буде остання версія інтерпретатору. Люди використовували і продовжать використовувати різні браузери і різні версії браузерів, різні версії Node.js і так далі. Використання нових синтаксичних конструкцій в такій ситуації практично неможливо. Запуск коду на платформі що не підтримує новий синтаксис призведе до синтаксичної помилки. 

Закономірним вирішенням цієї проблеми стала поява Babel~--- програми, яка бере вказаний код і повертає той же код, але трансльований в стару версію JS. Фактично, в сучасному світі Babel став невід'ємною частиною JS. Всі нові проекти так чи інакше розробляють з його використанням~\cite{davidgeary2019}.

\subsubsection{Розробка компонентів}

В термінах React, всі частини-відображення іменуються компонентами. В роботі спроектована серія компонентів для різних частин системи та розроблено прототипи деяких з них. 

На рис.~\ref{fig:AdminPanelUserCreation} зображено компонент для створення адміністратором нового користувача системи. Слід наголосити, що кожен компонент є окремою частиною і тому можливий для використання у подальшому в інших системах при виконанні певних вимог (так зване «повторне використання»).

\addimg{AdminPanelUserCreation.png}{0.85}{Адміністративна панель (створення користувача)}{fig:AdminPanelUserCreation}

На приведеному вище зображенні натискання на кожну з кнопок призводить до виклику відповідного методу API шляхом надсилання певного HTTP запиту.

Перед доступом до адміністративної панелі (рис.~\ref{fig:AdminPanelUserManagement}) адміністратору необхідно авторизуватися у системі, в результаті чого буде створено і збережено силами його веб-браузеру JWT. Після цього, якщо він має відповідні права  та верифікація токену пройшла успішно, йому буде відображена відповідна панель.

\addimg{AdminPanelUserManagement.png}{0.85}{Адміністративна панель (керування користувачами)}{fig:AdminPanelUserManagement}

Слід зауважити, що подібний процес перевірки відбувається при виконанні кожного запиту, крім тих, що не потребують авторизації (доступні для незареєстрованих користувачів).

Перша прерогатива адміністратора системи~--- створення нових користувачів, при цьому відбувається вищеописана процедура.

Певна частина об’єктів системи може вважатися більш-менш константною, це структура закладу вищої освіти (рис.~\ref{fig:AdminPanelFacultyManagement}), окремі словникові дані (зокрема, назви посад професорсько-викладацького складу та види занять).

\addimg{AdminPanelFacultyManagement.png}{0.85}{Адміністративна панель (керування факультетами)}{fig:AdminPanelFacultyManagement}

Окремо можна відзначити використання об’єктів часу (номера занять впродовж дня, рис.~\ref{fig:AdminPanelTimeManagement}).  В спроектованій системі одним з можливих шляхів доступу до розкладу є його експорт до сервісу Google Calendar в серію календарів. 

\addimg{AdminPanelTimeManagement.png}{0.85}{Адміністративна панель (керування часом)}{fig:AdminPanelTimeManagement}

В подальшому, потенційні користувачі можуть підписуватися на оновлення відповідних календарів (зокрема, груп та викладачів) для отримання актуальної інформації в довільний момент часу, котру підтримуватимуть користувачі системи з привілегією редагування даних про розклад (рис.~\ref{fig:AdminPanelScheduleEdit}).

\addimg{ScheduleEdit.png}{0.95}{Адміністративна панель (редагування розкладу)}{fig:AdminPanelScheduleEdit}

В процесі розробки було підготовлено серію компонентів, кожен з яких відповідає за певну конкретну задачу та може використовуватися в багатьох місцях додатку (так зване <<повторне використання>>). На рис.~\ref{fig:ReactComponent} представлено один із компонентів, що відображає списки у системі.

Представлений компонент відноситься до компонентів-відображень. При цьму, компоненти діляться на декілька груп. 

Компоненти-контейнери відповідають за дані і операції з ними. Їх стан передається у вигляді властивостей в компоненти-відображення.

Компоненти-відображення є <<дурними>> в тому сенсі, що вони не представляють, звідки беруться дані, тобто вони нічого не знають про стан.

Компоненти-відображення не повинні змінювати дані. Фактично, будь-який компонент, який одержує властивості від батьківського компоненту, повинен залишати їх незмінними. У той же час, вони можуть будь-яким чином форматувати дані (наприклад, конвертуючи Unix timestamp в дату-час для відображення).

Компоненти-контейнери найчастіше є батьківськими для компонентів-відображень і забезпечують зв'язок між відображенням і іншими частинами програми. Їх також називають «розумними» компонентами, оскільки вони <<знають>> про програму в цілому.

\addCodeAsImg{\lstinputlisting[numbers=left]{code/ReactComponent.tex}}{React компонент відображення списку}{fig:ReactComponent}


\subsection{Розробка мобільного додатку}
\subsubsection{Вимоги до додатку та системи виконання}

Одним із завдань, поставлених для реалізації мети є розроблення вимог щодо можливостей мобільного додатку та його інтерфейсу.

Після проведення аналізу предметної області, додатків аналогів та технологій було сформульовано наступні вимоги:

\begin{enumerate}
    \item Мобільний додаток повинен виконуватися на платформі $ios$.
    \item Можливість вибору та збереження групи користувача.
    \item Перегляд поточного розкладу.
    \item Перегляд розкладу в конкретний день тижня в минулому або майбутньому. 
    \item Функція сповіщення про зміни у розклади користувача.
    \item Можливість пошуку та перегляду розкладу занять викладачів.
    \item Можливість редагування збереженої інформації про користувача додатку.
\end{enumerate}

\subsubsection{React Native}

\addimg{ReactNative.png}{0.7}{Структура React Naive}{fig:ReactNative}

React Native~--- це JS-фреймворк для створення нативних iOS і Android додатків. В його основі лежить розроблена в Facebook JS-бібліотека React (проаналізовано та описано в пункті \ref{subs:React}), призначена для створення користувацьких інтерфейсів. Але замість браузерів вона орієнтована на мобільні платформи. Іншими словами, якщо ви веб-розробник, то можете використовувати React Native для написання  швидких мобільних додатків, з комфортом використання у розробці звичного фреймворка і єдиної кодової бази JavaScript~\cite{jeffgothelf2016}.

Популярність React має ряд причин. Вона компактна і має високу продуктивність, особливо при роботі з швидкоплинними даними. Завдяки компонентної структурі, React заохочує до написання модульного коду и пропагує повторне його використання.

Спроектовано та розроблено серію компонентів для різних частин додатку. На рис.~\ref{fig:interphace} зображено сукупність компонентів, яка формує стартовове вікна додатку, у якому користувач може брати номер своєї групи для отримання розкладу (зліва) та сукупність компонентів, яка формує вікно відображення поточного розкладу з можливістю обрання конкретної дати для відображення (справа). Перший компонент зберігає номер групи локально на мобільному пристрої завдяки використанню функції $AsynkStorage()$, котра надається платформою React Native (код компоненту на рис.~\ref{fig:ReactNativeCode}).

\addtwoimghere{Mobile1.png}{Mobile1.png}{0.45}{Інтерфейс мобільного додатку}{fig:interphace}

React Native~--- це той же React, але для мобільних платформ. У нього є ряд відмінностей: замість тега $div$ використовується компонент $View$, а замість тега $img$~--- $Image$. Вам може стати в нагоді знання Objective-C, Swift або Java~\cite{9781787282537}. 

Нативну (native) розробку можна назвати <<рідною>> для операційних систем~--- Android, IOS, Win Phone і т.д. Такі мобільні додатки пишуться на мовах програмування, затверджених розробниками програмного забезпечення під кожну конкретну платформу, а тому органічно вбудовуються в самі операційні системи.

У React компонент описує власне відображення, а потім бібліотека обробляє для вас рендеринг (деталі у пункті \ref{fig:ReactNative}). Ці дві функції розділені ясним рівнем абстракції. Якщо потрібно відобразити компоненти для вебу, то React використовує стандартні HTML-теги. Завдяки тому ж рівню абстракції~--- <<мосту>>~--- для рендеринга в iOS і Android React Native викликаються відповідні API.

\addCodeAsImg{\lstinputlisting[numbers=left]{code/ReactNativeCode.tex}}{React Native компонент стартового вікна}{fig:ReactNativeCode}

Замість компіляції в нативний код, React Native запускає його за допомогою JS-движка хост-платформи, без блокування основного UI-потоку. Ви отримуєте переваги нативних продуктивності, анімації і поведінки без необхідності писати на Objective-C або Java. Інші методи розробки кроссплатформенних додатків, на кшталт Cordova або Titanium, ніколи не досягнуть такого рівня нативної продуктивності або відображення.

Головна перевага нативних додатків~--- то, що вони оптимізовані під конкретні операційні системи, а значить працюють коректно і швидко. Також вони мають доступ до апаратної частини пристроїв, тобто можуть використовувати в своєму функціоналі камеру смартфона, мікрофон, акселерометр, геолокацію, адресну книгу, плеєр і т.д. Можна налаштувати отримання push-повідомлень. Ще один плюс - економну витрату ресурсів телефону (батарея, пам'ять).

Головна особливість React Native~--- він «дійсно» нативний. Інші рішення JavaScript для мобільних платформ просто обертають ваш JS-код в веб-відображення. Вони можуть перереалізовать яке-небудь нативное поведінку інтерфейсу, наприклад, анімацію, але все ж  залишаються веб-додатоком.

\paragraph{Переваги для розробника}

У порівнянні зі стандартною розробкою під iOS і Android, React Native має набагато більше переваг. Оскільки додаток здебільшого складається з JavaScript, можливо користуватися численними перевагами веб-розробки. Наприклад, щоб побачити внесені в код зміни, можна миттєво «оновити» додаток замість тривалого очікування завершення традиційної компіляції~\cite{robinwieruch2018}. 

Крім того, React Native надає «розумну» систему повідомлень про помилки і стандартні інструменти налагодження JavaScript, що сильно полегшує процес мобільного розробки.

\paragraph{Обробка декількох платформ}

React Native витончено обробляє різні платформи. Переважна більшість API у фреймворку~--- кросплатформені, так що досить просто написати компонент React Native, і він буде без проблем працювати на iOS і Android платформах.

Якщо вам потрібно писати залежний від платформи код - в зв'язку з різними правилами взаємодії в iOS і Android, або через переваг платформозалежного API~--- то з цим не буде труднощів. React Native дозволяє призначати платформозалежні версії кожного компонента, які ви можете потім інтегрувати в свій додаток.


\subsection{Менеджери стану}
\subsubsection{Порівняння менеджерів стану}

\paragraph{Flux}

Flux-архітектура -- архітектурний підхід або набір шаблонів програмування для побудови призначеного для користувача інтерфейсу веб-додатків, що поєднується з реактивним програмуванням і побудований на односпрямованих потоках даних.

Згідно із задумом творців і незважаючи на те, що Facebook надав реалізацію Flux на додаток до React, Flux не є ще одним веб-фреймворком, а є архітектурним рішенням~\cite{petrenko2015порівняння}.

Основною відмінною рисою Flux є одностороння спрямованість передачі даних між компонентами Flux-архітектури. Архітектура накладає обмеження на потік даних, зокрема, виключаючи можливість поновлення стану компонентів самими собою. Такий підхід робить потік даних передбачуваним і дозволяє легше простежити причини можливих помилок в програмному забезпеченні.

У мінімальному варіанті Flux-архітектура може містити три шари, які взаємодіють один по одному:

\begin{enumerate}
    \item actions (дії);
    \item stores (сховища);
    \item views (подання).
\end{enumerate}

Хоча зазвичай між діями і сховищами додають Dispatcher (диспетчер).

В першу чергу Flux працює з інформаційною архітектурою, яка потім відбивається в архітектурі програмного забезпечення, тому рівень уявлень слабо зачеплений з іншими рівнями системи.

\subparagraph{Дії}
Дії (англ. Actions) -- вираз подій (часто для дій використовуються просто імена - рядки, що містять деяке «дієслово»). Диспатчери передають дії нижчого компонентів (сховищ) по одному. Нова дія не передається поки попереднє повністю не оброблено компонентами. Дії через роботу джерела дії, наприклад, користувача, надходять асинхронно, але їх диспетчеризація явлется синхронним процесом. Крім імені (англ. Name), дії можуть мати корисне навантаження (англ. Payload), що містить пов'язані з дією дані.

\subparagraph{Диспетчер}
Диспетчер (англ. Dispatcher) призначений для передачі дій сховищ. У спрощеному варіанті диспетчер може взагалі не виділятися, як єдиний на весь додаток. У диспетчері сховища реєструють свої функції зворотного виклику (callback) і залежності між сховищами.

\subparagraph{Сховища}
Сховище (англ. Store) є місцем, де зосереджено стан (англ. State) додатку. Інші компоненти, згідно Flux, не мають значного (з точки зору архітектури) стану. Зміна стану сховища відбувається строго на основі даних дії і старого стану сховища.

\subparagraph{Вид}
Вид (англ. View) -- компонент, звичайно відповідає за видачу інформації користувачеві. У Flux-архітектурі, яка може технічно не торкатися внутрішнього облаштування уявлень взагалі, це - кінцева точка потоків даних. Для інформаційної архітектури важливо тільки, що дані потрапляють в систему (тобто, назад в сховища) тільки через дії.

\paragraph{Redux}

Redux - це інструмент управління як станом даних, так і станом інтерфейсу в JavaScript-додатках. Він підходить для односторінкових додатків, в яких управління станом може з часом стає складним. Redux не пов'язаний з якимось певним фреймворком, і хоча розроблявся для React, може використовуватися з Angular або jQuery.

У React не рекомендується реалізовувати пряму взаємодію компонент-компонент. Це вважається поганою практикою, призводить до помилок і заплутаність коду. Redux пропонує зберігати все стан додатки в одному місці, званому «store» («сховище»). Компоненти «відправляють» зміну стану в сховище, а не безпосередньо до інших компонентів. Компоненти, які повинні бути в курсі цих змін, «підписуються» на сховище.

Сховище може розглядатися як «посередник» у всіх змінах стану в додатку. З Redux компоненти не зв'язуються один з одним безпосередньо, всі зміни повинні пройти через єдине джерело істини, через сховище.

C Redux всі компоненти отримують свої дані зі сховища. Також ясно, куди компонент повинен відправити інформацію про зміну стану - знову ж в сховище. Компонент тільки ініціює зміну і не піклується про інші компоненти, які повинні отримати цю зміну. Таким чином, Redux робить потік даних більш зрозумілим.

Загальна концепція використання сховищ для координації стану програми - це шаблон, відомий як Flux. Цей шаблон проектування доповнює односпрямований потік даних як в React.

\paragraph{MobX}

MobX - це бібліотека, яка робить управління станом додатку простим і масштабується, застосовуючи функціонально-реактивне програмування~\cite{mezzalira2018mobx}. 

React і Mobx разом - потужна комбінація. React видображає стан додатку, надаючи механізми для перекладу його в дерево відображення компонентів. MobX надає механізм зберігання та оновлення стану, який потім може використовувати React.

React + MobX предcтавляют оптимальне рішення загальних проблем розробки додатків~\cite{9781789344837}. React дає можливість оптимально відображати інтерфейс за допомогою Virtual DOM, що зменшує кількість дорогих мутацій оригінального DOM. MobX дозволяє синхронізувати стан між React-компонентами, використовуючи реактивну віртуальну залежність графічного стану, яке оновлюється лише коли це дійсно потрібно.

MobX додає можливість спостерігати за існуючими структурами даних, такими як об'єкти, масиви і екземпляри класів. Це досить просто зробити, обернувши необхідну властивість класу за допомогою декоратора $@observable$ (рис.~\ref{fig:Code10}).

\addCodeAsImg{\lstinputlisting[numbers=left]{code/Code10.tex}}{Приклад використання декоратора $@observable$}{fig:Code10}

Використання $@observable$ схоже на перетворення властивості об'єкта в клітинку електронної таблиці. Але на відміну від електронних таблиць ці властивості можуть бути не тільки примітивними значеннями, але також посиланнями, об'єктами і масивами.

Значеннями, що спостерігаються, можуть бути JS-примітиви, посилання, об'єкти, екземпляри класів, масиви і т.д. Якщо значенням є масив або об'єкт, при виведенні змінної буде повернутий $Observable Array$ або $Observable Object$ відповідно.

\subsubsection{Redux}

Redux -- це інструмент управління як станом даних, так і станом інтерфейсу в JavaScript-додатках. Він підходить для односторінкових додатків, в яких управління станом з часом може ускладнюватися. Redux не пов'язаний з якимось певним фреймворком, і хоча розроблявся для React, може використовуватися з Angular або jQuery.

C Redux всі компоненти отримують свої дані зі сховища. Також зрозуміло, куди компонент повинен відправити інформацію про зміну стану — знову ж в сховище. Компонент тільки ініціює зміну і не піклується про інших компонентах, які повинні отримати цю зміну. Таким чином, Redux робить потік даних більш зрозумілим.

Загальна концепція використання сховищ для координації стану програми — це шаблон, відомий як Flux. Цей шаблон проектування доповнює односпрямований потік даних як в React.

Redux використовує тільки одне сховище для всього стану програми. Оскільки стан знаходиться в одному місці, його називає єдиним джерелом істини. Структура даних сховища повністю залежить від вас, але для реального застосування це, як правило, об'єкт з декількома рівнями укладення.

Такий підхід єдиного сховища є основною відмінністю між Redux і Flux з його численними сховищами~\cite{9781617294976}.

Згідно з документацією Redux, <<Єдиний спосіб змінити стан - передати action — об'єкт, що описує, що сталося>>. Це означає, що програма не може безпосередньо змінити стан. Замість цього, необхідно передати $action$, щоб висловити намір змінити стан в сховищі (рис.~\ref{fig:ReactReduxCommunicate}).

\subsubsection{Обробка побічних ефектів}

\paragraph{Порівняння бібліотек}

Redux-thunk і Redux-saga -- обидві бібліотеки для обробки побічних ефектів при розробці з використанням бібліотеки Redux~\cite{mezzalira2018mobx}. Проміжні обробники Redux - це код, який перехоплює дії, що надходять у сховище, за допомогою методу $dispatch ()$.


Крім диспетчеризації стандартних дій, Redux-Thunk middleware дозволяє розсилати спеціальні функції, які називаються $thunks$.

Thunks (в Redux) зазвичай мають структуру, представлену на рис.~\ref{fig:Code11}.

\addCodeAsImg{\lstinputlisting[numbers=left]{code/Code11.tex}}{Структура Thunks в Redux}{fig:Code11}

Тобто thunk - це функція, яка (необов'язково) приймає деякі параметри і повертає іншу функцію. Внутрішня функція приймає функцію диспетчеризації та функцію $getState$ - обидві з них будуть поставлятися за допомогою бібліотеки Redux-Thunk.

Redux-Saga middleware дозволяє висловлювати складну логіку застосування як чисті функції, що називаються сагами. Чисті функції бажані з точки зору тестування, тому що вони передбачувані і повторювані, що робить їх порівняно легкими для перевірки~\cite{hung2018architectural}.

Саги реалізуються за допомогою спеціальних функцій, званих функціями генератора. Це нова функція ES6 JavaScript. Усередині функції-генератора є ключове слово yield з синтаксисом, схожим на return. Відмінність в тому, що функція (у тому числі функція-генератор) може повернути значення тільки один раз, але віддати значення функція-генератор може скільки завгодно раз. Вираз yield призупиняє виконання генератора, так що його можна пізніше відновити.

Функція генератора представлена на рис.~\ref{fig:Code12}. Зверніть увагу на зірочку після ключового слова функції.

\addCodeAsImg{\lstinputlisting[numbers=left]{code/Code12.tex}}{Структура функції генератора}{fig:Code12}

\paragraph{Redux Thunk}

TODO

\subsection{Контроль якості програмного забезпечення}
\subsubsection{Задачі контролю якості}

При написанні функції зазвичай уявляється, що вона повинна робити, яке значення на яких аргументах видавати.

У процесі розробки час від часу перевіряється, чи правильно працює функція. Найпростіший спосіб перевірити~--- це запустити її, наприклад в консолі, і подивитися результат. Але такі ручні запуски~--- дуже недосконалий засіб перевірки.

Коли перевіряєтья робота коду вручну~--- легко його «недотестувати».

Наприклад, пишемо функцію $f$. Написали, тестуємо з різними аргументами. Виклик функції $f(a)$ працює, а ось $f(b)$ не працює. Після редагування коду~--- стало працювати $f(b)$. Але при цьому забули заново протестувати $f(a)$~--- можлива помилка в коді.

Автоматизоване тестування~--- це коли тести написані окремо від коду, і можна в будь-який момент запустити їх і перевірити всі важливі випадки використання.


Розглянутий приклад входить в методику тестування, яка входить в BDD~--- Behavior Driven Development. Підхід BDD давно і з успіхом використовується в багатьох проектах.

BDD~--- це не просто тести; тести BDD~--- це три в одному: тести, документація, приклади використання.

Як правило, потік розробки такий:

\begin{enumerate}
	\item Пишеться специфікація, яка описує самий базовий функціонал.
	\item Робиться початкова реалізація.
	\item Для перевірки відповідності специфікації задіюється фреймворк (в нашому випадку Mocha). Фреймворк запускає всі тести it і виводить помилки, якщо вони виникнуть. При помилках вносяться виправлення.
	\item Специфікація розширюється, в неї додаються можливості, які поки, можливо, не підтримуються реалізацією.
	\item Повертаємося до пункту б, робимо реалізацію. І так до завершення розробки.
\end{enumerate}

Розробка ведеться ітеративно: один прохід за іншим, поки специфікація і реалізація не будуть завершені.


\anonsection{ВИСНОВКИ}

Для виконання поставлених завдань було проведено аналіз характеристик існуючих систем планування, зокрема обсяг їх можливостей.

При підготовці до проектування було приділено увагу окремим частини процесу підготовки розкладу на прикладі факультету комп’ютерних наук, фізики та математики ХДУ.

На основі проведеного аналізу розроблено базові вимоги щодо можливостей додатку та його інтерфейсу.

Суттєву частку роботи приділено аналізу існуючих технологій всіх рівнів для створення веб-додатків. Детально досліджено роботу клієнт-серверних додатків та взаємодію через API (прикладний програмний інтерфейс). 

Відповідно до створених вимог розроблено проект додатку у відповідності до загальноприйнятих принципів побудови веб-додатків.

При реалізації веб-додатку використано бібліотеку React, завдяки якій було розроблено набір компонентів, що використовувались як в веб-додатку, так і в мобільному додатку. Останнє було досягнуто завдяки вибору платформи React Native~--- фреймворку для розробки мобільних додатків з відкритим кодом.

Розроблені додатки використовують публічний API для реалізації бізнес-логіки та взамодії з серверною частиною.


При розробці проекту використовується система контролю версій git з публічним репозиторієм на сервісі GitHub (github.com/ Rembut/gCalShedule), що дозволяє використовувати сучасні методи сумісної роботи та, одночасно з тим, дозволяє використовувати результати проведеного дослідження всім охочим під ліцензією MIT.
 % Заключение
\begingroup 
\renewcommand{\section}[2]{\anonsection{БІБЛІОГРАФІЯ}}
\begin{thebibliography}{00}

\printbibliography[heading=none]

\end{thebibliography}
\endgroup
 % Библиографический список

\end{document}
%%% Конец документа
