% За основу взят github.com/Amet13/bachelor-diploma

\documentclass[a4paper,14pt]{extarticle} % 14й шрифт
\input{PREAMBLE.tex} % Подключаем преамбулу

\hypersetup{
    colorlinks, urlcolor={black}, % Все ссылки черного цвета, кликабельные
    linkcolor={black}, citecolor={black}, filecolor={black},
    pdfauthor={Воробйов Євгеній Андрійович},
    pdftitle={Проектування та розробка UI веб-додатку редагування розкладу}
}

\addbibresource{bibliography.bib} % Библиографический справочник


%%% Начало документа
\begin{document}
\thispagestyle{empty}

{\centering
МІНІСТЕРСТВО ОСВІТИ І НАУКИ УКРАЇНИ

ХЕРСОНСЬКИЙ ДЕРЖАВНИЙ УНІВЕРСИТЕТ

ФАКУЛЬТЕТ КОМП'ЮТЕРНИХ НАУК, ФІЗИКИ ТА ІНФОРМАТИКИ

КАФЕДРА ІНФОРМАТИКИ, ПРОГРАМНОЇ ІНЖЕНЕРІЇ ТА 

ЕКОНОМІЧНОЇ КІБЕРНЕТИКИ

\vfill

ПРОЕКТУВАННЯ ТА РОЗРОБКА UI ВЕБ-ДОДАТКУ РЕДАГУВАННЯ РОЗКЛАДУ

Дипломна робота

на здобуття ступеня вищої освіти бакалавр

}

\vfill

\hfill\begin{minipage}[t]{0.6\textwidth}
Виконав: 

студент 4 курсу 431 групи \\ спеціальності 6.040302  Інформатика

Воробйов Євгеній Андрійович

Керівник:

кандидат педагогічних наук, доцент

Круглик Владислав Сергійович

\end{minipage}

\vfill

{\centering
Херсон --- 2019

}

\tableofcontents % Содержание 
\clearpage

\anonsection{ВСТУП}

Якість підготовки спеціалістів у закладах освіти і особливо ефективність використання науково-педагогічного потенціалу залежать певною мірою від рівня організації навчального процесу.

Одна з основних складових цього процесу — розклад занять — регламентує трудовий ритм, впливає на творчу віддачу викладачів, тому його можна вважати фактором оптимізації використання обмежених ресурсів — викладацького складу і аудиторного фонду.

Проблему складання розкладу слід розглядати не тільки як трудомісткий процес, об'єкт автоматизації з використанням комп’ютера, але і як проблему оптимального керування. 

Оскільки всі фактори, що впливають на розклад, практично неможливо врахувати, а інтереси учасників навчального процесу різноманітні, задача складання розкладу є багатокритеріальною з нечіткою множиною факторів.

Незалежно від алгоритму побудови розкладу, виникає прикладна проблема з інструментів різних рівнів, що використовуються в процесі. Саме ним і буде присвячено проведену роботу.

Актуальність дослідження полягає в необхідності забезпечення всіх учасників освітнього процесу доступом до актуальної версії розкладу занять у будь-який час, а також можливості спрощення процесу формування розкладу та подальшої інформатизації освітнього процесу.

Об’єкт дослідження — системи для планування розкладу. Предмет дослідження —  веб-додаток для планування розкладу в закладах освіти з поділом учнів (вихованців, здобувачів освіти тощо) на стабільні академічні групи.

Метою роботи є проектування розширюваного веб-додатку редагування розкладу та мобільного додатку для перегляду розкладу в закладах освіти з можливістю використання всіма учасниками освітнього процесу та розробка їх робочих прототипу.

Для реалізації мети поставлено наступні завдання:
\begin{enumerate}
	\item проаналізувати характеристики існуючих систем планування, зокрема обсяг їх можливостей;
	\item проаналізувати окремі частини процесу підготовки розкладу на прикладі факультету комп’ютерних наук, фізики та математики ХДУ;
	\item на основі проведеного аналізу розробити вимоги щодо можливостей додатків та їх інтерфейсів;
	\item відповідно до створених вимог розробити проект додатку;
	\item розробити робочий прототип і інтерфейс додатку;
	\item використовувати публічне API розробленого сервісу системи пыдтримки редагування розкладу для збережання та отримання даних;
	\item обґрунтувати використані технології при проектуванні клієнтської частини.
\end{enumerate}

Очікується, що спроектований продукт буде придатний до використання всіма учасниками освітнього процесу в ЗВО.
Робота складається з 2 розділів, містить \totalfigures\ рисунків.
 % Введение

\section{АНАЛІЗ СИСТЕМ ПЛАНУВАННЯ ТА ІНФОРМУВАННЯ}

\subsection{Задачі систем планування} Ідея планування робіт існує стільки, скільки існує людська цивілізація, адже ще в неоліті, з переходом до тваринництва і землеробства, постають задачі з контролем циклічних процесів, що і викликало у подальшому створення календаря і писемності для фіксування задач.

З розвитком та індустріалізацією суспільства класи задач, що вимагають бути покритими детальним плануванням, суттєво розширилися. Черговим етапом розвитку таких технологій стало виникнення електронно-обчислювальних машин і впровадження їх у використання в промисловості.

В подальшому використання таких систем виходить за межі корпоративних систем підприємств і все частіше ними починають користуватися люди для планування власного часу і вирішення особистих задач.


\subsection{Порівняння сервісів та додатків планування} На сьогодні, кожна людина так чи інакше стикається у повсякденному житті з системами, пов'язаними з контролем часу та завдань (рис.~\ref{fig:LightningOutlook}).

\addtwoimghere{MozillaLightning.png}{MSOutlook.png}{0.45}{Mozilla Lightning та MicroSoft Outlook}{fig:LightningOutlook}

\input{ComplexPlaningSystem.tex}
\subsubsection{Мобільні додатки}

TODO посмотреть на каждый и по абзацу два ключевых вещей что он делает === перевод описания на украинский


\paragraph{Calendar+ Schedule Planner App}

https://play.google.com/store/apps/details?id=com.joshy21.vera.free.calendarplus

\paragraph{School Timetable - Class, University Plan Schedule}

https://play.google.com/store/apps/details?id=com.pranapps.schooltimetable

\paragraph{My Class Schedule: Timetable}

https://play.google.com/store/apps/details?id=de.rakuun.MyClassSchedule.free

\paragraph{Class Timetable}

https://play.google.com/store/apps/details?id=com.icemediacreative.timetable

\paragraph{Class Schedule – super broker of work}

https://play.google.com/store/apps/details?id=com.lixiangdong.classschedule

\paragraph{Class Planner}

https://play.google.com/store/apps/details?id=com.apps.ips.classplanner2

\paragraph{College Schedule Builder}

https://play.google.com/store/apps/details?id=com.kwidil.collegeschedulebuilder


\subsection{Корпоративні інформаційні системи} \label{subsubs:KIS}

Корпоративна інформаційна система — це інформаційна система, яка підтримує автоматизацію функцій управління на підприємстві і постачає інформацію для прийняття управлінських рішень. У ній реалізована управлінська ідеологія, яка об'єднує бізнес-стратегію підприємства і прогресивні інформаційні технології.

У загальному визначенні «автоматизована система» — сукупність керованого об’єкта й автоматичних керувальних пристроїв, у якій частину функцій керування виконує людина. Вона представляє собою організаційно-технічну систему, що забезпечує вироблення рішень на основі автоматизації інформаційних процесів у різних сферах діяльності. 

Сучасні автоматизовані системи управління навчальним процесом у  закладах вищої освіти здатні вирішувати велику кількість функцій, а саме:
\begin{itemize}
	\item планування, контроль та аналіз навчальної діяльності;
	\item оперативний доступ до інформації про навчальний процес;
	\item єдину систему звітів, як внутрішніх, так і за вимогами МОН України;
	\item системи безпеки даних з урахуванням вимог законодавства;
	\item облік контингенту студентів та співробітників;
	\item проведення вступної кампанії;
	\item формування пакетів даних з метою виготовлення тих чи інших документів.
\end{itemize}

Функціонування будь-якої автоматизованої системи можна швидко адаптувати до особливостей навчального процесу конкретного навчального закладу, до локальних мереж різного рівня, що допомагає розширити коло користувачів (адміністрації, викладачів і студентів) для оперативного забезпечення їх необхідною інформацією. 

Отже, використання таких систем дає змогу не тільки удосконалити якість планування навчального процесу, а й оперативність управління ним.

Не зважаючи на всі переваги, які надає використання автоматизованих систем, досі далеко не в кожному закладі вони впроваджені чи використовуються в повній мірі з тих чи інших причин — інерційності поглядів адміністрації, супротив працівників або «саботаж» на місцях, відсутність фінансової або організаційної можливості.


\subsection{Обґрунтування використаних технологій} У розробці веб-додатків на сьогодняшній день використувується велика кількість мов програмування, таких як: C\#, Java, JavaScript, Python, PHP. Кожен розробник обирає для себе ту мову, яка йому здається найбільш відповідною для певної задачі. В загалом усі вище перераховані мови, окрім, може JavaScript, традиційно використовуються для розробки бекенд-частини веб-ресурсів, або для генерації фронтенду на серверній стороні та відправки сгенерованої сторінки клієнту. Для кожної конкретної мови є певні фреймворки, що спрощують створення веб-додатків.

Фреймворк - це інфраструктура програмних рішень, що полегшує розробку складних систем.

Взагалом фреймворки можуть включати в себе бібліотеки, що створені для вирішення певних задач та найкращі сталі практики з вирішення тих чи інших питань. Головне завдання фреймворків - зменшити об'єм однотипної праці, що розробник додатку виконує у кожному своєму проекті. Хоча кожна мова програмування й має свої фреймворки, їх сутність у цілому залишається для певних задач доволі близькою: використання певних шаблонів проектування, що допомагають зробити програмний код більш зрозумілим та простим для подальшої підтримки та ускладнення.

Найбільш поширеними фреймворками є: для C\# є .NET, для Java - Spring, Hibernate, для JavaScript - Node, React, Vue, Angular, для Python - Django, для PHP - Laravel. Кожен фреймворк та мова програмування мають свої переваги та недоліки. Тому, коли перед нами постало завдання вибрати, на якому саме стеку технологій ми будемо розробляти веб-додаток, нами було проаналізовано кожну із вищепредставлених мов.

Наш вібір зупинився на JavaScript як для серверної, так і для клієнтської частини. Адже, якщо серверна та клієнтська частина написани з використанням однієї мови програмування, це сильно спрощує розробку та підримку веб-додатку~\cite{vedantani2017}. Для серверної частини нами було вирішено використовувати фреймворк Node.js, тому що він добре поєднується із будь-яким фронтенд фреймворком, має велике коло шанувальників, постійно оновлюється, що підвищує його безпечність, та хорошу документацію.  

\subsubsection{Семантичне версіювання}

В процесі розробки програмного забезпечення можливе виникнення проблеми під назвою <<пекло залежностей>>. 

Суть полягає в тому, що при збільшенні розмірів програмної системи, збільшується кількість бібліотек та пакетів, що використовуються в ній. При цьому, кожен з них, зазвичай, вимагає для своєї роботи деякі інші бібліотеки певних версій. У разі, якщо документація програмного забезпечення надто вільна, то рано чи піздно виникає проблема невідповідності між фактично необхідною версією, вказаною в документації та встановленою, що негативно позначається на всьому процесі розробки програмного забезпечення.

Для вирішення цієї проблеми пропонується простий набір правил і вимог, що визначають як встановлюються і збільнуються номери версій. Для роботи системи необхідно створити і описати публічне API програмного продукту. Після цього будь-які зміни в версії визначаються певною зміної її номера.

Розглянемо формат версій X.Y.Z (мажорна, мінорна, патч).

Зміни, що не впливають на API, збільнують патч-версію. Зворотньо-сумістні зміни та розширення API збільшують мінорну версію. І, нарешті, несумістні зміни API збільшують мажорну версію.

Ця система називатиметься <<Семантичне версіювання>>.

Мажорна версія <<0>> (0.Y.Z) призначена для початкової розробки, публічний API не має розглядатися як стабільний. Версія 1.0.0 визначає публічний API, після цього релізу вона змінюватиметься відповідно до змін в API. Після чергової зміни мінорної версії патч-версія змінюється на <<0>>, аналогічні зміни відбуваються зі зміною мажорної версії.

Крім зазначених правил, специфікація семантичного версіонування~\cite{semver} визначає додатково певні деталі та поради щодо його практичного використання, зокрема для продуктів, що мають складну систему релізів та передрелізних версій.

\subsubsection{Latex} \label{subsub:latex}

\TeX -- це створена чудовим американським математиком і програмістом Дональдом Кнутом система для верстки текстів з формулами. Сам по собі TEX є спеціалізованою мовою програмування (Кнут не тільки придумав мову, а й написав для нього транслятор, причому таким чином, що він працює абсолютно однаково на самих різних комп'ютерах), на якому пишуться видавничі системи, що використовуються на практиці. Точніше кажучи, кожна видавнича система на базі TEXа є пакетом макросів (макропакет) цієї мови. LATEX -- це створена Леслі Лампортом видавнича система на базі TEXа~\cite{львовский2003latex}.

Всі видавничі системи на базі TEXа володіють перевагами, закладеними в самому TEXе. Для новачка їх можна описати однією фразою: надрукований текст виглядає «зовсім як у книзі». LATEX, як видавнича система, надає зручні і гнучкі засоби досягти цього книжкового якості. Зокрема, вказавши за допомогою простих засобів структуру тексту, автор може не вникати в деталі оформлення, причому ці деталі при необхідності неважко змінити (щоб, скажімо, змінити шрифт, яким друкуються заголовки, не треба нишпорити по всьому тексту, змінюючи все заголовки , а досить замінити одну сходинку в «стильовому файлі»). Такі речі, як нумерація розділів, посилання, зміст і т. П. Виходять майже що «самі собою». Величезним плюсом систем на базі TEXа є висока якість та гнучкість форматування абзаців і математичних формул (в останньому відношенні краще TEXа цю задачу не вирішує жодна програма).

TEX (і всі видавничі системи на його базі) невибагливий до використовуваної техніки. З іншого сторони, TEXовські файли мають високий ступінь переносимості: Ви можете підготувати LATEXовський вихідний текст на своєму IBM PC, переслати його до видавництва, і бути впевненими, що там Ваш текст буде правильно оброблений і на друку вийде в точності те ж, що вийшло у Вас при пробному друку на Вашому улюбленому матричному принтері (з тією єдиною різницею, що фотоскладальний автомат дасть текст більш високої якості). Завдяки цій обставині TEX став дуже популярний як мова міжнародного обміну статтями з математики та фізики.

LaTeX - це високоякісна набірна система; він включає функції, призначені для виготовлення технічної та наукової документації. LaTeX є фактичним стандартом для комунікації та публікації наукових документів \cite{lamport1994latex}. LaTeX доступний як вільне програмне забезпечення.

При роботі над звітом також використано сервіс Overleaf -- сучасний інструмент, розроблений у 2012 році. Він був створений щоб допомогти редагувати свої наукові статті, технічні звіти, тези, презентації, блок-схеми та інші документи, написані на мові розмітки LaTeX. 

При цьому, було використано всі переваги хмарних технологій, в тому числі можливість миттєвого початку роботи на практично будь-якому комп'ютері, збереження версій та одночасної роботи над проектом кількох користувачів.

Також нівелюється необхідність у встановленні на комп'ютері додаткового програмного забезпечення, що може бути названим одним із недоліків використання окремої системи, як LaTeX.

\subsubsection{Система контролю версій Git}

Активну популярність мають розподілені системи контролю версій (SCM).

Найбільш поширеними з таких є Subversion (SVN), Microsoft Visual Source Safe (VSS), Revision Control System (RCS), Concurrent Versions System (CVS), Gіt та Mercurіal. Знання подібних систем підвищує затребуваність ІT фахівців на ринку праці, покращує продуктивність розробників та полегшує рішення щоденних завдань. Саме передача знань є вирішальною у процесі експорту-імпорту технологій~\cite{киричек2012модель}.

В процесі роботи використано систему контролю версій Git з віддаленим репозиторієм на сервісі GitHub~\cite{gCalShedule}.

Система контролю дозволяє зберігати попередні версії файлів та завантажувати їх за потребою. Вона зберігає повну інформацію про версію кожного з файлів, а також повну структуру проекту на всіх стадіях розробки. Місце зберігання даних файлів називають репозиторієм. В середині кожного з репозиторіїв можуть бути створені паралельні лінії розробки — гілки.

Git підтримує швидке розділення та злиття версій, містить можливості для візуалізації і навігації за нелінійною історією розробки. 

\subsubsection{Sublime Text}

Sublime Text - прорієтарний текстовий редактор. Підтримує плагіни на мові програмування Python.

Розробник дає можливість безкоштовно і без обмежень ознайомитися з редактором, однак програма періодично буде повідомляти про необхідність придбання ліцензії.

Редактор містить різні візуальні теми, а також можливість завантаження додаткових тем.

Коли користувач набере код, Sublime Text, в залежності від використовуваного мови, буде пропонувати різні варіанти для завершення запису. Також редактор може автоматично додавати розділові знаки (<<\{>>, <<\}>>, <<;>>).

Sublime Text дозволяє збирати програми у готовий проект і запускати їх без необхідності використання зовнішньої командної строки. Користувач також може налаштувати свою систему компіляції і включити автоматичну збірку програм кожного разу при збереженні коду. Ця система схожа з відповідним плагіном для зйомки тексту від віддаленого LaTeX (розділ~\ref{subsub:latex}).

Використовується плагін LaTeXTools. Плагін LaTeXTools надає кілька функцій, які спрощують роботу з файлами LaTeX.

Команда ST збирає компіляцію джерела LaTeX у PDF за допомогою texify (Windows / MikTeX) або latexmk (OSX / MacTeX, Windows / TeXlive, Linux / TeXlive). Потім він розбирає файл журналу і перераховує помилки та попередження. Нарешті, він запускає програму перегляду PDF і, на підтримуваних переглядачах (Sumatra PDF на Windows, Skim на OSX і Evince на Linux за замовчуванням) переходить до поточної позиції курсора.

Додатково реалізована функція автозбереження, що допомагає користувачам не втратити пророблену роботу.


\subsubsection{Бібліотеки JS}

\paragraph{Express}

При розробці використано бібліотеку Express — гнучкий фреймворк для веб-застосунків, побудованих на Node.js, що надає широкий набір функціональності, полегшуючи створення надійних API.

Express забезпечує тонкий прошарок базової функціональності для веб-застосунків, що не спотворює звичну та зручну функціональність Node.js., при отриманні запиту він оброблюватиметься відповідно до визначення маршруту (рис.~\ref{fig:Route}), де app є екземпляром express, METHOD є методом HTTP-запиту, PATH є шляхом на сервері, HANDLER є функцією-обробником, що спрацьовує, коли даний маршрут затверджено як співпадаючий \cite{hahn2016express}.

Маршрутизація визначає, як додаток відповідає на клієнтський запит до конкретної адреси (кінцевій точці), яким є URI (або шлях), і певного методу запиту HTTP (GET, POST і т.д.).

Кожен маршрут може мати одну або кілька функцій обробки, які виконуються при зіставленні маршруту.

Express підтримує перераховані далі методи маршрутизації, які відповідають методам HTTP: get, post, put, head, delete, options, trace, copy, lock, mkcol, move, purge, propfind, proppatch, unlock, report, mkactivity, checkout, merge, m -search, notify, subscribe, unsubscribe, patch, search і connect.

Шляхи маршрутів, в поєднанні з методом запиту, визначають конкретні адреси (кінцеві точки), в яких можуть бути створені запити. Шляхи маршрутів можуть являти собою рядки, шаблони рядків або регулярні вирази.

Для обробки запиту можна вказати кілька функцій зворотного виклику, подібних middleware (детальніше про них в пункті~\ref{subs:middleware}). Єдиним винятком є те, що ці зворотні виклики можуть ініціювати next ('route') для обходу інших зворотних викликів маршруту. За допомогою цього механізму можна включити в маршрут попередні умови, а потім передати управління подальшим маршрутами, якщо продовжувати роботу з поточним маршрутом не потрібно.

В Express немає засобів для роботи з базою даних \cite{simon2015nodeexpress}. Їх надають модулі та бібліотеки Node.js, що дозволяють взаємодіяти з будь-якою базою даних. В роботі використовується ORM Sequalize (детальніше в пункті \ref{subs:orm})


\paragraph{JSON Web Token}

Для забезпечення конфіденційності при обміні даними використовується JSON Web Token. Деталі роботи з ним розглянуто в розділі \ref{subsubsection:jwt}. Для роботи з JSON Web Token використовується бібліотека jsonwebtoken.

JWT визначає особливу структуру інформації, яка відправляється по мережі. Вона представлена в двох формах - серіалізовані і десеріалізованной. Перша використовується безпосередньо для передачі даних із запитами і відповідями. З іншого боку, щоб читати і записувати інформацію в токен, потрібна його десеріалізація.



\paragraph{bcrypt.js}

Оптимізовано bcrypt в JavaScript з нульовими залежностями. Сумісний з C ++ bcrypt прив'язка на node.js і також працює в браузері.

Міркування безпеки
Крім включення солі для захисту від атак аеродинамічних таблиць, bcrypt є адаптивною функцією: з плином часу кількість ітерацій може бути збільшена, щоб зробити її більш повільною, тому вона залишається стійкою до атаки з використанням грубої сили навіть при збільшенні потужності обчислення. (подивитися)

Хоча bcrypt.js є сумісним з C ++ bcrypt прив'язка, він написаний на чистому JavaScript і, таким чином, повільніше (близько 30\%), ефективно зменшуючи кількість ітерацій, які можуть бути оброблені в рівний проміжок часу.

Максимальна вхідна довжина становить 72 байти (зауважимо, що символи UTF8 кодуються до 4 байт), а довжина згенерованих хешей - 60 символів.


\clearpage
\section{ПРОЕКТУВАННЯ ТА РОЗРОБКА КЛІЄНТСЬКОЇ ЧАСТИНИ}

\subsection{Розробка SPA використовуючи бібліотеку React}
\subsubsection{Аналіз існуючих бібліотек для розробки SPA}

На сьогодні, одними з росповсюджених фреймворків для розробки SPA (single page applications) є React, Angular та Vue (рис.~\ref{fig:ReactAngularVue}).

\addimg{ReactAngularVue.png}{0.35}{React, Angular та Vue}{fig:ReactAngularVue}

Angular -- Javascript-фреймворк, створений на основі TypeScript. Розроблений і підтримуваний компанією Google, він описується як JavaScript MVW-фреймворк. На даний момент останньою версією є 4. Фреймворк Angular використовується такими компаніями, як Google, Wix, weather.com, healthcare.gov і Forbes.

\label{subs:vue}
Vue -- ще один JS-фреймворк. Творці Vue описують його як «інтуїтивно зрозумілий та швидкий, призначений для створення інтерактивних інтерфейсів». Вперше він був представлений колишнім співробітником компанії Google Еваном Ю (Evan You) в лютому 2014 року. На даний момент фреймворк використовується такими компаніями, як Alibaba, Baidu, Expedia, Nintendo, GitLab.

\subsubsection{Вимоги до додатку}

Одним із завдань, поставлених для реалізації мети є розроблення вимог щодо можливостей веб-додатку та його інтерфейсу.

Після проведення аналізу предметної області, додатків аналогів та технологій було сформульовано наступні вимоги:

\begin{enumerate}
    \item Веб-додаток повинен коректно відражатися у останніх версіях популярних веб-браузерів(Google Chrome, Mozilla Firefox, Opera, Safari).
    \item Адаптивність інтерфейсу до розмірів вікна браузера або пристрою.
    \item Можливість зміни типу відображення поточного розкладу.
    \item Зміна інтерфейсу відповідно до прав доступу поточного користувача.
    \item Створення нових користувачів у системі адміністратором.
    \item Авторизація користувачів у системі.
    \item Редагування даних користувачами системи у відповідності до їх прав доступу.
    \item Редагування даних адміністраторами системи у відповідності до їх прав доступу.
    \item Інтеграція з Google сервісами, зокрема Google calendar та Google Sheets.
    \item Інтерфейс для імпорту та експорту даних між системою та сервісами Google або специфікованими форматами даних.
    \item Забезпечення цілісності даних.
\end{enumerate}



\subsubsection{React} \label{subs:React}


При розробці мобільного додатку використано фреймворк React Native, в основі якого знаходиться бібліотека React, призначена для створення користувацьких інтерфейсів. На відміну від React, призначеного для розробки односторінкових веб-додатків, React Native спрямований на мобільні платформи~\cite{davidgeary2019}.

React~--- відкрита JavaScript бібліотека для створення інтерфейсів користувача, яка покликана вирішувати проблеми часткового оновлення вмісту веб-сторінки, з якими стикаються в розробці односторінкових застосунків~\cite{ericmasiello2017}. React використовують Facrbook, Airbnb, Uber, Netflix, Twitter, Pinterest, Reddit, Udemy, Wix, Paypal, Imgur, Feedly, Stripe, Tumblr, Walmart та інші.

React дозволяє розробникам створювати великі веб-додатки, які використовують дані, котрі змінюються з часом, без перезавантаження сторінки. React обробляє тільки користувацький інтерфейс у застосунках. Це відповідає видові у шаблоні модель-вид-контролер (MVC) і може бути використане у поєднанні з іншими JavaScript бібліотеками або в великих фреймворках MVC, таких як AngularJS. Він також може бути використаний з React на основі надбудов, щоб піклуватися про частини без користувацького інтерфейсу побудови веб-застосунків.

React підтримує віртуальний DOM, а не покладається виключно на DOM браузера. Це дозволяє бібліотеці визначити, які частини DOM змінилися, порівняно зі збереженою версією віртуального DOM, і таким чином визначити, як найефективніше оновити DOM браузера. Як бібліотека інтерфейсу користувача React часто використовується разом з іншими бібліотеками, такими як Redux, проте у його використанні при розробці проекту не було необхідності.

React надає розробникам безліч методів, які викликаються під час життєвого циклу компонента (рис.~\ref{fig:ReactReduxCommunicate}), вони дозволяють нам оновлювати UI і стан додатку. Коли необхідно використовувати кожен з них, що необхідно робити і в яких методах, а від чого краще відмовитися, є ключовим моментом до розуміння як працювати з React.

\addimg{ReactReduxCommunicate.png}{0.85}{Взаємодія Redux та React}{fig:ReactReduxCommunicate}

Конструктори є основною ООП — це спеціальна функція, яка буде викликатися щоразу, коли створюється новий об'єкт. Важливо викликати функцію super в випадках, коли наш клас розширює поведінку іншого класу, який має конструктор. Виконання цієї спеціальної функції буде викликати конструктор нашого батьківського класу і дозволяти йому проініціаліззувати себе. 
Конструктори~--- це відмінне місце для ініціалізації компонента~--- створення будь-яких полів (змінні, що починаються з this.).

Це також єдине місце де слід встановлювати стан безпосередньо перезаписуючи поле this.state. У всіх інших випадках необхідно використовувати $this.setState$.

За замовчуванням, всі компоненти будуть перемальовувати себе всякий раз, коли їх стан змінюється, змінюється контекст або вони приймають props від батьківського компонента. Якщо перерисовка компонента досить важка (наприклад генерація графіка), то у розробників є доступ до спеціальної функції, яка дозволяє контролювати цей процес.

\subsubsection{Webpack}

При створенні сайту досить стандартною практикою є мати певний процес збірки на місці, щоб полегшити розробку і підготовку файлів до роботи.

Можливо використовувати Grunt або Gulp, побудувавши ланцюжки перетворень, які дадуть можливість подати код в один кінець ланцюжка і отримати мінімізовані CSS та JavaScript на іншому.

Подібні інструменти розробки досить популярні і корисні в наші дні. Проте, є й інший метод полегшення розробки — Webpack.

Webpack є так званим «збиральником модулів». Він приймає модулі JavaScript, аналізує їх залежності один від одного, а потім з'єднує їх найефективнішим способом, випускаючи у кінці лише один JavaScript файл~\cite{juhovepsalainen2016}.

З Webpack, модулі не обмежені тільки файлами JavaScript. Завдяки частині loaders, Webpack розуміє, що модуль JavaScript може потребувати CSS файл, а цей CSS файл може потребувати зображення. Результат роботи Webpack буде містити тільки те, що потрібно у проекті.

\subsubsection{Babel}

Нажаль, при постійному розвитку мов програмування невід'ємною є ситуація, що реалізація часто відстає від специфікації. Більш того, різні реалізації по-різному відстають від специфікації. Написавши код, ми не можемо гарантувати, де він буде запускатися, а де - ні.

Виходячи з цього можна зробити висновок, що потрібно писати код, дотримуючись старих стандартів. На щастя, є інший шлях: ми можемо писати код з використанням всіх найновіших можливостей, але перед публікацією автоматично транслювати його (тобто переводити з одного виду в інший) в стару версію. 

Сама природа JS і його способи використання готують нас до того, що ніколи не настане моменту, коли у всіх користувачів буде остання версія інтерпретатору. Люди використовували і продовжать використовувати різні браузери і різні версії браузерів, різні версії Node.js і так далі. Використання нових синтаксичних конструкцій в такій ситуації практично неможливо. Запуск коду на платформі що не підтримує новий синтаксис призведе до синтаксичну помилку. 

Закономірним вирішенням цієї проблеми стала поява Babel - програми, яка бере вказаний код і повертає той же код, але транслювали в стару версію JS. Фактично, в сучасному світі Babel став невід'ємною частиною JS. Всі нові проекти так чи інакше розробляють з його використанням~\cite{davidgeary2019}.

\subsubsection{Розробка компонентів}

В термінах React, всі частини-відображення іменуються компонентами. В роботі спроектована серія компонентів для різних частин системи та розроблено прототипи деяких з них. 

На рис.~\ref{fig:AdminPanelUserCreation} зображено компонент для створення адміністратором нового користувача системи. Слід наголосити, що кожен компонент є окремою частиною і тому можливий для використання у подальшому в інших системах при виконанні певних вимог (так зване «повторне використання»).

\addimg{AdminPanelUserCreation.png}{0.85}{Адміністративна панель (створення користувача)}{fig:AdminPanelUserCreation}

На приведеному вище зображенні натискання на кожну з кнопок призводить до виклику відповідного методу API шляхом надсилання певного HTTP запиту.

Перед доступом до адміністративної панелі (рис.~\ref{fig:AdminPanelUserManagement}) адміністратору необхідно авторизуватися у системі, в результаті чого буде створено і збережено силами його веб-браузеру JWT. Після цього, якщо він має відповідні права  та верифікація токену пройшла успішно, йому буде відображена відповідна панель.

\addimg{AdminPanelUserManagement.png}{0.85}{Адміністративна панель (керування користувачами)}{fig:AdminPanelUserManagement}

Слід зауважити, що подібний процес перевірки відбувається при виконанні кожного запиту, крім тих, що не потребують авторизації (доступні для незареєстрованих користувачів).

Перша прерогатива адміністратора системи~--- створення нових користувачів, при цьому відбувається вищеописана процедура.

Певна частина об’єктів системи може вважатися більш-менш константною, це структура закладу вищої освіти (рис.~\ref{fig:AdminPanelFacultyManagement}), окремі словникові дані (зокрема, назви посад професорсько-викладацького складу та види занять).

\addimg{AdminPanelFacultyManagement.png}{0.85}{Адміністративна панель (керування факультетами)}{fig:AdminPanelFacultyManagement}

Окремо можна відзначити використання об’єктів часу (номера занять впродовж дня, рис.~\ref{fig:AdminPanelTimeManagement}).  В спроектованій системі одним з можливих шляхів доступу до розкладу є його експорт до сервісу Google Calendar в серію календарів. 

\addimg{AdminPanelTimeManagement.png}{0.85}{Адміністративна панель (керування часом)}{fig:AdminPanelTimeManagement}

В подальшому, потенційні користувачі можуть підписуватися на оновлення відповідних календарів (зокрема, груп та викладачів) для отримання актуальної інформації в довільний момент часу, котру підтримуватимуть користувачі системи з привілегією редагування даних про розклад (рис.~\ref{fig:AdminPanelScheduleEdit}).

\addimg{ScheduleEdit.png}{0.95}{Адміністративна панель (редагування розкладу)}{fig:AdminPanelScheduleEdit}

В процесі розробки було підготовлено серію компонентів, кожен з яких відповідає за певну конкретну задачу та може використовуватися в багатьох місцях додатку (так зване <<повторне використання>>). На рис.~\ref{fig:ReactComponent} представлено один із компонентів, що відображає списки у системі.

Представлений компонент відноситься до компонентів-відображень. При цьму, компоненти діляться на декілька груп. 

Компоненти-контейнери відповідають за дані і операції з ними. Їх стан передається у вигляді властивостей в компоненти-відображення.

Компоненти-відображення є <<дурними>> в тому сенсі, що вони не представляють, звідки беруться дані, тобто вони нічого не знають про стан.

Компоненти-відображення не повинні змінювати дані. Фактично, будь-який компонент, який одержує властивості від батьківського компоненту, повинен залишати їх незмінними. У той же час, вони можуть будь-яким чином форматувати дані (наприклад, конвертуючи Unix timestamp в дату-час для відображення).

Компоненти-контейнери найчастіше є батьківськими для компонентів-відображень і забезпечують зв'язок між відображенням і іншими частинами програми. Їх також називають «розумними» компонентами, оскільки вони <<знають>> про програму в цілому.

\addCodeAsImg{\lstinputlisting[numbers=left]{code/ReactComponent.tex}}{React компонент відображення списку}{fig:ReactComponent}


\subsection{Розробка мобільного додатку}
\subsubsection{Вимоги до додатку та системи виконання}

Одним із завдань, поставлених для реалізації мети є розроблення вимог щодо можливостей мобільного додатку та його інтерфейсу.

Після проведення аналізу предметної області, додатків аналогів та технологій було сформульовано наступні вимоги:

\begin{enumerate}
    \item Мобільний додаток повинен виконуватися на платформі $ios$.
    \item Можливість вибору та збереження групи користувача.
    \item Перегляд поточного розкладу.
    \item Перегляд розкладу в конкретний день тижня в минулому або майбутньому. 
    \item Функція сповіщення про зміни у розклади користувача.
    \item Можливість пошуку та перегляду розкладу занять викладачів.
    \item Можливість редагування збереженої інформації про користувача додатку.
\end{enumerate}

\subsubsection{React Native}

\addimg{ReactNative.png}{0.7}{Структура React Naive}{fig:ReactNative}

React Native~--- це JS-фреймворк для створення нативних iOS і Android додатків. В його основі лежить розроблена в Facebook JS-бібліотека React (проаналізовано та описано в пункті \ref{subs:React}), призначена для створення користувацьких інтерфейсів. Але замість браузерів вона орієнтована на мобільні платформи. Іншими словами, якщо ви веб-розробник, то можете використовувати React Native для написання  швидких мобільних додатків, з комфортом використання у розробці звичного фреймворка і єдиної кодової бази JavaScript~\cite{jeffgothelf2016}.

Популярність React має ряд причин. Вона компактна і має високу продуктивність, особливо при роботі з швидкоплинними даними. Завдяки компонентної структурі, React заохочує до написання модульного коду и пропагує повторне його використання.

Спроектовано та розроблено серію компонентів для різних частин додатку. На рис.~\ref{fig:interphace} зображено сукупність компонентів, яка формує стартовове вікна додатку, у якому користувач може брати номер своєї групи для отримання розкладу (зліва) та сукупність компонентів, яка формує вікно відображення поточного розкладу з можливістю обрання конкретної дати для відображення (справа). Перший компонент зберігає номер групи локально на мобільному пристрої завдяки використанню функції $AsynkStorage()$, котра надається платформою React Native (код компоненту на рис.~\ref{fig:ReactNativeCode}).

\addtwoimghere{Mobile1.png}{Mobile1.png}{0.45}{Інтерфейс мобільного додатку}{fig:interphace}

React Native~--- це той же React, але для мобільних платформ. У нього є ряд відмінностей: замість тега $div$ використовується компонент $View$, а замість тега $img$~--- $Image$. Вам може стати в нагоді знання Objective-C, Swift або Java~\cite{9781787282537}. 

Нативну (native) розробку можна назвати <<рідною>> для операційних систем~--- Android, IOS, Win Phone і т.д. Такі мобільні додатки пишуться на мовах програмування, затверджених розробниками програмного забезпечення під кожну конкретну платформу, а тому органічно вбудовуються в самі операційні системи.

У React компонент описує власне відображення, а потім бібліотека обробляє для вас рендеринг (деталі у пункті \ref{fig:ReactNative}). Ці дві функції розділені ясним рівнем абстракції. Якщо потрібно відобразити компоненти для вебу, то React використовує стандартні HTML-теги. Завдяки тому ж рівню абстракції~--- <<мосту>>~--- для рендеринга в iOS і Android React Native викликаються відповідні API.

\addCodeAsImg{\lstinputlisting[numbers=left]{code/ReactNativeCode.tex}}{React Native компонент стартового вікна}{fig:ReactNativeCode}

Замість компіляції в нативний код, React Native запускає його за допомогою JS-движка хост-платформи, без блокування основного UI-потоку. Ви отримуєте переваги нативних продуктивності, анімації і поведінки без необхідності писати на Objective-C або Java. Інші методи розробки кроссплатформенних додатків, на кшталт Cordova або Titanium, ніколи не досягнуть такого рівня нативної продуктивності або відображення.

Головна перевага нативних додатків~--- то, що вони оптимізовані під конкретні операційні системи, а значить працюють коректно і швидко. Також вони мають доступ до апаратної частини пристроїв, тобто можуть використовувати в своєму функціоналі камеру смартфона, мікрофон, акселерометр, геолокацію, адресну книгу, плеєр і т.д. Можна налаштувати отримання push-повідомлень. Ще один плюс - економну витрату ресурсів телефону (батарея, пам'ять).

Головна особливість React Native~--- він «дійсно» нативний. Інші рішення JavaScript для мобільних платформ просто обертають ваш JS-код в веб-відображення. Вони можуть перереалізовать яке-небудь нативное поведінку інтерфейсу, наприклад, анімацію, але все ж  залишаються веб-додатоком.

\paragraph{Переваги для розробника}

У порівнянні зі стандартною розробкою під iOS і Android, React Native має набагато більше переваг. Оскільки додаток здебільшого складається з JavaScript, можливо користуватися численними перевагами веб-розробки. Наприклад, щоб побачити внесені в код зміни, можна миттєво «оновити» додаток замість тривалого очікування завершення традиційної компіляції~\cite{robinwieruch2018}. 

Крім того, React Native надає «розумну» систему повідомлень про помилки і стандартні інструменти налагодження JavaScript, що сильно полегшує процес мобільного розробки.

\paragraph{Обробка декількох платформ}

React Native витончено обробляє різні платформи. Переважна більшість API у фреймворку~--- кросплатформені, так що досить просто написати компонент React Native, і він буде без проблем працювати на iOS і Android платформах.

Якщо вам потрібно писати залежний від платформи код - в зв'язку з різними правилами взаємодії в iOS і Android, або через переваг платформозалежного API~--- то з цим не буде труднощів. React Native дозволяє призначати платформозалежні версії кожного компонента, які ви можете потім інтегрувати в свій додаток.


\subsection{Менеджери стану}
\subsection{Менеджери стану}

Redux — це інструмент управління як станом даних, так і станом інтерфейсу в JavaScript-додатках. Він підходить для односторінкових додатків, в яких управління станом може з часом стає складним. Redux не пов'язаний з якимось певним фреймворком, і хоча розроблявся для React, може використовуватися з Angular або jQuery.

C Redux всі компоненти отримують свої дані зі сховища. Також зрозуміло, куди компонент повинен відправити інформацію про зміну стану — знову ж в сховище. Компонент тільки ініціює зміну і не піклується про інших компонентах, які повинні отримати цю зміну. Таким чином, Redux робить потік даних більш зрозумілим.

Загальна концепція використання сховищ для координації стану програми — це шаблон, відомий як Flux. Цей шаблон проектування доповнює односпрямований потік даних як в React.

Redux використовує тільки одне сховище для всього стану програми. Оскільки стан знаходиться в одному місці, його називає єдиним джерелом істини. Структура даних сховища повністю залежить від вас, але для реального застосування це, як правило, об'єкт з декількома рівнями укладення.

Такий підхід єдиного сховища є основною відмінністю між Redux і Flux з його численними сховищами.

Згідно з документацією Redux, «Єдиний спосіб змінити стан - передати action — об'єкт, що описує, що сталося». Це означає, що програма не може безпосередньо змінити стан. Замість цього, необхідно передати «action», щоб висловити намір змінити стан в сховищі (рис.~\ref{fig:ReactReduxCommunicate}).


\subsection{Контроль якості програмного забезпечення}
\subsubsection{Задачі контролю якості програмного забезпечення}



\include{conclusion} % Заключение
\include{bibliography} % Библиографический список

\end{document}
%%% Конец документа
