% За основу взят github.com/Amet13/bachelor-diploma

\documentclass[a4paper,14pt]{extarticle} % 14й шрифт
\input{preamble} % Подключаем преамбулу

\hypersetup{
    colorlinks, urlcolor={black}, % Все ссылки черного цвета, кликабельные
    linkcolor={black}, citecolor={black}, filecolor={black},
    pdfauthor={Воробйов Євгеній Андрійович},
    pdftitle={Проектування та розробка UI веб-додатку редагування розкладу}
}

\addbibresource{bibliography.bib} % Библиографический справочник


%%% Начало документа
\begin{document}
\thispagestyle{empty}

{\centering{\bfseries
МІНІСТЕРСТВО ОСВІТИ І НАУКИ УКРАЇНИ

ХЕРСОНСЬКИЙ ДЕРЖАВНИЙ УНІВЕРСИТЕТ

Факультет комп'ютерних наук, фізики та інформатики

Кафедра інформатики, програмної інженерії та економічної кібернетики

\vfill

ПРОЕКТУВАННЯ ТА РОЗРОБКА UI ВЕБ-ДОДАТКУ РЕДАГУВАННЯ РОЗКЛАДУ

\vfill

Дипломна робота

}

на здобуття ступеня вищої освіти бакалавр

}

\vfill

\hfill\begin{minipage}[t]{0.65\textwidth}
Виконав студент 4 курсу 431 групи 

напряму підготовки 6.040302 Інформатика

Воробйов Євгеній Андрійович

Керівники: доктор педагогічних наук, доцент

Круглик Владислав Сергійович,

кандидат фізико-математичних наук, доцент

Єрмолаєв Вадим Анатолійович 

Рецензент: 

кандидат фізико-математичних наук, професор

Кузьмич Валерій Іванович


\end{minipage}

\vfill

{\centering
Херсон --- 2019

}

\tableofcontents % Содержание 
\clearpage

\anonsection{ВСТУП}

Якість підготовки спеціалістів у закладах освіти і особливо ефективність використання науково-педагогічного потенціалу залежать певною мірою від рівня організації навчального процесу.

Одна з основних складових цього процесу — розклад занять — регламентує трудовий ритм, впливає на творчу віддачу викладачів, тому його можна вважати фактором оптимізації використання обмежених ресурсів — викладацького складу і аудиторного фонду.

Проблему складання розкладу слід розглядати не тільки як трудомісткий процес, об'єкт автоматизації з використанням комп’ютера, але і як проблему оптимального керування. 

Оскільки всі фактори, що впливають на розклад, практично неможливо врахувати, а інтереси учасників навчального процесу різноманітні, задача складання розкладу є багатокритеріальною з нечіткою множиною факторів.

Незалежно від алгоритму побудови розкладу, виникає прикладна проблема з інструментів різних рівнів, що використовуються в процесі. Саме ним і буде присвячено проведену роботу.

Актуальність дослідження полягає в необхідності забезпечення всіх учасників освітнього процесу доступом до актуальної версії розкладу занять у будь-який час, а також можливості спрощення процесу формування розкладу та подальшої інформатизації освітнього процесу.

Об’єкт дослідження — системи для планування розкладу. Предмет дослідження —  веб-додаток для планування розкладу в закладах освіти з поділом учнів (вихованців, здобувачів освіти тощо) на стабільні академічні групи.

Метою роботи є проектування розширюваного веб-додатку редагування розкладу та мобільного додатку для перегляду розкладу в закладах освіти з можливістю використання всіма учасниками освітнього процесу та розробка їх робочих прототипу.

Для реалізації мети поставлено наступні завдання:
\begin{enumerate}
	\item проаналізувати характеристики існуючих систем планування, зокрема обсяг їх можливостей;
	\item проаналізувати окремі частини процесу підготовки розкладу на прикладі факультету комп’ютерних наук, фізики та математики ХДУ;
	\item на основі проведеного аналізу розробити вимоги щодо можливостей додатків та їх інтерфейсів;
	\item відповідно до створених вимог розробити проект додатку;
	\item розробити робочий прототип і інтерфейс додатку;
	\item використовувати публічне API розробленого сервісу системи пыдтримки редагування розкладу для збережання та отримання даних;
	\item обґрунтувати використані технології при проектуванні клієнтської частини.
\end{enumerate}

Очікується, що спроектований продукт буде придатний до використання всіма учасниками освітнього процесу в ЗВО.
Робота складається з 2 розділів, містить \totalfigures\ рисунків.
 % Введение

\section{АНАЛІЗ СИСТЕМ ПЛАНУВАННЯ ТА ІНФОРМУВАННЯ}
\subsection{Задачі систем планування} Ідея планування робіт існує стільки, скільки існує людська цивілізація, адже ще в неоліті, з переходом до тваринництва і землеробства, постають задачі з контролем циклічних процесів, що і викликало у подальшому створення календаря і писемності для фіксування задач.

З розвитком та індустріалізацією суспільства класи задач, що вимагають бути покритими детальним плануванням, суттєво розширилися. Черговим етапом розвитку таких технологій стало виникнення електронно-обчислювальних машин і впровадження їх у використання в промисловості.

В подальшому використання таких систем виходить за межі корпоративних систем підприємств і все частіше ними починають користуватися люди для планування власного часу і вирішення особистих задач.


\subsection{Порівняння сучасних сервісів планування та інформування} На сьогодні, кожна людина так чи інакше стикається у повсякденному житті з системами, пов'язаними з контролем часу та завдань (рис.~\ref{fig:LightningOutlook}).

\addtwoimghere{MozillaLightning.png}{MSOutlook.png}{0.45}{Mozilla Lightning та MicroSoft Outlook}{fig:LightningOutlook}

\input{MicrosoftOutlook.tex}
\input{Lightning.tex}
\subsubsection{Google Calendar}

Google Calendar — безкоштовний веб-додаток для тайм-менеджменту розроблений Google. Інтерфейс подібний до аналогічних календарних додатків, таких як Microsoft Outlook. Має різні режими перегляду, зокрема денний, тижневий та місячний. Події зберігаються онлайн, а тому календар можна переглядати з будь-якого пристрою, обладнаного доступом до мережі Інтернет. Додаток може імпортувати та експортувати файли календаря різних форматів, а для існуючих — задавати різні права доступу. 

Слід зазначити, що Google Calendar, як і інші сервіси Google, має відкрите API, що дозволяє взаємодіяти з ним через власні додатки після відповідних налаштувань.


\subsection{Корпоративні інформаційні системи} \label{subsubs:KIS}

Корпоративна інформаційна система — це інформаційна система, яка підтримує автоматизацію функцій управління на підприємстві і постачає інформацію для прийняття управлінських рішень. У ній реалізована управлінська ідеологія, яка об'єднує бізнес-стратегію підприємства і прогресивні інформаційні технології.

У загальному визначенні «автоматизована система» — сукупність керованого об’єкта й автоматичних керувальних пристроїв, у якій частину функцій керування виконує людина. Вона представляє собою організаційно-технічну систему, що забезпечує вироблення рішень на основі автоматизації інформаційних процесів у різних сферах діяльності. 

Сучасні автоматизовані системи управління навчальним процесом у  закладах вищої освіти здатні вирішувати велику кількість функцій, а саме:
\begin{itemize}
	\item планування, контроль та аналіз навчальної діяльності;
	\item оперативний доступ до інформації про навчальний процес;
	\item єдину систему звітів, як внутрішніх, так і за вимогами МОН України;
	\item системи безпеки даних з урахуванням вимог законодавства;
	\item облік контингенту студентів та співробітників;
	\item проведення вступної кампанії;
	\item формування пакетів даних з метою виготовлення тих чи інших документів.
\end{itemize}

Функціонування будь-якої автоматизованої системи можна швидко адаптувати до особливостей навчального процесу конкретного навчального закладу, до локальних мереж різного рівня, що допомагає розширити коло користувачів (адміністрації, викладачів і студентів) для оперативного забезпечення їх необхідною інформацією. 

Отже, використання таких систем дає змогу не тільки удосконалити якість планування навчального процесу, а й оперативність управління ним.

Не зважаючи на всі переваги, які надає використання автоматизованих систем, досі далеко не в кожному закладі вони впроваджені чи використовуються в повній мірі з тих чи інших причин — інерційності поглядів адміністрації, супротив працівників або «саботаж» на місцях, відсутність фінансової або організаційної можливості.

\subsubsection{Інформаційно-аналітична система} \label{subs:ias}

В ХДУ використовується корпоративна інтегрована система «Інформаційно-аналітична система (IAS)». Вона дозволяє вести облік працівників і студентів, бухгалтерський облік, контроль за матеріальними цінностями тощо (рис.~\ref{fig:IasSubsustem}). 

\addimg{IasSubsustem.png}{0.7}{Структура ІАС}{fig:IasSubsustem}
		
Система дозволяє вносити і ефективно стежити за будь-якими змінами. В основі системи лежить ядро, на основі ядра виконується розширення системи до будь-якої кількості компонентів. При цьому основна функціональність може бути розширена за рахунок додаткових компонентів. 

Програма IAS орієнтована на платформу Windows з використанням MS SQL Server. Вона має багаторівневу архітектуру, що складається з бази даних, бізнес-логіки та клієнтського інтерфейсу. Внутрішній журнал реєстрації подій дозволяє вести та слідкувати за записами, що стосуються усіх подій.

Відсутність компонентів, пов’язаних з формуванням розкладу занять, та відсутність у використанні сторонніх рішень ставить задачу з проектування власного додатку для забезпечення всіх учасників освітнього процесу доступом до актуальної версії розкладу занять у будь-який час, а також можливості спрощення процесу формування розкладу та подальшої інформатизації освітнього процесу.
\subsubsection{Єдине інформаційно-освітнє серидовище ХДУ}

Разом з осучасненням освітнього процесу у цілому актуальною є заадча з об'єднання існуючих інформаційних систем університету.

В результаті проведеної інтеграції планується отримати так званий <<Особистий кабінет студента>>, в котрому студент, як основний учасник освітнього процесу, матиме доступ до всієї необхідної інформації. Наразі інформаційне серидовище включає в себе низку в цілому незалежних один від одного проектів, а саме:
\begin{itemize}
	\item web-портал університету \cite{KspuEdu};
	\item система дистанційного навчання <<KSU Online>> \cite{KsuOnline};
	\item система дистанційної освіти <<Херсонський Віртуальний Університет>> \cite{KsuDis}.
	\item програмний комплекс <<ST-Абітурієнт>>, що використовується для підтримки процесу прийому документів абітурієнтів та обробки заяв про зарахування, результатів вступних іспитів тощо;
	\item програмний комплекс <<Інформаційно-аналітична система (ІАС)>>, що дозволяє вести облік співробітників і студентів, бухгалтерський облік, контроль за матеріальними цінностями (розділ~\ref{subs:ias}), проте лише частково охоплює навчальний процес;
	\item сервіс <<KSU Feedback>> призначений для проведення анонімного або звичайного голосування за визначеними критеріями серед строго респондентів \cite{KsuFeedback};
	\item сервіс  <<Пошук книг в електронному каталозі бібліотеки>> надає доступ до каталогу в будь який момент \cite{eLibrary};
	\item web-портал <<Збірник наукових праць <<Інформаційні технології в освіті>> (ІТО)>> є каналом поширення та передачі знань, де вчені, практики та дослідники можуть обговорювати, аналізувати, критикувати, синтезувати, спілкуватися та підтримувати розробку та впровадження ІТ і пов'язаних з ними наслідків у всіх аспектах їх використання у сфері освіти \cite{ITO};
	\item web-портал <<Чорноморський ботанiчний журнал>>, но котрому у відкритому доступі знаходяться електронні версії всіх статей у форматі pdf, опублікованих у журналі з 2005 року.
\end{itemize}

При цьому наразі відсутня будь-яка інтеграція між сервісами та сайтами, крім посилань на певну частину з нах на головній сторінці web-порталу університету.


\section{ПРОЕКТУВАННЯ FRONT-END ЧАСТИНИ}
\subsection{Проектування бази даних}
\subsubsection{Розвиток комп’ютерних баз даних}

База даних~--- сукупність даних, організованих відповідно до певної прийнятої концепції, яка описує характеристику цих даних і взаємозв'язки між їхніми елементами. Дані у базі організовують відповідно до моделі організації даних. 

В загальному випадку базою даних можна вважати будь-який впорядкований набір даних, наприклад, паперову картотеку бібліотеки. Але все частіше термін «база даних» використовуєтьсяу контексті використання баз даних в інформаційних системах, як і самі бази даних переносяться в електронні системи в процесі інформатизації. На даний час додатки для роботи з базами даних є одними з найпоширеніших прикладних програм \cite{ситник2004проектування}.

Через тісний зв'язок баз даних з системами керування базами даних (СКБД) під терміном «база даних» нерідко неточно мається на увазі система керування базами даних. Але варто розрізняти базу даних — сховище даних, та СКБД — засоби для роботи з базою даних. Надалі, в роботі під терміном «база даних», в залежності від контексту, може матися на увазі як сукупність даних чи певні її параметри, так і СКБД, крім випадків де це не очевидно.

Розроблення перших баз даних розпочинається в 1960-ті роки. Переважно, дослідницькі роботи ведуться в проектах IBM та найбільших університетів. Пізніше, на початку 1970-х років Едгар Ф. Кодд обґрунтовує основи реляційної моделі \cite{codd1970relational}. Уперше цю модель було використано у бази даних Ingres та System R, що були лише дослідними прототипами. Проте вже в 1980-ті рр. з’являються перші комерційних версій реляційних БД Oracle та DB2. Реляційні бази даних починають успішно витісняти мережні та ієрархічні. Починаються дослідження розподілених (децентралізованих) баз даних.

\subsubsection{Реляційна модель даних}\label{subsection:relationModel}

Реляційна модель даних~--- логічна модель даних, вперше описана Едгаром Ф. Коддом \cite{codd1970relational}. В даний час ця модель є фактичним стандартом, на який орієнтуються більшість сучасних СКБД.

У реляційній моделі досягається більш високий рівень абстракції даних, ніж в ієрархічній або мережевій. Стверджується, що «реляційна модель надає засоби опису даних на основі тільки їх природної структури, тобто без потреби введення якоїсь додаткової структури для цілей машинного представлення»~ \cite{codd1970relational}. А це означає, що подання даних не залежить від способу їх фізичної організації, що забезпечується за рахунок використання математичного поняття відношення.

До складу реляційної моделі даних зазвичай включається теорія нормалізації. Дейт визначив наступні частини реляційної моделі даних~\cite{дейт2008введение}:
\begin{enumerate}
	\item структурна;
	\item маніпуляційна;
	\item цілісна.
\end{enumerate}

Структурна частина моделі визначає, що єдиною структурою даних є нормалізоване n-арне відношення.


\subsubsection{Нормалізація бази даних}

Нормалізація схеми бази даних~--- процес розбиття одного відношення (таблиці в поняттях СУБД) відповідно до алгоритму нормалізації на кілька відношень на основі функціональних залежностей.

Нормальна форма визначається як сукупність вимог, яким має задовольняти відношення, з точки зору надмірності, яка потенційно може призвести до логічно помилкових результатів вибірки.

Таким чином, схема реляційної бази даних покроково, у процесі виконання відповідного алгоритму, переходить у першу, другу, третю і так далі нормальні форми. Якщо відношення відповідає критеріям n-ої нормальної форми та всіх попередніх нормальних форм, тоді вважається, що це відношення знаходиться у нормальній формі n-ого рівня.

\subsubsection{СКБД PostgreSQL}

PostgreSQL~--- широко розповсюджена система керування базами даних з відкритим вихідним кодом. Прототип був розроблений в Каліфорнійському університеті Берклі в 1987 році, пізніше проект Берклі було зупинено, а реалізацію було викладено в Інтернет під назвою Postgres95 після вдосконалення вихідного коду. Наразі підтримкою й розробкою займається група спеціалістів, які добровільно приєднались до проекту.

Сервер PostgreSQL написаний на мові C. Розповсюджується у вигляді вихідного коду, який необхідно відкомпілювати. Разом з кодом розповсюджується детальна документація.

\subsubsection{Шаблон проектування ORM} \label{subs:orm}

ORM~--- шаблон програмування, який зв'язує бази даних з концепціями об'єктно\,--\,орієнтованих мов програмування, створюючи «віртуальну об'єктну базу даних». В об'єктно-орієнтованому програмуванні об'єкти в програмі представляють об'єкти з реального світу. 

Суть проблеми полягає в перетворенні таких об'єктів у форму, в якій вони можуть бути збережені у файлах або базах даних, і які легко можуть бути витягнуті в подальшому, зі збереженням властивостей об'єктів і відношень між ними. Ці об'єкти називають «постійними». Існує кілька підходів до розв'язання цієї задачі. Деякі пакети вирішують цю проблему, надаючи бібліотеки класів, здатних виконувати такі перетворення автоматично. Маючи список таблиць в базі даних і об'єктів в програмі, вони автоматично перетворять запити з одного вигляду в інший.

В проекті використано ORM Sequelize. Спроектовано на реалізовано у вигляді моделей та відповідним їм таблиць структуру бази даних (рис.~\ref{fig:DbScheme}).

\addCodeAsImg{\begin{umlstyle}


\umlclass[x=4, y=0, fill=green!30]{Teacher}{
	+ Surname \\
	+ Name \\
	+ Patronym \\
	+ Department \\
	+ Post \\
	}{}

\umlclass[x=9, y = 12, fill=red!30]{Post}{
	+ Name \\
	+ ShortName \\
}{}

\umlclass[x=8, y=-9, fill=red!30]{Group}{
	+ Name \\
	+ Department \\
	}{}

\umlclass[x=0, y=-3, fill=blue!30]{Department}{
	+ Name \\
	+ ShortName \\
	+ Faculty \\
	}{}
	
\umlclass[x=8, y=0, fill=green!30]{Discipline}{
	+ Name \\
	}{}
	
\umlclass[x=12, y=-3]{Lesson}{
	+ Discipline \\
	+ Teacher \\
	+ Group \\
	+ LessonForm \\
	+ LessonNumber \\
	+ Day \\
	+ Week \\
	+ Schedule \\
	}{}
	
\umlclass[x=12, y=3]{LessonForm}{
	+ Name \\
	+ ShortName \\	
}{}

\umlclass[x=10, y=6]{Faculty}{
	+ Name \\
	+ ShortName \\	
}{}

\umlclass[x=4, y=-9, fill=red!30]{LessonNumber}{
	+ Name \\
	}{}
	
\umlclass[x=8, y=-9, fill=red!30]{Day}{
	+ Name \\
	+ ShortName \\
	}{}

\umlclass[x=8, y=-3]{Week}{
	+ Name \\
	}{}
	
\umlclass[x=4, y=-6]{Worker}{
	+ Surname \\
	+ Patronym \\
	+ Faculty \\
	}{}
	
\umlclass[x=4, y=-9, fill=red!30]{Schedule}{
	+ Name \\
	+ Chair \\
	}{}
	
\umlclass[x=8, y=-9, fill=red!30]{Group}{
	+ Name \\
	+ DateBegin \\
	+ DateEnd \\
	+ Faculty \\
	}{}

\umlaggreg[geometry=|-,mult1=1, mult2=n, pos1=0.2, pos2=1.9]{Department}{Teacher}
\umlcompo[geometry=--,mult1=1, mult2=n, pos1=0.2, pos2=1.9]{Faculty}{Department}
\umlassoc[geometry=--,mult1=1, mult2=1, pos1=0.2, pos2=0.9]{Subject}{Teacher}
\umlcompo[geometry=|-,mult1=1, mult2=n, pos1=0.2, pos2=1.9]{Department}{Speciality}
\umlassoc[geometry=--,mult1=1, mult2=1, pos1=0.2, pos2=0.9]{Speciality}{Group}
\umlcompo[geometry=--,mult1=1, mult2=n, pos1=0.2, pos2=0.9]{Group}{Subgroup}
\umlassoc[geometry=--,mult1=n, mult2=1, pos1=0.2, pos2=0.9]{Schedule}{Faculty}
\umlcompo[geometry=--,mult1=1, mult2=n, pos1=0.2, pos2=1.9]{Schedule}{Lesson}
\umlassoc[geometry=|-,mult1=n, mult2=1, pos1=0.2, pos2=1.9]{Schedule}{Worker}
\umlassoc[geometry=--,mult1=, mult2=, pos1=0.2, pos2=1.9]{Faculty}{Worker}
\umlassoc[geometry=--,mult1=1, mult2=1, pos1=0.2, pos2=0.9]{Lesson}{Subgroup}
\umlassoc[geometry=|-,mult1=1, mult2=1, pos1=0.2, pos2=1.9]{Lesson}{Subject}

\end{umlstyle}
}{Схематична структура бази даних}{fig:DbScheme}

З погляду програміста система повинна виглядати як постійне сховище об'єктів. Він може просто створювати об'єкти і працювати з ними, а вони автоматично зберігатимуться в реляційній базі даних.

\input{RelationalDB.tex}
\input{DBNormalization.tex}
\input{PostgreSql.tex}
\input{Orm.tex}

\subsection{Аналіз існуючих бібліотек для розробки SPA}
\subsubsection{Аналіз існуючих бібліотек для розробки SPA}

На сьогодні, одними з росповсюджених фреймворків для розробки SPA (single page applications) є React, Angular та Vue (рис.~\ref{fig:ReactAngularVue}).

\addimg{ReactAngularVue.png}{0.35}{React, Angular та Vue}{fig:ReactAngularVue}

Angular -- Javascript-фреймворк, створений на основі TypeScript. Розроблений і підтримуваний компанією Google, він описується як JavaScript MVW-фреймворк. На даний момент останньою версією є 4. Фреймворк Angular використовується такими компаніями, як Google, Wix, weather.com, healthcare.gov і Forbes.

\label{subs:vue}
Vue -- ще один JS-фреймворк. Творці Vue описують його як «інтуїтивно зрозумілий та швидкий, призначений для створення інтерактивних інтерфейсів». Вперше він був представлений колишнім співробітником компанії Google Еваном Ю (Evan You) в лютому 2014 року. На даний момент фреймворк використовується такими компаніями, як Alibaba, Baidu, Expedia, Nintendo, GitLab.

\subsubsection{Вимоги до додатку}

Одним із завдань, поставлених для реалізації мети є розроблення вимог щодо можливостей веб-додатку та його інтерфейсу.

Після проведення аналізу предметної області, додатків аналогів та технологій було сформульовано наступні вимоги:

\begin{enumerate}
    \item Веб-додаток повинен коректно відражатися у останніх версіях популярних веб-браузерів(Google Chrome, Mozilla Firefox, Opera, Safari).
    \item Адаптивність інтерфейсу до розмірів вікна браузера або пристрою.
    \item Можливість зміни типу відображення поточного розкладу.
    \item Зміна інтерфейсу відповідно до прав доступу поточного користувача.
    \item Створення нових користувачів у системі адміністратором.
    \item Авторизація користувачів у системі.
    \item Редагування даних користувачами системи у відповідності до їх прав доступу.
    \item Редагування даних адміністраторами системи у відповідності до їх прав доступу.
    \item Інтеграція з Google сервісами, зокрема Google calendar та Google Sheets.
    \item Інтерфейс для імпорту та експорту даних між системою та сервісами Google або специфікованими форматами даних.
    \item Забезпечення цілісності даних.
\end{enumerate}



\subsection{React}
\subsubsection{React} \label{subs:React}


При розробці мобільного додатку використано фреймворк React Native, в основі якого знаходиться бібліотека React, призначена для створення користувацьких інтерфейсів. На відміну від React, призначеного для розробки односторінкових веб-додатків, React Native спрямований на мобільні платформи~\cite{davidgeary2019}.

React~--- відкрита JavaScript бібліотека для створення інтерфейсів користувача, яка покликана вирішувати проблеми часткового оновлення вмісту веб-сторінки, з якими стикаються в розробці односторінкових застосунків~\cite{ericmasiello2017}. React використовують Facrbook, Airbnb, Uber, Netflix, Twitter, Pinterest, Reddit, Udemy, Wix, Paypal, Imgur, Feedly, Stripe, Tumblr, Walmart та інші.

React дозволяє розробникам створювати великі веб-додатки, які використовують дані, котрі змінюються з часом, без перезавантаження сторінки. React обробляє тільки користувацький інтерфейс у застосунках. Це відповідає видові у шаблоні модель-вид-контролер (MVC) і може бути використане у поєднанні з іншими JavaScript бібліотеками або в великих фреймворках MVC, таких як AngularJS. Він також може бути використаний з React на основі надбудов, щоб піклуватися про частини без користувацького інтерфейсу побудови веб-застосунків.

React підтримує віртуальний DOM, а не покладається виключно на DOM браузера. Це дозволяє бібліотеці визначити, які частини DOM змінилися, порівняно зі збереженою версією віртуального DOM, і таким чином визначити, як найефективніше оновити DOM браузера. Як бібліотека інтерфейсу користувача React часто використовується разом з іншими бібліотеками, такими як Redux, проте у його використанні при розробці проекту не було необхідності.

React надає розробникам безліч методів, які викликаються під час життєвого циклу компонента (рис.~\ref{fig:ReactReduxCommunicate}), вони дозволяють нам оновлювати UI і стан додатку. Коли необхідно використовувати кожен з них, що необхідно робити і в яких методах, а від чого краще відмовитися, є ключовим моментом до розуміння як працювати з React.

\addimg{ReactReduxCommunicate.png}{0.85}{Взаємодія Redux та React}{fig:ReactReduxCommunicate}

Конструктори є основною ООП — це спеціальна функція, яка буде викликатися щоразу, коли створюється новий об'єкт. Важливо викликати функцію super в випадках, коли наш клас розширює поведінку іншого класу, який має конструктор. Виконання цієї спеціальної функції буде викликати конструктор нашого батьківського класу і дозволяти йому проініціаліззувати себе. 
Конструктори~--- це відмінне місце для ініціалізації компонента~--- створення будь-яких полів (змінні, що починаються з this.).

Це також єдине місце де слід встановлювати стан безпосередньо перезаписуючи поле this.state. У всіх інших випадках необхідно використовувати $this.setState$.

За замовчуванням, всі компоненти будуть перемальовувати себе всякий раз, коли їх стан змінюється, змінюється контекст або вони приймають props від батьківського компонента. Якщо перерисовка компонента досить важка (наприклад генерація графіка), то у розробників є доступ до спеціальної функції, яка дозволяє контролювати цей процес.

\subsection{Проектування та прототипування front-end частини}
\input{Frontend.tex}


\subsection{Подальша робота}
\subsubsection{Використання QR-кодів}
\addimg{QRcode.png}{0.25}{Приклад QR-коду з посиланням}{fig:QRcode}

Було проаналізовано перспективи при використанні QR-кодів (рис.~\ref{fig:QRcode}) з метою супроводження традиційного паперового розкладу (та інших документів), що публікується на стендах.

Хоча термін «QR code» є зареєстрованим товарним знаком японської корпорації «DENSO Corporation», їх використання не обкладається ніякими ліцензійними відрахуваннями, коди описані та опубліковані як стандарти ISO~\cite{воронкін2014можливості}. Основна перевага QR-коду – легке розпізнавання скануючим обладнанням (за допомогою мобільного телефону, планшета або ноутбука з камерою, на яких встановлена програма для зчитування кодів, тощо).

Одним з способів використання QR-кодів в навчальному процесі, крім запропонованих (зокрема, задля забезпечення швидкого доступу до навчально-методичного забезпечення, довідкової літератури, веб-сервісів навчального закладу) \cite{воронкін2014можливості},  можна назвати надання доступу до електронної версії розкладу.


\anonsection{ВИСНОВКИ}

Для виконання поставлених завдань було проведено аналіз характеристик існуючих систем планування, зокрема обсяг їх можливостей. При підготовці до проектування було приділено увагу окремим частинам процесу підготовки розкладу на прикладі факультету комп’ютерних наук, фізики та математики ХДУ.

На основі проведеного аналізу розроблено базові вимоги щодо можливостей додатку та його інтерфейсу.
Суттєву частину роботи приділено аналізу існуючих технологій всіх рівнів для створення веб-додатків. Детально досліджено роботу клієнт-серверних додатків та проектуванню API. 

Відповідно до створених вимог розроблено проект додатку та бекенд частини. Розроблено робочий прототип бекенд частини (зокрема реалізовано структуру бази даних засобами PostgreSQL, моделі з використанням ORM Squalize та окремі частини API і інтерфейсу додатку.

Сформовано проект документації до публічного API. При написанні ключових частин використано спеціальну форму коментарів, що забезпечують інтеграцію опису функцій та їх параметрів в підказки популярних IDE (інтегрованих середовищ розробки). Останнє є корисним при подальшій розробці, особливо при використанні існуючої кодової бази сторонніми розробниками, що є цілком можливим, зважаючи на модульність проекту при використанні мікросервісної архітектури.

При розробці проекту використовується система контролю версій git з публічним репозиторієм на сервісі GitHub (github.com/ Rembut/gCalShedule), що дозволяє використовувати сучасні методи сумісної роботи та, одночасно з тим, дозволяє використовувати результати проведеного дослідження всім охочим під ліцензією MIT.
 % Заключение
\include{BIBLIOGRAPHY} % Библиографический список

\end{document}
%%% Конец документа
