\subsubsection{Розробка компонентів}

В термінах React, всі частини-відображення іменуються компонентами. В роботі спроектована серія компонентів для різних частин системи та розроблено прототипи деяких з них. 

На рис.~\ref{fig:AdminPanelUserCreation} зображено компонент для створення адміністратором нового користувача системи. Слід наголосити, що кожен компонент є окремою частиною і тому можливий для використання у подальшому в інших системах при виконанні певних вимог (так зване «повторне використання»).

\addimg{AdminPanelUserCreation.png}{0.85}{Адміністративна панель (створення користувача)}{fig:AdminPanelUserCreation}

На приведеному вище зображенні натискання на кожну з кнопок призводить до виклику відповідного методу API шляхом надсилання певного HTTP запиту.

Перед доступом до адміністративної панелі (рис.~\ref{fig:AdminPanelUserManagement}) адміністратору необхідно авторизуватися у системі, в результаті чого буде створено і збережено силами його веб-браузеру JWT. Після цього, якщо він має відповідні права  та верифікація токену пройшла успішно, йому буде відображена відповідна панель.

\addimg{AdminPanelUserManagement.png}{0.85}{Адміністративна панель (керування користувачами)}{fig:AdminPanelUserManagement}

Слід зауважити, що подібний процес перевірки відбувається при виконанні кожного запиту, крім тих, що не потребують авторизації (доступні для незареєстрованих користувачів).

Перша прерогатива адміністратора системи~--- створення нових користувачів, при цьому відбувається вищеописана процедура.

Певна частина об’єктів системи може вважатися більш-менш константною, це структура закладу вищої освіти (рис.~\ref{fig:AdminPanelFacultyManagement}), окремі словникові дані (зокрема, назви посад професорсько-викладацького складу та види занять).

\addimg{AdminPanelFacultyManagement.png}{0.85}{Адміністративна панель (керування факультетами)}{fig:AdminPanelFacultyManagement}

Окремо можна відзначити використання об’єктів часу (номера занять впродовж дня, рис.~\ref{fig:AdminPanelTimeManagement}).  В спроектованій системі одним з можливих шляхів доступу до розкладу є його експорт до сервісу Google Calendar в серію календарів. 

\addimg{AdminPanelTimeManagement.png}{0.85}{Адміністративна панель (керування часом)}{fig:AdminPanelTimeManagement}

В подальшому, потенційні користувачі можуть підписуватися на оновлення відповідних календарів (зокрема, груп та викладачів) для отримання актуальної інформації в довільний момент часу.
