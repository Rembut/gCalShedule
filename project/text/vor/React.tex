\subsubsection{React} \label{subs:React}


При розробці мобільного додатку використано фреймворк React Native, в основі якого знаходиться бібліотека React, призначена для створення користувацьких інтерфейсів. На відміну від React, призначеного для розробки односторінкових веб-додатків, React Native спрямований на мобільні платформи~\cite{davidgeary2019}.

React~--- відкрита JavaScript бібліотека для створення інтерфейсів користувача, яка покликана вирішувати проблеми часткового оновлення вмісту веб-сторінки, з якими стикаються в розробці односторінкових застосунків~\cite{ericmasiello2017}. React використовують Facrbook, Airbnb, Uber, Netflix, Twitter, Pinterest, Reddit, Udemy, Wix, Paypal, Imgur, Feedly, Stripe, Tumblr, Walmart та інші.

React дозволяє розробникам створювати великі веб-додатки, які використовують дані, котрі змінюються з часом, без перезавантаження сторінки. React обробляє тільки користувацький інтерфейс у застосунках. Це відповідає видові у шаблоні модель-вид-контролер (MVC) і може бути використане у поєднанні з іншими JavaScript бібліотеками або в великих фреймворках MVC, таких як AngularJS. Він також може бути використаний з React на основі надбудов, щоб піклуватися про частини без користувацького інтерфейсу побудови веб-застосунків.

React підтримує віртуальний DOM, а не покладається виключно на DOM браузера. Це дозволяє бібліотеці визначити, які частини DOM змінилися, порівняно зі збереженою версією віртуального DOM, і таким чином визначити, як найефективніше оновити DOM браузера. Як бібліотека інтерфейсу користувача React часто використовується разом з іншими бібліотеками, такими як Redux, проте у його використанні при розробці проекту не було необхідності.

React надає розробникам безліч методів, які викликаються під час життєвого циклу компонента (рис.~\ref{fig:ReactReduxCommunicate}), вони дозволяють нам оновлювати UI і стан додатку. Коли необхідно використовувати кожен з них, що необхідно робити і в яких методах, а від чого краще відмовитися, є ключовим моментом до розуміння як працювати з React.

\addimg{ReactReduxCommunicate.png}{0.85}{Взаємодія Redux та React}{fig:ReactReduxCommunicate}

Конструктори є основною ООП — це спеціальна функція, яка буде викликатися щоразу, коли створюється новий об'єкт. Важливо викликати функцію super в випадках, коли наш клас розширює поведінку іншого класу, який має конструктор. Виконання цієї спеціальної функції буде викликати конструктор нашого батьківського класу і дозволяти йому проініціаліззувати себе. 
Конструктори~--- це відмінне місце для ініціалізації компонента~--- створення будь-яких полів (змінні, що починаються з this.).

Це також єдине місце де слід встановлювати стан безпосередньо перезаписуючи поле this.state. У всіх інших випадках необхідно використовувати $this.setState$.

За замовчуванням, всі компоненти будуть перемальовувати себе всякий раз, коли їх стан змінюється, змінюється контекст або вони приймають props від батьківського компонента. Якщо перерисовка компонента досить важка (наприклад генерація графіка), то у розробників є доступ до спеціальної функції, яка дозволяє контролювати цей процес.
