\subsubsection{Babel}

Нажаль, при постійному розвитку мов програмування невід'ємною є ситуація, що реалізація часто відстає від специфікації. Більш того, різні реалізації по-різному відстають від специфікації. Написавши код, ми не можемо гарантувати, де він буде запускатися, а де - ні.

Виходячи з цього можна зробити висновок, що потрібно писати код, дотримуючись старих стандартів. На щастя, є інший шлях: ми можемо писати код з використанням всіх найновіших можливостей, але перед публікацією автоматично транслювати його (тобто переводити з одного виду в інший) в стару версію. 

Сама природа JS і його способи використання готують нас до того, що ніколи не настане моменту, коли у всіх користувачів буде остання версія інтерпретатору. Люди використовували і продовжать використовувати різні браузери і різні версії браузерів, різні версії Node.js і так далі. Використання нових синтаксичних конструкцій в такій ситуації практично неможливо. Запуск коду на платформі що не підтримує новий синтаксис призведе до синтаксичну помилку. 

Закономірним вирішенням цієї проблеми стала поява Babel - програми, яка бере вказаний код і повертає той же код, але транслювали в стару версію JS. Фактично, в сучасному світі Babel став невід'ємною частиною JS. Всі нові проекти так чи інакше розробляють з його використанням.
