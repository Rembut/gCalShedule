\section{ПРОЕКТУВАННЯ БАЗИ ДАНИХ}

\subsection{Історія виникнення комп’ютерних баз даних}

База даних – сукупність даних, організованих відповідно до певної прийнятої концепції, яка описує характеристику цих даних і взаємозв'язки між їхніми елементами. Дані у базі організовують відповідно до моделі організації даних. 

В загальному випадку базою даних можна вважати будь-який впорядкований набір даних, наприклад, паперову картотеку бібліотеки. Але все частіше термін «база даних» використовуєтьсяу контексті використання баз даних в інформаційних системах, як і самі бази даних переносяться в електронні системи в процесі інформатизації. На даний час додатки для роботи з базами даних є одними з найпоширеніших прикладних програм \cite{ситник2004проектування}.

Через тісний зв'язок баз даних з системами керування базами даних (СКБД) під терміном «база даних» нерідко неточно мається на увазі система керування базами даних. Але варто розрізняти базу даних — сховище даних, та СКБД — засоби для роботи з базою даних. Надалі, в роботі під терміном «база даних», в залежності від контексту, може матися на увазі як сукупність даних чи певні її параметри, так і СКБД, крім випадків де це не очевидно.

Розроблення перших баз даних розпочинається в 1960-ті роки. Переважно, дослідницькі роботи ведуться в проектах IBM та найбільших університетів. Пізніше, на початку 1970-х років Едгар Ф. Кодд обґрунтовує основи реляційної моделі \cite{codd1970relational}. Уперше цю модель було використано у бази даних Ingres та System R, що були лише дослідними прототипами. Проте вже в 1980-ті рр. з’являються перші комерційних версій реляційних БД Oracle та DB2. Реляційні бази даних починають успішно витісняти мережні та ієрархічні. Починаються дослідження розподілених (децентралізованих) баз даних.

\subsection{Реляційні бази даних}\label{subsection:relationModel}

Реляційна модель даних — логічна модель даних, вперше описана Едгаром Ф. Коддом \cite{codd1970relational}. В даний час ця модель є фактичним стандартом, на який орієнтуються більшість сучасних СКБД.

У реляційній моделі досягається більш високий рівень абстракції даних, ніж в ієрархічній або мережевій. Стверджується, що «реляційна модель надає засоби опису даних на основі тільки їх природної структури, тобто без потреби введення якоїсь додаткової структури для цілей машинного представлення» \cite{codd1970relational}. А це означає, що подання даних не залежить від способу їх фізичної організації, що забезпечується за рахунок використання математичного поняття відношення.

До складу реляційної моделі даних зазвичай включається теорія нормалізації. Дейт визначив наступні частини реляційної моделі даних \cite{дейт2008введение}:
\begin{itemize}
	\item структурна;
	\item маніпуляційна;
	\item цілісна.
\end{itemize}

Структурна частина моделі визначає, що єдиною структурою даних є нормалізоване n-арне відношення.

\subsubsection{Нормалізація бази даних}

Нормалізація схеми бази даних — процес розбиття одного відношення (таблиці в поняттях СУБД) відповідно до алгоритму нормалізації на кілька відношень на основі функціональних залежностей.

Нормальна форма визначається як сукупність вимог, яким має задовольняти відношення, з точки зору надмірності, яка потенційно може призвести до логічно помилкових результатів вибірки.

Таким чином, схема реляційної бази даних покроково, у процесі виконання відповідного алгоритму, переходить у першу, другу, третю і так далі нормальні форми. Якщо відношення відповідає критеріям n-ої нормальної форми та всіх попередніх нормальних форм, тоді вважається, що це відношення знаходиться у нормальній формі n-ого рівня.

\subsubsection{СКБД PostgreSQL}

PostgreSQL — широко розповсюджена система керування базами даних з відкритим вихідним кодом. Прототип був розроблений в Каліфорнійському університеті Берклі в 1987 році, пізніше проект Берклі було зупинено, а реалізацію було викладено в Інтернет під назвою Postgres95 після вдосконалення вихідного коду. Наразі підтримкою й розробкою займається група спеціалістів, які добровільно приєднались до проекту.

Сервер PostgreSQL написаний на мові C. Розповсюджується у вигляді вихідного коду, який необхідно відкомпілювати. Разом з кодом розповсюджується детальна документація.

\subsection{Технологія ORM}

ORM — технологія програмування, яка зв'язує бази даних з концепціями об'єктно-орієнтованих мов програмування, створюючи «віртуальну об'єктну базу даних». В об'єктно-орієнтованому програмуванні об'єкти в програмі представляють об'єкти з реального світу. 

Суть проблеми полягає в перетворенні таких об'єктів у форму, в якій вони можуть бути збережені у файлах або базах даних, і які легко можуть бути витягнуті в подальшому, зі збереженням властивостей об'єктів і відношень між ними. Ці об'єкти називають «постійними». Існує кілька підходів до розв'язання цієї задачі. Деякі пакети вирішують цю проблему, надаючи бібліотеки класів, здатних виконувати такі перетворення автоматично. Маючи список таблиць в базі даних і об'єктів в програмі, вони автоматично перетворять запити з одного вигляду в інший.

В проекті використано ORM Sequelize. Спроектовано на реалізовано у вигляді моделей та відповідним їм таблиць структуру бази даних (рис.~\ref{fig:DbScheme}).

\addCodeAsImg{\begin{umlstyle}

\umlclass[x=0, y=-3, fill=blue!30]{Department}{
	+ Name \\
	}{}
	
\umlclass[x=4, y=-3, fill=blue!30]{Faculty}{
	+ Name \\
	}{}
	
\umlclass[x=4, y=0, fill=green!30]{Teacher}{
	+ Name \\
	+ Department \\
	+ Post \\
	}{}
	
\umlclass[x=8, y=0, fill=green!30]{Subject}{
	+ Discipline \\
	+ Type \\
	+ Teacher \\
	}{}
	
\umlclass[x=12, y=-3]{Lesson}{
	+ Subject \\
	+ Subgroup \\
	+ Teacher \\
	+ Audience \\
	+ Time \\
	}{}
	
\umlclass[x=8, y=-3]{Schedule}{
	+ Name \\
	+ Using time \\
	+ Faculty \\
	+ Worker \\
	}{}
	
\umlclass[x=4, y=-6]{Worker}{
	+ Name \\
	+ Faculty \\
	+ Auth data \\
	}{}
	
\umlclass[x=4, y=-9, fill=red!30]{Speciality}{
	+ Name \\
	+ Chair \\
	}{}
	
\umlclass[x=8, y=-9, fill=red!30]{Group}{
	+ Name \\
	+ Speciality \\
	+ Cource \\
	+ Number \\
	}{}
	
\umlclass[x=12, y=-9, fill=red!30]{Subgroup}{
	+ Group \\
	+ Specialization \\
	}{}

\umlaggreg[geometry=|-,mult1=1, mult2=n, pos1=0.2, pos2=1.9]{Department}{Teacher}
\umlcompo[geometry=--,mult1=1, mult2=n, pos1=0.2, pos2=1.9]{Faculty}{Department}
\umlassoc[geometry=--,mult1=1, mult2=1, pos1=0.2, pos2=0.9]{Subject}{Teacher}
\umlcompo[geometry=|-,mult1=1, mult2=n, pos1=0.2, pos2=1.9]{Department}{Speciality}
\umlassoc[geometry=--,mult1=1, mult2=1, pos1=0.2, pos2=0.9]{Speciality}{Group}
\umlcompo[geometry=--,mult1=1, mult2=n, pos1=0.2, pos2=0.9]{Group}{Subgroup}
\umlassoc[geometry=--,mult1=n, mult2=1, pos1=0.2, pos2=0.9]{Schedule}{Faculty}
\umlcompo[geometry=--,mult1=1, mult2=n, pos1=0.2, pos2=1.9]{Schedule}{Lesson}
\umlassoc[geometry=|-,mult1=n, mult2=1, pos1=0.2, pos2=1.9]{Schedule}{Worker}
\umlassoc[geometry=--,mult1=, mult2=, pos1=0.2, pos2=1.9]{Faculty}{Worker}
\umlassoc[geometry=--,mult1=1, mult2=1, pos1=0.2, pos2=0.9]{Lesson}{Subgroup}
\umlassoc[geometry=|-,mult1=1, mult2=1, pos1=0.2, pos2=1.9]{Lesson}{Subject}

\end{umlstyle}}{Структура бази даних}{fig:DbScheme}

З погляду програміста система повинна виглядати як постійне сховище об'єктів. Він може просто створювати об'єкти і працювати з ними, а вони автоматично зберігатимуться в реляційній базі даних.
