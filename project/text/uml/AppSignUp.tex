\begin{umlstyle}

\begin{umlseqdiag}
	\umlactor[no ddots, x=1]{Admin}
	\umlboundary[no ddots, x=5]{App}
	\umldatabase[no ddots, x=14, fill=blue!20]{DB}
	
	\begin{umlcall}[op=Sign up request, type=synchron, return=Response, padding=3]{Admin}{App}
	
		\begin{umlfragment}[type=SignUp]
			\umlcreatecall[no ddots, x=8]{App}{DbUser}
			\begin{umlcall}[op=Init, type=synchron, return=Response]{App}{DbUser}
				\begin{umlcall}[op=Set data, type=synchron, return=]{DbUser}{DbUser}
					\umlcreatecall[no ddots, x=11]{DbUser}{DbRole}
					\begin{umlcall}[op=Set data, type=synchron, return=Role]{DbUser}{DbRole}
						\begin{umlcall}[op=Get all, type=synchron, return=List]{DbRole}{DB}\end{umlcall}
					\end{umlcall}
				\end{umlcall}
				
				\begin{umlfragment}[type=Creating, label=OK, fill=green!10]				
					\begin{umlcall}[op=Store, type=synchron]{DbUser}{DB}\end{umlcall}
					\begin{umlcall}[op=Success, type=synchron]{DbUser}{DbUser}\end{umlcall}
					\umlfpart[Error]				
					\begin{umlcall}[op=Error, type=synchron]{DbUser}{DbUser}\end{umlcall}
				\end{umlfragment}

			\end{umlcall}
		\end{umlfragment}
		
	\end{umlcall}
	
\end{umlseqdiag}

\end{umlstyle}
