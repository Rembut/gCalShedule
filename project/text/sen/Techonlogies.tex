У розробці веб-додатків на сьогодняшній день використувується велика кількість мов програмування, таких як: C#, Java, JavaScript, Python, PHP. Кожен розробник обирає для себе ту мову, яка йому здається найбільш відповідною для певної задачі. В загалом усі вище перераховані мови, окрім, може JavaScript, традиційно використовуються для розробки бекенд-частини веб-ресурсів, або для генерації фронтенду на серверній стороні та відправки сгенерованої сторінки клієнту. Для кожної конкретної мови є певні фреймворки, що спрощують створення веб-додатків.

Фреймворк - це інфраструктура програмних рішень, що полегшує розробку складних систем.

Взагалом фреймворки можуть включати в себе бібліотеки, що створені для вирішення певних задач та найкращі сталі практики з вирішення тих чи інших питань. Головне завдання фреймворків - зменшити об'єм однотипної праці, що розробник додатку виконує у кожному своєму проекті. Хоча кожна мова програмування й має свої фреймворки, їх сутність у цілому залишається для певних задач доволі близькою: використання певних шаблонів проектування, що допомагають зробити програмний код більш зрозумілим та простим для подальшої підтримки та ускладнення.

Найбільш поширеними фреймворками є: для C# є .NET, для Java - Spring, Hibernate, для JavaScript - Node, React, Vue, Angular, для Python - Django, для PHP - Laravel. Кожен фреймворк та мова програмування мають свої переваги та недоліки. Тому, коли перед нами постало завдання вибрати, на якому саме стеку технологій ми будемо розробляти веб-додаток, нами було проаналізовано кожну із вищепредставлених мов.
Наш вібір зупинився на JavaScript як для серверної, так і для клієнтської частини. Адже, якщо серверна та клієнтська частина написани з використанням однієї мови програмування, це сильно спрощує розробку та підримку веб-додатку. Для серверної частини нами було вирішено використовувати фреймворк Node.js, тому що він добре поєднується із будь-яким фронтенд фреймворком, має велике коло шанувальників, постійно оновлюється, що підвищує його безпечність, та хорошу документацію.  
