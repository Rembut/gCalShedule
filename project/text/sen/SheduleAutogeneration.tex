\subsubsection{Автоматична генерація розкладу занять}

Наступною задачею є автоматизація процесу  генерації розкладу занять окремого структурного підрозділу та закладу освіти у цілому. 

Розклад сам по собі залежить від багатьох факторів. Їх можна розділити на об'єктивні (жорсткі) та суб'єктивні (непостійні) параметри. Об'єктивні~--- це база даних університету, в якій зберігається інформація про аудиторії та предмети. Суб'єктивні~--- це побажання студентів та викладачів.

Розклад~--- це саме по собі поняття тривіальне з точки зору сучасного життя, а от задачу його формування тяжко назвати тривіальною. За класичним означенням, розклад~--- це документ підприємства, який регламентує робочий ритм, визначає часові обмеження всіх робочих процесів і формує оптимальне розділення такого важливого ресурсу як час.

При реалізації таких задач використовуються різні підходи, частина з яких може використовувати складний математичний апарат та алгоритми, пов'язані з елементами штучного інтелекту \cite{рубан2013аналіз}, генетичними алгоритмами \cite{мулява2016система} тощо.
