\subsubsection{Семантичне версіювання} \label{subs:semver}

В процесі розробки програмного забезпечення можливе виникнення проблеми під назвою <<пекло залежностей>>. 

Ця проблема полягає в тому, що при збільшенні розмірів програмної системи, збільшується кількість бібліотек та пакетів, що використовуються в ній. При цьому, кожен з них, зазвичай, вимагає для своєї роботи деякі інші бібліотеки певних версій. У разі, якщо документація програмного забезпечення надто вільна, то рано чи піздно виникає проблема невідповідності між фактично необхідною версією, вказаною в документації та встановленою, що негативно позначається на всьому процесі розробки програмного забезпечення.

Для вирішення цієї проблеми пропонується простий набір правил і вимог, що визначають як встановлюються і збільнуються номери версій. Для роботи системи необхідно створити і описати публічне API програмного продукту. Після цього будь-які зміни в версії визначаються певною зміної її номера.

Розглянемо формат версій X.Y.Z (мажорна, мінорна, патч).

Зміни, що не впливають на API, змінюють номер патч-версії. Зворотньо-сумістні зміни та розширення API збільшують мінорну версію. І, нарешті, несумістні зміни API збільшують мажорну версію.

Ця система називатиметься <<Семантичне версіювання>>.

Мажорна версія <<0>> (0.Y.Z) призначена для початкової розробки, публічний API не має розглядатися як стабільний. Версія 1.0.0 визначає публічний API, після цього релізу вона змінюватиметься відповідно до змін в API. Після чергової зміни мінорної версії патч-версія змінюється на <<0>>, аналогічні зміни відбуваються зі зміною мажорної версії.

Крім зазначених правил, специфікація семантичного версіонування~\cite{semver} визначає додатково певні деталі та поради щодо його практичного використання, зокрема для продуктів, що мають складну систему релізів та передрелізних версій.
