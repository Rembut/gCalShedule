У розробці веб-додатків на сьогодняшній день використовується велика кількість мов програмування, таких як: C\#, Java, JavaScript, Python, PHP. Кожен розробник обирає для себе ту мову, яка йому здається найбільш відповідною для певної задачі. В загалом усі вище перераховані мови, окрім, може JavaScript, традиційно використовуються для розробки бекенд-частини веб-ресурсів, або для генерації фронтенду на серверній стороні та відправки сгенерованої сторінки клієнту. Для кожної конкретної мови є певні фреймворки, що спрощують створення веб-додатків.

Фреймворк - це інфраструктура програмних рішень, що полегшує розробку складних систем.

Взагалом фреймворки можуть включати в себе бібліотеки, що створені для вирішення певних задач та найкращі сталі практики з вирішення тих чи інших питань. Головне завдання фреймворків - зменшити об'єм однотипної праці, що розробник додатку виконує у кожному своєму проекті. Хоча кожна мова програмування й має свої фреймворки, їх сутність у цілому залишається для певних задач доволі близькою: використання певних шаблонів проектування, що допомагають зробити програмний код більш зрозумілим та простим для подальшої підтримки та ускладнення.

Найбільш поширеними фреймворками є: для C\# є .NET, для Java - Spring, Hibernate, для JavaScript - Node, React, Vue, Angular, для Python - Django, для PHP - Laravel. Кожен фреймворк та мова програмування мають свої переваги та недоліки. Тому, коли перед нами постало завдання вибрати, на якому саме стеку технологій ми будемо розробляти веб-додаток, нами було проаналізовано кожну із вищепредставлених мов.

Наш вибір зупинився на JavaScript як для серверної, так і для клієнтської частини. Адже, якщо серверна та клієнтська частина написана з використанням однієї мови програмування, це сильно спрощує розробку та підтримку веб-додатку. Для серверної частини нами було вирішено використовувати фреймворк Node.js, тому що він добре поєднується із будь-яким фронтенд фреймворком, має велике коло шанувальників, постійно оновлюється, що підвищує його безпечність, та хорошу документацію.  

\subsubsection{Семантичне версіювання} \label{subs:semver}

В процесі розробки програмного забезпечення можливе виникнення проблеми під назвою <<пекло залежностей>>. 

Ця проблема полягає в тому, що при збільшенні розмірів програмної системи, збільшується кількість бібліотек та пакетів, що використовуються в ній. При цьому, кожен з них, зазвичай, вимагає для своєї роботи деякі інші бібліотеки певних версій. У разі, якщо документація програмного забезпечення надто вільна, то рано чи піздно виникає проблема невідповідності між фактично необхідною версією, вказаною в документації та встановленою, що негативно позначається на всьому процесі розробки програмного забезпечення.

Для вирішення цієї проблеми пропонується простий набір правил і вимог, що визначають як встановлюються і збільнуються номери версій. Для роботи системи необхідно створити і описати публічне API програмного продукту. Після цього будь-які зміни в версії визначаються певною зміної її номера.

Розглянемо формат версій X.Y.Z (мажорна, мінорна, патч).

Зміни, що не впливають на API, змінюють номер патч-версії. Зворотньо-сумістні зміни та розширення API збільшують мінорну версію. І, нарешті, несумістні зміни API збільшують мажорну версію.

Ця система називатиметься <<Семантичне версіювання>>.

Мажорна версія <<0>> (0.Y.Z) призначена для початкової розробки, публічний API не має розглядатися як стабільний. Версія 1.0.0 визначає публічний API, після цього релізу вона змінюватиметься відповідно до змін в API. Після чергової зміни мінорної версії патч-версія змінюється на <<0>>, аналогічні зміни відбуваються зі зміною мажорної версії.

Крім зазначених правил, специфікація семантичного версіонування~\cite{semver} визначає додатково певні деталі та поради щодо його практичного використання, зокрема для продуктів, що мають складну систему релізів та передрелізних версій.

\subsubsection{LaTeX} \label{subsub:latex}

\LaTeX -- це високоякісна набірна система; він включає функції, призначені для виготовлення технічної та наукової документації. LaTeX є фактичним стандартом для комунікації та публікації наукових документів \cite{lamport1994latex}. LaTeX доступний як вільне програмне забезпечення.

\TeX -- це створена чудовим американським математиком і програмістом Дональдом Кнутом система для верстки текстів з формулами. Сам по собі TEX є спеціалізованою мовою програмування (Кнут не тільки придумав мову, а й написав для нього транслятор, причому таким чином, що він працює абсолютно однаково на самих різних комп'ютерах), на якому пишуться видавничі системи, що використовуються на практиці. Точніше кажучи, кожна видавнича система на базі TEXа є пакетом макросів (макропакет) цієї мови. LATEX -- це створена Леслі Лампортом видавнича система на базі TEXа~\cite{львовский2003latex}.

Всі видавничі системи на базі TEXа володіють перевагами, закладеними в самому TEXе. Для новачка їх можна описати однією фразою: надрукований текст виглядає «зовсім як у книзі». LATEX, як видавнича система, надає зручні і гнучкі засоби досягти цього книжкового якості. Зокрема, вказавши за допомогою простих засобів структуру тексту, автор може не вникати в деталі оформлення, причому ці деталі при необхідності неважко змінити (щоб, скажімо, змінити шрифт, яким друкуються заголовки, не треба нишпорити по всьому тексту, змінюючи все заголовки , а досить замінити одну сходинку в «стильовому файлі»). Такі речі, як нумерація розділів, посилання, зміст і т. П. Виходять майже що «самі собою». Величезним плюсом систем на базі TEXа є висока якість та гнучкість форматування абзаців і математичних формул (в останньому відношенні краще TEXа цю задачу не вирішує жодна програма).

TEX (і всі видавничі системи на його базі) невибагливий до використовуваної техніки. З іншого сторони, TEXовські файли мають високий ступінь переносимості: Ви можете підготувати LATEXовський вихідний текст на своєму IBM PC, переслати його до видавництва, і бути впевненими, що там Ваш текст буде правильно оброблений і на друку вийде в точності те ж, що вийшло у Вас при пробному друку на Вашому улюбленому матричному принтері (з тією єдиною різницею, що фотоскладальний автомат дасть текст більш високої якості). Завдяки цій обставині TEX став дуже популярний як мова міжнародного обміну статтями з математики та фізики.

Є у TEXа і недоліки. Перший з них (розділяється TEXом з усіма іншими видавничими системами) такий: він працює відносно повільно, займає багато пам'яті. Друга особливість TEXа, яка може не сподобатися тим, хто звик до традиційних редакторів офісних пакетів -- це те, що він не є системою типу WYSIWYG ("те що ти бачиш, те ти і отримаєш"): робота з вихідним текстом і перегляд того, як текст буде виглядати після друку -- різні операції. Втім, завдяки цій особливості час на підготовку тексту істотно скорочується.

При роботі над звітом також використано сервіс Overleaf -- сучасний інструмент, розроблений у 2012 році. Він був створений щоб допомогти редагувати свої наукові статті, технічні звіти, тези, презентації, блок-схеми та інші документи, написані на мові розмітки LaTeX. 

\subsubsection{Sublime Text}

Sublime Text - прорієтарний текстовий редактор. Підтримує плагіни на мові програмування Python.

Розробник дає можливість безкоштовно і без обмежень ознайомитися з редактором, однак програма періодично буде повідомляти про необхідність придбання ліцензії.

Редактор містить різні візуальні теми, а також можливість завантаження додаткових тем.

Коли користувач набере код, Sublime Text, в залежності від використовуваного мови, буде пропонувати різні варіанти для завершення запису. Також редактор може автоматично додавати розділові знаки (<<\{>>, <<\}>>, <<;>>).

Sublime Text дозволяє збирати програми у готовий проект і запускати їх без необхідності використання зовнішньої командної строки. Користувач також може налаштувати свою систему компіляції і включити автоматичну збірку програм кожного разу при збереженні коду. Ця система схожа з відповідним плагіном для зйомки тексту від віддаленого LaTeX (розділ~\ref{subsub:latex}).

Використовується плагін LaTeXTools. Плагін LaTeXTools надає кілька функцій, які спрощують роботу з файлами LaTeX.

Команда ST збирає компіляцію джерела LaTeX у PDF за допомогою texify (Windows / MikTeX) або latexmk (OSX / MacTeX, Windows / TeXlive, Linux / TeXlive). Потім він розбирає файл журналу і перераховує помилки та попередження. Нарешті, він запускає програму перегляду PDF і, на підтримуваних переглядачах (Sumatra PDF на Windows, Skim на OSX і Evince на Linux за замовчуванням) переходить до поточної позиції курсора.

Додатково реалізована функція автозбереження, що допомагає користувачам не втратити пророблену роботу.

\subsubsection{Система контролю версій Git}

Активну популярність мають розподілені системи контролю версій (SVC).

Найбільш поширеними з таких є Subversion (SVN), Microsoft Visual Source Safe (VSS), Revision Control System (RCS), Concurrent Versions System (CVS), Gіt та Mercurіal. Знання подібних систем підвищує затребуваність ІT фахівців на ринку праці, покращує продуктивність розробників та полегшує рішення щоденних завдань. Саме передача знань є вирішальною у процесі експорту-імпорту технологій~\cite{киричек2012модель}.

Під час дослідження було використано систему контролю версій Git з віддаленим репозиторієм на сервісі GitHub~\cite{gCalShedule}.

Система контролю дозволяє зберігати попередні версії файлів та завантажувати їх за потребою. Вона зберігає повну інформацію про версію кожного з файлів, а також повну структуру проекту на всіх стадіях розробки. Місце зберігання даних файлів називають репозиторієм. В середині кожного з репозиторіїв можуть бути створені паралельні лінії розробки -- гілки~\cite{loeliger2012version}.

Git підтримує швидке розділення та злиття версій, містить можливості для візуалізації і навігації за нелінійною історією розробки. 


\subsubsection{Node.JS}

Node.js - кросплатформенне середовище виконання з відкритим вихідним кодом, яка дозволяє розробникам створювати серверні інструменти і додатки використовуючи мову JavaScript. Середовище виконання призначене для використання поза контекстом браузера (тобто виконується безпосередньо на комп'ютері або на серверній ОС). Таким чином, середа виключає API-інтерфейси JavaScript для браузера і додає підтримку більш традиційних OS API-інтерфейсів, включаючи бібліотеки HTTP (детальніше про HTTP методи в пункті~\ref{subs:crud}) і файлових систем.

JavaScript є відносно новою мовою програмування і має певні переваги в порівнянні з іншими мовами програмування для веб-серверів (PHP, Python і т.д.).

Багато інших мов програмування компілюються (конвертуються) в JavaScript, тому можна також використовувати CoffeeScript, ClosureScript, Scala, LiveScript тощо.

Менеджер пакетів Node (NPM) забезпечує доступ до сотень тисяч багаторазових пакетів. Він також має краще в своєму класі розподілення залежностей і може використовуватися для автоматизації більшості інструментів побудови. Деякі проблеми роботи з залежностями та версіями пакетів і засобами їх вирішення розглянуто в пункті~\ref{subs:semver}. 

Він портативний, має версії для Microsoft Windows, OS X, Linux, Solaris, FreeBSD, OpenBSD, WebOS, і NonStop OS. Крім того, він має хорошу підтримку серед багатьох хостинг-провайдерів, які часто надають конкретну інфраструктуру і документацію для розміщення сайтів, що працюють на Node.

З точки зору веб-серверної розробки Node має ряд переваг.

Node був розроблений для оптимізації пропускної здатності і масштабованості в веб-додатках і дуже добре справляється з багатьма поширеними проблемами веб-розробки (наприклад, веб-додатки реального часу).

Код написаний на «звичайному JavaScript», а це означає, що витрачається менше часу при написанні коду для браузера і веб-сервера пов'язане з «перемиканням технологій» між мовами.

JavaScript є відносно новою мовою програмування і має переваги від поліпшення дизайну мови в порівнянні з іншими традиційними мовами для веб-серверів (наприклад, Python, PHP, і т.д.). Велика частина інших популярних мов програмування компілюється / конвертується в JavaScript, тому ви можете також використовувати CoffeeScript, ClosureScript, Scala, LiveScript, etc.

Менеджер пакетів Node (NPM) забезпечує доступ до сотень тисяч багаторазових пакетів. Він також має краще в своєму класі дозвіл залежностей і може також використовуватися для автоматизації більшості інструментів побудови.

Він портативний, має версії для Microsoft Windows, OS X, Linux, Solaris, FreeBSD, OpenBSD, WebOS, і NonStop OS. Крім того, він має хорошу підтримку серед багатьох хостинг-провайдерів, які часто надають конкретну інфраструктуру і документацію для розміщення сайтів, що працюють на Node.

Він має дуже розвинену екосистему та спільноту розробників, які завжди готові допомогти.

Ви можете створити простий веб-сервер для відповіді на будь-який запит (рис.~\ref{fig:httpSample}), використовуючи тільки HTTP-пакет Node, як показано нижче. Це створить сервер і прослухає будь HTTP-запит на URL $http://127.0.0.1:8000/$; коли він буде отриманий, він відправить текстову відповідь «Hello World».

\addCodeAsImg{\lstinputlisting[numbers=left]{code/NodeJsHttp.tex}}{Приклад HTTP серверу}{fig:httpSample}

Express -- найпопулярніший веб-фреймворк для Node (детальніше в пункті~\ref{par:Express}). Він є базовою бібліотекою для ряду інших популярних веб-фреймворків Node. Він надає наступні механізми:

\begin{enumerate}
	\item Написання обробників для запитів з різними HTTP-методами в різних URL-адресах (в пункті~\ref{subs:crud}).
	\item Інтеграцію з механізмами рендеринга «view», для генерації відповідей, вставляючи дані в шаблони.
	\item Установка загальних параметрів веб-додатку, такі як порт для підключення і розташування шаблонів, які використовуються для відображення відповіді.
	\item «Проміжне ПО» для додаткової обробки запиту в будь-який момент в конвеєрі обробки запитів (приклади такого використання в пункті \ref{subs:middleware}).
	\item У той час як сам Express досить мінімалістичний, розробники створили сумісні пакети проміжного програмного забезпечення для вирішення практично будь-якої проблеми з веб-розробкою. Існують бібліотеки для роботи з кукі-файлами, сеансами, входами користувачів, параметрами URL, даними POST, заголовками безпеки і багатьма іншими. Опубліковані списки рекомендованих розробниками пакетів проміжного програмного забезпечення, підтримуваних командою Express в Express Middleware (поряд зі списком деяких популярних пакетів сторонніх виробників).
\end{enumerate}


