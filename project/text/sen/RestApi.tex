\subsubsection{REST API}

REST — підхід до архітектури мережевих протоколів, які забезпечують доступ до інформаційних ресурсів. Був описаний і популяризований 2000 року Роєм Філдінгом, одним із творців протоколу HTTP. В основі REST закладено принципи функціонування Всесвітньої павутини і, зокрема, можливості HTTP. Філдінг розробив REST паралельно з HTTP 1.1, при цьому базувався на попередньому протоколі HTTP 1.0.

Дані можуть передаватися у вигляді певних стандартних форматів (наприклад, HTML, XML, JSON). Будь-який REST протокол (HTTP в тому числі) має підтримувати кешування, не повинен залежати від мережевого прошарку, не повинен зберігати стан системи між парами «запит-відповідь». Такий підхід забезпечує масштабовність системи і дозволяє їй розвиватись з отриманням нових вимог. Ці особливості сприяють використанню REST API при проектуванні мікросервісних додатків \cite{кучер2018мікросервісна}.

REST, як і кожен архітектурний стиль відповідає ряду обмежень архітектури додатку. Це гібридний стиль який успадковує обмеження з інших архітектурних стилів.

\paragraph{Клієнт-сервер}

Основна архітектура від якої він успадковує обмеження — це клієнт-сервер. Вона вимагає розділення відповідальності між частинами системи, які займаються зберіганням та обробкою даних (сервером), і тими компонентами, які займаються відображенням даних на стороні користувача та дії з цим інтерфейсом (клієнтом). Таке розділення дозволяє компонентам розвиватись незалежно один від одного та спрощує внесення змін у систему.

\paragraph{Відсутність стану}

Ще одним обмеженням є те, що акти взаємодії між сервером та клієнтом не мають стану, а отже кожен окремий запит містить всю необхідну інформацію для його обробки, і не використовує при цьому те, що сервер знає ту чи іншу інформацію з попереднього запиту.

Проте відсутність стану не означає що стану немає. Відсутність стану означає, що сервер не знає про стан клієнта. Коли клієнт, наприклад, запитує головну сторінку сайту, сервер відповідає на запитання і забуває про клієнта. Клієнт може залишити сторінку відкритою протягом кількох років, перш ніж натиснути посилання, і тоді сервер відповість на наступний запит. Тим часом сервер може відповідати на запити інших клієнтів, або нічого не робити — для клієнта це не має значення.

Таким чином, наприклад дані про стан сесії (користувача, який виконав вхід до системи) зберігаються на клієнті, і передаються окремо разом з кожним запитом. Це покращує масштабовність, бо сервер після закінчення обробки запиту може звільнити всі ресурси, задіяні для цієї операції, для використання їх в обробці інших запитів, без жодного ризику втратити цінну інформацію. Також спрощується контроль і пошук помилок, бо для аналізу того, що відбувається в певному запиті, досить переглянути лише на той один запит. Збільшується надійність, адже помилка в одному запиті не зачіпає інші.
