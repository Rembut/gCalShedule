\subsubsection{REST API}

REST — підхід до архітектури мережевих протоколів, які забезпечують доступ до інформаційних ресурсів. Був описаний і популяризований 2000 року Роєм Філдінгом, одним із творців протоколу HTTP. В основі REST закладено принципи функціонування Всесвітньої павутини і, зокрема, можливості HTTP. Філдінг розробив REST паралельно з HTTP 1.1 базуючись на попередньому протоколі HTTP 1.0.

Дані повинні передаватися у вигляді невеликої кількості стандартних форматів (наприклад, HTML, XML, JSON). Будь-який REST протокол (HTTP в тому числі) повинен підтримувати кешування, не повинен залежати від мережевого прошарку, не повинен зберігати інформації про стан між парами «запит-відповідь». Стверджується, що такий підхід забезпечує масштабовність системи і дозволяє їй еволюціонувати з новими вимогами. Ці особливості сприяють використанню REST API при проектуванні мікросервісних додатків \cite[158]{кучер2018мікросервісна}.

REST, як і кожен архітектурний стиль відповідає ряду архітектурних обмежень (англ. architectural constraints). Це гібридний стиль який успадковує обмеження з інших архітектурних стилів.

\paragraph{Клієнт-сервер}

Перша архітектура від якої він успадковує обмеження — це клієнт-серверна архітектура. Її обмеження вимагає розділення відповідальності між компонентами, які займаються зберіганням та оновленням даних (сервером), і тими компонентами, які займаються відображенням даних на інтерфейсі користувача та реагування на дії з цим інтерфейсом (клієнтом). Таке розділення дозволяє компонентам еволюціонувати незалежно.

\paragraph{Відсутність стану}

Наступним обмеженням є те, що взаємодії між сервером та клієнтом не мають стану, тобто кожен запит містить всю необхідну інформацію для його обробки, і не покладається на те, що сервер знає щось з попереднього запиту.

Відсутність стану не означає що стану немає. Відсутність стану означає, що сервер не знає про стан клієнта. Коли клієнт, наприклад, запитує головну сторінку сайту, сервер відповідає на запитання і забуває про клієнта. Клієнт може залишити сторінку відкритою протягом кількох років, перш ніж натиснути посилання, і тоді сервер відповість на інший запит. Тим часом сервер може відповідати на запити інших клієнтів, або нічого не робити — для клієнта це не має значення.

Таким чином, наприклад дані про стан сесії (користувача, який автентифікувався) зберігаються на клієнті, і передаються з кожним запитом. Це покращує масштабовність, бо сервер після закінчення обробки запиту може звільнити всі ресурси, задіяні для цієї операції, без жодного ризику втратити цінну інформацію. Також спрощується моніторинг і зневадження, бо для того аби розібратись, що відбувається в певному запиті, досить подивитись лише на той один запит. Збільшується надійність, бо помилка в одному запиті не зачіпає інші.
s