На сьогодні, кожна людина так чи інакше стикається у повсякденному житті з системами, пов'язаними з контролем часу та завдань. Крім цього, такі технології знаходять застосування в освітньому процесі~\cite{ліщина2014проблеми}.

\subsubsection{Microsoft Outlook}

Microsoft Outlook — додаток-органайзер, входить до пакету офісних програм Microsoft Office. Дозволяє працювати з електронною поштою, надає функції календаря, планувальника завдань, записника і менеджера контактів. Крім того, Outlook дозволяє відстежувати роботу з документами пакету Microsoft Office для автоматичного складання щоденника роботи~\cite{франчук2016використання}.

Outlook може використовуватися  і як окремий додаток, так і виступати в ролі клієнта для Microsoft Exchange Server, що надає додаткові функції для спільної роботи всіх користувачів організації: загальні поштові скриньки, папки завдань, календарі, планування часу загальних зустрічей, узгодження документів тощо.

Крім цього, дозволяє підключати через протоколи POP3/IMAP інші поштові сервіси та додатки, що надаються ними. Зокрема, нижче буде розглянуто синхронізацію MS Outlook з сервісами Google.

\subsubsection{Lightning}

Lightning — проект Mozilla Foundation, що додає функції календаря і планувальника в Mozilla Thunderbird — безкоштовну кросплатформну програму для роботи з електронною поштою і новинами, що може вважатися відкритим аналогом для відповідних продуктів з пакету Microsoft Office.


\subsubsection{Google Calendar}

Google Calendar — безкоштовний веб-додаток для тайм-менеджменту розроблений Google. Інтерфейс подібний до аналогічних календарних додатків, таких як Microsoft Outlook. Має різні режими перегляду, зокрема денний, тижневий та місячний. Події зберігаються онлайн, а тому календар можна переглядати з будь-якого пристрою, обладнаного доступом до мережі Інтернет. Додаток може імпортувати та експортувати файли календаря різних форматів, а для існуючих — задавати різні права доступу~\cite{олексюк2013деякі}. 

Слід зазначити, що Google Calendar, як і інші сервіси Google, має відкрите API, що дозволяє взаємодіяти з ним через власні додатки після відповідних налаштувань.

Окремо слід звернути увагу на розвинені технології вбудовування документів Google (зокрема календарів Google Calendar) у власні веб додатки. 

Одним з прикладів такого використання в контексті розвитку інформаційної інфраструктури університету можна навести інтеграцію календаря подій факультету комп'ютерних наук, фізики та математики ХДУ в відповідну сторінку (kspu.edu/About/Faculty/FPhysMathemInformatics.aspx) на офіційному веб-сайті (рис.~\ref{fig:CalendarKspuEdu}).

\addimg{CalendarKspuEdu.png}{1}{Календар подій факультету}{fig:CalendarKspuEdu}

В наведеному прикладі події різних календарів, об'єднаних для відображення відображаються різними кольорами, в назву події включено час початку, а при натисканні на неї - відображаються деталі, зокрема опис, місце та посилання на подію в Google Calendar, де, крім іншого, можливо додати її для відслідковування та нагадування у власний календар, за умови, якщо користувач попередньо авторизувався в свій акаунт.

Документи, для яких встановлені публічні права для перегляду, можна включати в вихідний код сайту у вигляді фрейму. Фрейм — окремий HTML-документ, який сам чи разом з іншими документами відображений у вікні веб-переглядача. При цьому, всю відповідальність за відображуване в фреймі несе сервіс-власник, тобто Google Calendar в наведеному прикладі, а в місці відображення знаходиться лише код інтеграції з посиланням та супутніми параметрами~\cite{ліщина2014проблеми}.

Інший приклад використання мікросервісів (детальніше про мікросервіси в розділі~\ref{subsec:microservices}) та фреймів у контексті розвитку інформаційної інфраструктури~--- сервіс замовлення довідки про навчання в університеті (працює для студентів факультету комп'ютерних наук, фізики та математики, рис.~\ref{fig:RequestCertificateFromDeansOffice}).

\addimg{RequestCertificateFromDeansOffice.png}{0.75}{Форма замовлення довідки}{fig:RequestCertificateFromDeansOffice}

Після заповнення відповідної форми (не є справжньою формою на основі тегу <form> в розумінні HTML у зв'язку з обмеженнями сайту, проте реалізована з використанням його компонентів; поведінку цілком відтворено з допомогою javascript) відбувається запит до сервісу EmailJS.com, котрий, у свою чергу, надсилає лист за шаблоном на основі запиту з сайту. Сервіс має певні обмеження щодо кількості листів в проміжок часу, проце цілком задовільняє поставлені вимоги. Після отримання відповідного листа на корпоративну робочу пошту, відповідальний за підготовку довідок працівник деканату формує її в інформаційно-аналітичній системі університету  (детальніше в розділі~\ref{subsubs:KIS}) та виконує інші необхідні операції. В результаті зменшується навантаження на працівника та виключається необхідність  для студента у попередньому зверненні для запису; зменшується кількість <<паперових>> процесів що сприятливо впливає на подальшій інформатизції освітнього процесу.

При генерації коду фрейму для інтеграції у адміністратора є можливість налаштувати колірне оформлення фрейму, його розміри, регіональні стандарти (мову відображення, день початку тиждня, часовий пояс), обсяг за замовчуванням (тиждень, місяць), додати або приховати елементи керування. В процесі редагування налаштувань отримується невеликий за обсягом код (HTML тег <iframe>, рис.~\ref{fig:CalendarIframe}) для розміщення в коді власної веб-сторінки. 

\addCodeAsImg{\lstinputlisting[numbers=left]{code/CalendarIframe.tex}}{Код інтеграції календаря факультету}{fig:CalendarIframe}

Аналогічним чином інтегруються інші сервіси Google, що вже знайшло використання при розміщенні матеріалів на сайті, як то презентації, текстові документи, таблиці, карти, що одночасно підтверджує, по-перше, перспективність використання хмарних сервісів для поступового осучаснення інформаційної інфраструктури та, по друге, можливість переходу до використання їх замість звичних офісних пакетів (Microsoft Office, Open Office, Libre Office тощо).


