\anonsection{ВИСНОВКИ}

Для виконання поставлених завдань було проведено аналіз характеристик існуючих систем планування, зокрема обсяг їх можливостей.

При підготовці до проектування було приділено увагу окремим частинам процесу підготовки розкладу на факультеті комп’ютерних наук, фізики та математики ХДУ.

На основі проведеного аналізу розроблено базові вимоги щодо можливостей серверної частини. Суттєву частину роботи приділено аналізу існуючих технологій всіх рівнів для створення веб-додатків веб-сервісів. Детально досліджено роботу клієнт-серверних додатків та супутніх технологій. Розглянуто та обгрунтовано використання мікрсервісного підходу при проектуванні системи. 

Відповідно до створених вимог розроблено серверну частину спроектованої системи. Після проведення аналізу популярних технологій розробки веб-сервісів для реалізації основної частини системи обрано Node.js з бібліотекою Express. Реалізовано структуру бази даних засобами PostgreSQL, моделі з використанням ORM Squalize.

Розроблено публічний прикладний програмний інтерфейс (API), та сформовано документацію до нього. При написанні ключових частин використано спеціальну форму коментарів, що забезпечують інтеграцію опису функцій та їх параметрів в підказки популярних IDE. Останнє є корисним при подальшій розробці, особливо при використанні існуючої кодової бази сторонніми розробниками, що є можливим, зважаючи на модульність проекту при використанні мікросервісної архітектури.

При розробці проекту використовується система контролю версій git з публічним репозиторієм на сервісі GitHub~\cite{gCalShedule}, що дозволяє використовувати сучасні методи сумісної роботи та дозволяє використовувати результати проведеного дослідження всім охочим під ліцензією MIT.
