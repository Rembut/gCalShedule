\anonsection{ВИСНОВКИ}

Для виконання поставлених завдань було проведено аналіз характеристики існуючих систем планування, зокрема обсяг їх можливостей. При підготовці до проектування було приділено увагу окремим частини процесу підготовки розкладу на прикладі факультету комп’ютерних наук, фізики та математики ХДУ.

На основі проведеного аналізу розроблено базові вимоги щодо можливостей додатку та його інтерфейсу.
Суттєву частку роботи приділено аналізу існуючих технологій всіх рівнів для створення веб-додатків. Детально досліджено роботу клієнт-серверних додатків та проектуванню API. 

Відповідно до створених вимог розробити проект додатку та бекенд частини. Розробити робочий прототип бекенд частини (зокрема реалізувано структуру бази даних засобами PostgreSQL, моделі з використанням ORM Squalize та окремі частини API і інтерфейсу додатку.

Сформовано проект документації до публічного API. При написанні ключових частин використано спеціальну форму коментарів, що забезпечують інтеграцію опису функцій та їх параметрів в підказки популярних IDE (інтегрованих середовищ розробки). Останнє є корисним при подальшій розробці, особливо при використанні існуючої кодової бази сторонніми розробниками, що є цілком можливим, зважаючи на модульність проекту при використанні мікросервісної архітектури.

При розробці проекту використовується система контролю версій git з публічним репозиторієм на сервісі GitHub (github.com/ Rembut/gCalShedule), що дозволяє використовувати сучасні методи сумісної роботи та, одночасно з тим, дозволяє використовувати результати проведеного дослідження всім охочим під ліцензією MIT.
