\subsubsection{Бібліотеки Chai та Mocha}

\addCodeAsImg{\lstinputlisting[numbers=left]{code/ChaiMochaSample.tex}}{Перевірка з використанням Chai та Mocha}{fig:ChaiMochaSample}

MochaJS - це JavaScript фреймворк, який використовується для автоматичного тестування додатків. Він може використовуватися як на стороні сервера Javascript, так і в браузері. 

ChaiJS - це бібліотека для node.js і, як Mocha, Chai може використовуватися на стороні сервера або в браузері. Chai може бути використаний спільно з будь-якою бібліотекою для тестування.

Ми описуємо, що ми хочемо перевірити, використовуючи describe() (на рис.~\ref{fig:ChaiMochaSample}). Функція приймає два параметри: String і callback. Цей рядок може бути будь-яким.

it() використовується для опису того, що буде протестовано в цьому блоці коду. Дозволено писати вкладені описи describe() і it().

В процесі роботи використовувалися засоби як ручного, так і автоматичного тестування. 
Можна окремо виділити тести наступних підсистем:

\begin{enumerate}
    \item запуск системи та відсутність  критичних помилок часу виконання на початковому етапі;
    \item підключення до бази даних;
    \item виконання CRUD-операцій над об'єктами;
    \item підключення до сторонніх сервісів, що використовуються в роботі;
    \item успішна взаємодія з Google API.
\end{enumerate}

