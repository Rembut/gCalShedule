\subsubsection{Моделі та CRUD-операції} \label{subs:crud}

В процесі проектування закладено серію моделей, що відповідають об’єктам предметної області. В рамках системи зберігаються в базі даних у вигляді таблиць з певними взаємозвязками (реляційну модель описано в підрозділі \ref{subsection:relationModel}). Для доступу до даних використовуються основні HTTP методи, що відповідають операціям CRUD (від Create, Read, Update, Delete), їх перелічено нижче.

\paragraph{GET}
\addCodeAsImg{% Auth sequence uml diagram

\documentclass[a4paper,10pt]{article}
\usepackage[english]{babel}
\usepackage[left=3cm, right=1cm, top=1cm, bottom=1cm]{geometry}

\usepackage{tikz-uml}

\sloppy
\hyphenpenalty 10000000

\begin{document}
\thispagestyle{empty}
\begin{center}
\begin{tikzpicture}
\begin{umlseqdiag}
	\umlactor[no ddots, x=1]{User}
	\umlboundary[no ddots, x=5]{App}
	\umldatabase[no ddots, x=14, fill=blue!20]{DB}
	
	\begin{umlcall}[op=Post request, type=synchron, return=Response, padding=3]{User}{App}
		\umlcreatecall[no ddots, x=8]{App}{JWT}
		\begin{umlcall}[op=Init, type=synchron, return=Response]{App}{JWT}
			\begin{umlcall}[op=Verify JWT, type=synchron]{JWT}{JWT}\end{umlcall}
		\end{umlcall}
		
		\begin{umlfragment}[type=Main, label=OK]
	
			\umlcreatecall[no ddots, x=11]{App}{Object}
			\begin{umlcall}[op=Parameters, type=synchron, return=Object]{App}{Object}
				\begin{umlcall}[op=Select query, type=synchron, return=Rows]{Object}{DB}\end{umlcall}
					
			\end{umlcall}	
			
			\umlfpart[Error]
			
			\begin{umlcall}[op=Error, type=synchron]{App}{App}\end{umlcall}
		
		\end{umlfragment}
	\end{umlcall}
		
	
\end{umlseqdiag}
\end{tikzpicture}
\end{center}

\end{document}
}{Виконання запиту на отримання об’єкта}{fig:ReadOperation}

Запитує вміст вказаного ресурсу, який може приймати параметри, що передаються в URI (рис.~\ref{fig:ReadOperation}). Згідно зі стандартом, ці запити є ідемпотентними — багатократне повторення одного і того ж запиту GET приводить до однакових результатів (за умови, що сам ресурс не змінився за час між запитами).

В запропонованій реалізації запит GET має дві версії — з параметром (ID) та без нього. Останній виконує дію (надає користувачу) не до конкретного об’єкту, а до всієї множини, що є необхідним в певних ситуаціях (наприклад, відображення списку всіх викладачів за певним критерієм).

\paragraph{HEAD}

Аналогічний GET, за винятком того, що у відповіді сервера відсутнє тіло. Це може бути необхідно для отримання мета-інформації.

\paragraph{POST}
\addCodeAsImg{% Auth sequence uml diagram

\documentclass[a4paper,10pt]{article}
\usepackage[english]{babel}
\usepackage[left=3cm, right=1cm, top=1cm, bottom=1cm]{geometry}

\usepackage{tikz-uml}

\sloppy
\hyphenpenalty 10000000

\begin{document}
\thispagestyle{empty}
\begin{center}
\begin{tikzpicture}
\begin{umlseqdiag}
	\umlactor[no ddots, x=1]{User}
	\umlboundary[no ddots, x=5]{App}
	\umldatabase[no ddots, x=14, fill=blue!20]{DB}
	
	\begin{umlcall}[op=Post request, type=synchron, return=Response, padding=3]{User}{App}
		\umlcreatecall[no ddots, x=8]{App}{JWT}
		\begin{umlcall}[op=Init, type=synchron, return=Response]{App}{JWT}
			\begin{umlcall}[op=Verify JWT, type=synchron]{JWT}{JWT}\end{umlcall}
		\end{umlcall}
		
		\begin{umlfragment}[type=Main, label=OK]
	
			\begin{umlfragment}[type=Create, fill=green!20]
				\umlcreatecall[no ddots, x=11]{App}{Object}
				\begin{umlcall}[op=Init, type=synchron, return=Object]{App}{Object}
					\begin{umlcall}[op=Store, type=synchron]{Object}{DB}\end{umlcall}
						
				\end{umlcall}	
			\end{umlfragment}
			
			\umlfpart[Error]
			
			\begin{umlcall}[op=Undo creation, type=synchron, return=Error]{App}{Object}\end{umlcall}
		
		\end{umlfragment}
	\end{umlcall}
		
	
\end{umlseqdiag}
\end{tikzpicture}
\end{center}

\end{document}



, label=OK, fill=green!10]
						\begin{umlcall}[op=Create token, type=synchron, return=Token]{DbUser}{JWT}\end{umlcall}		
						\begin{umlcall}[op=Success, type=synchron]{DbUser}{DbUser}\end{umlcall}
						\umlfpart[Error]		
						
						
\begin{umlcall}[op=Find one, type=synchron, return=User]{Object}{DB}\end{umlcall}	
				
				\begin{umlcall}[op=Check password, type=synchron, return=Response]{Object}{JWT}\end{umlcall}
				
				\begin{umlfragment}[type=Validate, label=OK, fill=green!10]
					\begin{umlcall}[op=Create token, type=synchron, return=Token]{Object}{JWT}\end{umlcall}		
					\begin{umlcall}[op=Success, type=synchron]{Object}{Object}\end{umlcall}
					\umlfpart[Error]				
					\begin{umlcall}[op=Error, type=synchron]{Object}{Object}\end{umlcall}
				\end{umlfragment}}{Виконання запиту на створення з аутентифікацією}{fig:CreateOperation}

Передає дані (наприклад, з форми на веб-сторінці) заданому ресурсу. При цьому передані дані включаються в тіло запиту. На відміну від методу GET, метод POST не є ідемпотентним, тобто багатократне повторення одних і тих же запитів POST може повертати різні результати (рис.~\ref{fig:CreateOperation}).

На першому етапі відбувається перевірка доступу користувача это створення об’єкту цього типу (авторизація), відповідно до прав доступу (рис.~\ref{fig:ApiAccess}).

\paragraph{PUT}
\addCodeAsImg{\begin{umlstyle}

\begin{umlseqdiag}
	\umlactor[no ddots, x=1]{User}
	\umlboundary[no ddots, x=5]{App}
	\umldatabase[no ddots, x=14, fill=blue!20]{DB}
	
	\begin{umlcall}[op=path request, type=synchron, return=sesponse, padding=3]{User}{App}
		\begin{umlcall}[op=auth procedure, type=synchron]{App}{App}\end{umlcall}
		
		\begin{umlfragment}[type=Update, label=OK, fill=green!20]
				\umlcreatecall[no ddots, x=11]{App}{Object}
				\begin{umlcall}[op=parameters, type=synchron, return=object]{App}{Object}
					\begin{umlcall}[op=select query, type=synchron, return=rows]{Object}{DB}\end{umlcall}
					\begin{umlcall}[op=change, type=synchron]{Object}{Object}\end{umlcall}
					\begin{umlcall}[op=store, type=synchron, return=result]{Object}{DB}\end{umlcall}
				\end{umlcall}	
			
			\umlfpart[Error]
			
			\begin{umlcall}[op=error, type=synchron]{App}{App}\end{umlcall}
		
		\end{umlfragment}
	\end{umlcall}
		
	
\end{umlseqdiag}

\end{umlstyle}
}{Виконання запиту на модифікацію існуючого об’єкту}{fig:UpdateOperation}

Завантажує вказаний ресурс на сервер. В розроблюваній системі використовується для редагування існуючих даних (рис.~\ref{fig:UpdateOperation}). 

В процесі виконання, спочатку з бази даних силами ORM вибирається конкретний об’єкт, в нього вносяться зміни, після чого він записується до сховища на заміну попередньої версії.

\paragraph{PATCH}

Завантажує частину ресурсу на сервер. При розробці необхідності у використанні не знайдено.

\paragraph{DELETE}
\addCodeAsImg{\begin{umlstyle}

\begin{umlseqdiag}
	\umlactor[no ddots, x=1]{User}
	\umlboundary[no ddots, x=5]{App}
	\umldatabase[no ddots, x=14, fill=blue!20]{DB}
	
	\begin{umlcall}[op=delete request, type=synchron, return=sesponse, padding=3]{User}{App}
		\begin{umlcall}[op=auth procedure, type=synchron]{App}{App}\end{umlcall}
		
		\begin{umlfragment}[type=Delete, label=OK, fill=green!20]
				\umlcreatecall[no ddots, x=11]{App}{Object}
				\begin{umlcall}[op=parameters, type=synchron, return=object]{App}{Object}
					\begin{umlcall}[op=select query, type=synchron, return=rows]{Object}{DB}\end{umlcall}
					\begin{umlcall}[op=set timestamp, type=synchron]{Object}{Object}\end{umlcall}
					\begin{umlcall}[op=store, type=synchron, return=result]{Object}{DB}\end{umlcall}
				\end{umlcall}	
				
				
			\umlfpart[Error]
			
			\begin{umlcall}[op=error, type=synchron]{App}{App}\end{umlcall}
		
		\end{umlfragment}
		
	\end{umlcall}
		
	\umlnote[x=8, y=-5.25, fill=cyan!20]{Object}{Record don't deletes really, but deleting timestamp sets to current}
	
\end{umlseqdiag}

\end{umlstyle}
}{Виконання запиту на видалення об’єкту}{fig:DeleteOperation}

Видаляє вказаний ресурс.
Слід звернути увагу, що в процесі виконання запиту на видалення об’єкту в системі, видалення як такого не відбувається. Замість цього в окреме поле таблиці вноситься інформація про час виконання цієї процедури (рис.~\ref{fig:DeleteOperation}).

Такий спосіб реалізації дозволяє з однієї сторони приховати дані, відмічені як видалені від подальшого використання, а з іншої — зберегти їх там, де вони вже використовуються. В іншому випадку, у зв’язку з реляційністю бази потрібно було б вирішувати дилему — або проводити циклічне видалення для збереження цілісності даних, втрачаючи всі об’єкти, що посилаються на той, що видаляється; або ускладнювати структури даних, що потенційно призведе до дублювання даних.
