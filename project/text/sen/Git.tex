\subsubsection{Система контролю версій Git}

Активну популярність мають розподілені системи контролю версій (SCM).

Найбільш поширеними з таких є Subversion (SVN), Microsoft Visual Source Safe (VSS), Revision Control System (RCS), Concurrent Versions System (CVS), Gіt та Mercurіal. Знання подібних систем підвищує затребуваність ІT фахівців на ринку праці, покращує продуктивність розробників та полегшує рішення щоденних завдань. Саме передача знань є вирішальною у процесі експорту-імпорту технологій~\cite{киричек2012модель}.

В процесі роботи використано систему контролю версій Git з віддаленим репозиторієм на сервісі GitHub~\cite{gCalShedule}.

Система контролю дозволяє зберігати попередні версії файлів та завантажувати їх за потребою. Вона зберігає повну інформацію про версію кожного з файлів, а також повну структуру проекту на всіх стадіях розробки. Місце зберігання даних файлів називають репозиторієм. В середині кожного з репозиторіїв можуть бути створені паралельні лінії розробки — гілки.

Git підтримує швидке розділення та злиття версій, містить можливості для візуалізації і навігації за нелінійною історією розробки. 

