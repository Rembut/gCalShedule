\subsubsection{Маршрути}

\addCodeAsImg{\lstinputlisting[numbers=left]{code/AppMethod.js}}{Структура визначення маршрутів}{fig:Route}

В процесі роботи використовується протокол прикладного рівня HTTP. Обмін повідомленнями йде за схемою «запит-відповідь». Для ідентифікації ресурсів HTTP використовує URI. 

В додатку обробляються основні  HTTP методи для взаємодії об’єктами (рис.~\ref{fig:ApiAccess}). 

Протокол HTTP не зберігає свого стану між парами «запит-відповідь». Компоненти, що використовують HTTP, можуть самостійно здійснювати збереження інформації про стан, пов'язаний з останніми запитами та відповідями. 

Одним з розповсюджених способів реалізації цього можна назвати так звані cookies — невеликі записи, що зберігаються браузером. Зазвичай, вони встановлюються при виконанні користувачем певних дій та надсилаються серверу разом з наступними запитами. 

На рис.~\ref{fig:AppSignUp} зображено процес створення адміністратором нового користувача системи.

\addCodeAsImg{\input{uml/AppSignUp}}{Процес створення адміністратором нового користувача системи}{fig:AppSignUp}

Після отримання зазначеного POST запиту, сервер перевірить, чи має користувач відповідні права (блок Auth рис.~\ref{fig:CreateOperation}), створить об’єкт користувача з відповідними правами та збереже його в базі даних.

Створений користувач може входити до системи (рис.~\ref{fig:AppSignIn}) для виконання певних задач з користування системою.

\addCodeAsImg{% Auth sequence uml diagram

\documentclass[a4paper,10pt]{article}
\usepackage[english]{babel}
\usepackage[left=3cm, right=1cm, top=1cm, bottom=1cm]{geometry}

\usepackage{tikz-uml}

\sloppy
\hyphenpenalty 10000000

\begin{document}
\thispagestyle{empty}
\begin{center}
\begin{tikzpicture}
\begin{umlseqdiag}
	\umlactor[no ddots, x=1]{User}
	\umlboundary[no ddots, x=5]{App}
	\umldatabase[no ddots, x=14, fill=blue!20]{DB}
	
	\begin{umlcall}[op=SignIn request, type=synchron, return=Response, padding=3]{User}{App}
	
		\begin{umlfragment}[type=SignIn]
			\umlcreatecall[no ddots, x=8]{App}{DbUser}
				\begin{umlcall}[op=Init, type=synchron, return=Response]{App}{DbUser}
					\begin{umlcall}[op=Find one, type=synchron, return=User]{DbUser}{DB}\end{umlcall}	
					
					\umlcreatecall[no ddots, x=11]{DbUser}{JWT}
					\begin{umlcall}[op=Check password, type=synchron, return=Response]{DbUser}{JWT}\end{umlcall}
					
					\begin{umlfragment}[type=Validate, label=OK, fill=green!10]
						\begin{umlcall}[op=Create token, type=synchron, return=Token]{DbUser}{JWT}\end{umlcall}		
						\begin{umlcall}[op=Success, type=synchron]{DbUser}{DbUser}\end{umlcall}
						\umlfpart[Error]				
						\begin{umlcall}[op=Error, type=synchron]{DbUser}{DbUser}\end{umlcall}
					\end{umlfragment}
					
				\end{umlcall}	
		\end{umlfragment}
		
	\end{umlcall}
	
\end{umlseqdiag}
\end{tikzpicture}
\end{center}

\end{document}
}{Вхід користувача до системи}{fig:AppSignIn}

Можна зазначити, що адміністратор теж є користувачем, проте з особливими правами (диференціація на рис.~\ref{fig:Route}).
