\subsubsection{Бібліотеки JS}

\paragraph{Express}

При розробці використано бібліотеку Express — гнучкий фреймворк для веб-застосунків, побудованих на Node.js, що надає широкий набір функціональності, полегшуючи створення надійних API.

Express забезпечує тонкий прошарок базової функціональності для веб-застосунків, що не спотворює звичну та зручну функціональність Node.js., при отриманні запиту він оброблюватиметься відповідно до визначення маршруту (рис.~\ref{fig:Route}), де app є екземпляром express, METHOD є методом HTTP-запиту, PATH є шляхом на сервері, HANDLER є функцією-обробником, що спрацьовує, коли даний маршрут затверджено як співпадаючий.

\paragraph{JSON Web Token}

Для забезпечення конфіденційності при обміні даними використовується JSON Web Token. Деталі роботи з ним розглянуто в розділі \ref{subsubsection:jwt}. Для роботи з JSON Web Token використовується бібліотека jsonwebtoken.

\paragraph{bcrypt.js}

Оптимізовано bcrypt в JavaScript з нульовими залежностями. Сумісний з C ++ bcrypt прив'язка на node.js і також працює в браузері.

Міркування безпеки
Крім включення солі для захисту від атак аеродинамічних таблиць, bcrypt є адаптивною функцією: з плином часу кількість ітерацій може бути збільшена, щоб зробити її більш повільною, тому вона залишається стійкою до атаки з використанням грубої сили навіть при збільшенні потужності обчислення. (подивитися)

Хоча bcrypt.js є сумісним з C ++ bcrypt прив'язка, він написаний на чистому JavaScript і, таким чином, повільніше (близько 30\%), ефективно зменшуючи кількість ітерацій, які можуть бути оброблені в рівний проміжок часу.

Максимальна вхідна довжина становить 72 байти (зауважимо, що символи UTF8 кодуються до 4 байт), а довжина згенерованих хешей - 60 символів.
