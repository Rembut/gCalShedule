\subsubsection{Бібліотеки JS}

\paragraph{bcrypt.js}

Optimized bcrypt in JavaScript with zero dependencies. Compatible to the C++ bcrypt binding on node.js and also working in the browser.

Security considerations
Besides incorporating a salt to protect against rainbow table attacks, bcrypt is an adaptive function: over time, the iteration count can be increased to make it slower, so it remains resistant to brute-force search attacks even with increasing computation power. (see)

While bcrypt.js is compatible to the C++ bcrypt binding, it is written in pure JavaScript and thus slower (about 30\%), effectively reducing the number of iterations that can be processed in an equal time span.

The maximum input length is 72 bytes (note that UTF8 encoded characters use up to 4 bytes) and the length of generated hashes is 60 characters.

\paragraph{Express}

При розробці використано бібліотеку Express — гнучкий фреймворк для веб-застосунків, побудованих на Node.js, що надає широкий набір функціональності, полегшуючи створення надійних API.

Express забезпечує тонкий прошарок базової функціональності для веб-застосунків, що не спотворює звичну та зручну функціональність Node.js., при отриманні запиту він оброблюватиметься відповідно до визначення маршруту (рис.~\ref{fig:Route}), де app є екземпляром express, METHOD є методом HTTP-запиту, PATH є шляхом на сервері, HANDLER є функцією-обробником, що спрацьовує, коли даний маршрут затверджено як співпадаючий.

\paragraph{JSON Web Token} \label{subsubsection:jwt}

Для забезпечення конфіденційності при обміні даними використовується JSON Web Token. Деталі роботи з ним розглянуто в розділі \ref{subsubsection:jwt}. Для роботи з JSON Web Token використовується бібліотека jsonwebtoken.

\paragraph{pg-hstore}

A node package for serializing and deserializing JSON data to hstore format

