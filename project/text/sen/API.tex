Прикладний програмний інтерфейс — набір визначень підпрограм, протоколів взаємодії та засобів для створення програмного забезпечення. Спрощено - це набір чітко визначених методів для взаємодії різних компонентів. API надає розробнику засоби для швидкої розробки програмного забезпечення. API може бути для веб-базованих систем, операційних систем, баз даних, апаратного забезпечення, програмних бібліотек тощо.
