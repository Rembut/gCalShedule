\subsubsection{JSON Web Token} \label{subsubsection:jwt}

\addCodeAsImg{\input{uml/ApiAccess}}{Доступ на виконання запитів до системи}{fig:ApiAccess}

Для забезпечення конфіденційності при обміні даними використовується JSON Web Token. Роути, що обробляють реєстраційні та авторизаційні запити, представлено на рис.~\ref{fig:ApiAccess}.

JSON Web Token це стандарт токена доступу на основі JSON, стандартизованого в RFC 7519. Використовується для верифікації тверджень. JSON Web Token складається з трьох частин: заголовка, вмісту і підпису.

В корисному навантаженні зберігається будь-яка інформація, яку потрібно перевірити. Кожен ключ в корисному навантаженні відомий як «заява». Як і заголовок, корисне навантаження кодується в base64. Після отримання заголовку і корисного навантаження, обчислюється підпис.
