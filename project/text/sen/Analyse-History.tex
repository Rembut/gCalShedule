Ідея планування робіт існує стільки, скільки існує людська цивілізація, адже ще в неоліті, з переходом до тваринництва і землеробства, постають задачі з контролем циклічних процесів, що і викликало у подальшому створення календаря і писемності для фіксування задач.

З розвитком та індустріалізацією суспільства класи задач, що вимагають бути покритими детальним плануванням, суттєво розширилися. Черговим етапом розвитку таких технологій стало виникнення електронно-обчислювальних машин і впровадження їх у використання в промисловості.

В подальшому використання таких систем виходить за межі корпоративних систем підприємств і все частіше ними починають користуватися люди для планування власного часу і вирішення особистих задач.
