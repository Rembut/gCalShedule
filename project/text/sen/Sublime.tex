\subsubsection{Sublime Text}

Sublime Text - прорієтарний текстовий редактор. Підтримує плагіни на мові програмування Python.

Розробник дає можливість безкоштовно і без обмежень ознайомитися з редактором, однак програма періодично буде повідомляти про необхідність придбання ліцензії.

Редактор містить різні візуальні теми, а також можливість завантаження додаткових тем.

Коли користувач набере код, Sublime Text, в залежності від використовуваного мови, буде пропонувати різні варіанти для завершення запису. Також редактор може автоматично додавати розділові знаки (<<\{>>, <<\}>>, <<;>>).

Sublime Text дозволяє збирати програми у готовий проект і запускати їх без необхідності використання зовнішньої командної строки. Користувач також може налаштувати свою систему компіляції і включити автоматичну збірку програм кожного разу при збереженні коду. Ця система схожа з відповідним плагіном для зйомки тексту від віддаленого LaTeX (розділ~\ref{subsub:latex}).

Використовується плагін LaTeXTools. Плагін LaTeXTools надає кілька функцій, які спрощують роботу з файлами LaTeX.

Команда ST збирає компіляцію джерела LaTeX у PDF за допомогою texify (Windows / MikTeX) або latexmk (OSX / MacTeX, Windows / TeXlive, Linux / TeXlive). Потім він розбирає файл журналу і перераховує помилки та попередження. Нарешті, він запускає програму перегляду PDF і, на підтримуваних переглядачах (Sumatra PDF на Windows, Skim на OSX і Evince на Linux за замовчуванням) переходить до поточної позиції курсора.

Додатково реалізована функція автозбереження, що допомагає користувачам не втратити пророблену роботу.
