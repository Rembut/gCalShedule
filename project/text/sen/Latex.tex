\subsubsection{LaTeX} \label{subsub:latex}

\LaTeX -- це високоякісна набірна система; він включає функції, призначені для виготовлення технічної та наукової документації. LaTeX є фактичним стандартом для комунікації та публікації наукових документів \cite{lamport1994latex}. LaTeX доступний як вільне програмне забезпечення.

\TeX -- це створена чудовим американським математиком і програмістом Дональдом Кнутом система для верстки текстів з формулами. Сам по собі TEX є спеціалізованою мовою програмування (Кнут не тільки придумав мову, а й написав для нього транслятор, причому таким чином, що він працює абсолютно однаково на самих різних комп'ютерах), на якому пишуться видавничі системи, що використовуються на практиці. Точніше кажучи, кожна видавнича система на базі TEXа є пакетом макросів (макропакет) цієї мови. LATEX -- це створена Леслі Лампортом видавнича система на базі TEXа~\cite{львовский2003latex}.

Всі видавничі системи на базі TEXа володіють перевагами, закладеними в самому TEXе. Для новачка їх можна описати однією фразою: надрукований текст виглядає «зовсім як у книзі». LATEX, як видавнича система, надає зручні і гнучкі засоби досягти цього книжкового якості. Зокрема, вказавши за допомогою простих засобів структуру тексту, автор може не вникати в деталі оформлення, причому ці деталі при необхідності неважко змінити (щоб, скажімо, змінити шрифт, яким друкуються заголовки, не треба нишпорити по всьому тексту, змінюючи все заголовки , а досить замінити одну сходинку в «стильовому файлі»). Такі речі, як нумерація розділів, посилання, зміст і т. П. Виходять майже що «самі собою». Величезним плюсом систем на базі TEXа є висока якість та гнучкість форматування абзаців і математичних формул (в останньому відношенні краще TEXа цю задачу не вирішує жодна програма).

TEX (і всі видавничі системи на його базі) невибагливий до використовуваної техніки. З іншого сторони, TEXовські файли мають високий ступінь переносимості: Ви можете підготувати LATEXовський вихідний текст на своєму IBM PC, переслати його до видавництва, і бути впевненими, що там Ваш текст буде правильно оброблений і на друку вийде в точності те ж, що вийшло у Вас при пробному друку на Вашому улюбленому матричному принтері (з тією єдиною різницею, що фотоскладальний автомат дасть текст більш високої якості). Завдяки цій обставині TEX став дуже популярний як мова міжнародного обміну статтями з математики та фізики.

Є у TEXа і недоліки. Перший з них (розділяється TEXом з усіма іншими видавничими системами) такий: він працює відносно повільно, займає багато пам'яті. Друга особливість TEXа, яка може не сподобатися тим, хто звик до традиційних редакторів офісних пакетів -- це те, що він не є системою типу WYSIWYG ("те що ти бачиш, те ти і отримаєш"): робота з вихідним текстом і перегляд того, як текст буде виглядати після друку -- різні операції. Втім, завдяки цій особливості час на підготовку тексту істотно скорочується.

При роботі над звітом також використано сервіс Overleaf -- сучасний інструмент, розроблений у 2012 році. Він був створений щоб допомогти редагувати свої наукові статті, технічні звіти, тези, презентації, блок-схеми та інші документи, написані на мові розмітки LaTeX. 
