\subsection{Клієнт-серверна архітектура веб-додатків}

Архітектура клієнт-сервер є одним із архітектурних шаблонів програмного забезпечення та є домінуючою концепцією у створенні розподілених мережних застосунків і передбачає взаємодію та обмін даними між ними. Вона передбачає такі основні компоненти:
\begin{itemize}
	\item набір серверів, які надають інформацію або інші послуги програмам, які звертаються до них;
	\item набір клієнтів, які використовують сервіси, що надаються серверами;
	\item мережа, яка забезпечує взаємодію між клієнтами та серверами.
\end{itemize}

Сервери є незалежними один від одного. Клієнти також функціонують паралельно і незалежно один від одного. Немає жорсткої прив'язки клієнтів до серверів. Більш ніж типовою є ситуація, коли один сервер одночасно обробляє запити від різних клієнтів; з іншого боку, клієнт може звертатися то до одного сервера, то до іншого. Клієнти мають знати про доступні сервери, але можуть не мати жодного уявлення про існування інших клієнтів.

Загальноприйнятим є положення, що клієнти та сервери — це перш за все програмні модулі. Найчастіше вони знаходяться на різних комп'ютерах, але бувають ситуації, коли обидві програми — і клієнтська, і серверна, фізично розміщуються на одній машині; в такій ситуації сервер часто називається локальним.
