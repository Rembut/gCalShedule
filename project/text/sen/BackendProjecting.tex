\subsection{Проектування та прототипування back-end частини}

Для обміну інформацією між користувацьким додатком та back-end частиною використовується протокол передачі гіпертекстових даних HTTP. Передачу даних забезпечує стек транспортних протоколів TCP/IP.
Одним з способів побудови мереживих HTTP-додатків є використання  асинхронного подієвого JavaScript–оточення Node. Для кожного з’єднання викликається функція зворотнього виклику, проте коли з’єднань немає Node засинає.

У Node не має функцій, що працюють напряму з I/O, тому процес не блокується ніколи. Як результат, на Node легко розробляти масштабовані системи \cite{zeiss2015node}.
Node широко використовує подієву модель, він приймає цикл подій за основу оточення, замість того, щоб використовувати його в якості бібліотеки. В інших системах відбувається блокування виклику для запуску циклу подій.

При розробці використано бібліотеку Express — гнучкий фреймворк для веб-застосунків, побудованих на Node.js, що надає широкий набір функціональності, полегшуючи створення надійних API.

Express забезпечує тонкий прошарок базової функціональності для веб-застосунків, що не спотворює звичну та зручну функціональність Node.js., при отриманні запиту він оброблюватиметься відповідно до визначення маршруту (рис.~\ref{fig:Route}), де app є екземпляром express, METHOD є методом HTTP-запиту, PATH є шляхом на сервері, HANDLER є функцією-обробником, що спрацьовує, коли даний маршрут затверджено як співпадаючий.
