% За основу взят github.com/Amet13/bachelor-diploma

\documentclass[a4paper,14pt]{extarticle} % 14й шрифт
%%% Преамбула %%%

\usepackage{fontspec} % XeTeX
\usepackage{xunicode} % Unicode для XeTeX
\usepackage{xltxtra}  % Верхние и нижние индексы
\usepackage{pdfpages} % Вставка PDF

\usepackage{listings} % Оформление исходного кода
\lstset{
    basicstyle=\small\ttfamily, % Размер и тип шрифта
    breaklines=true, % Перенос строк
    tabsize=2, % Размер табуляции
    literate={--}{{-{}-}}2 % Корректно отображать двойной дефис
}

% Шрифты, xelatex
\defaultfontfeatures{Ligatures=TeX}
\setmainfont{Times New Roman} % Нормоконтроллеры хотят именно его
\newfontfamily\cyrillicfont{Times New Roman}
%\setsansfont{Liberation Sans} % Тут я его не использую, но если пригодится
\setmonofont{FreeMono} % Моноширинный шрифт для оформления кода

% Украинский язык
\usepackage{polyglossia}
\setdefaultlanguage{ukrainian}

\usepackage[autostyle]{csquotes} % Стиль кавычек
\DeclareQuoteAlias{russian}{ukrainian} % csquotes не знает об украинском языке. Примем, что кавычки аналогичные русскому

\usepackage[backend=biber, babel=hyphen, bibstyle=gost-standard, style=numeric-comp]{biblatex} % Библиография

\usepackage{amssymb,amsfonts,amsmath} % Математика
\usepackage{tikz-uml} %  UML диаграммы
\numberwithin{equation}{section} % Формула вида секция.номер

\usepackage{enumerate} % Тонкая настройка списков
\usepackage{indentfirst} % Красная строка после заголовка
\usepackage{float} % Расширенное управление плавающими объектами
\usepackage{multirow} % Сложные таблицы

% Пути к каталогам с изображениями
\usepackage{graphicx} % Вставка картинок и дополнений
\graphicspath{{img/}}

% Формат подрисуночных записей
\usepackage{chngcntr}
\counterwithin{figure}{section}

% Гиперссылки
\usepackage{hyperref}
\hypersetup{
    colorlinks, urlcolor={black}, % Все ссылки черного цвета, кликабельные
    linkcolor={black}, citecolor={black}, filecolor={black},
    pdfauthor={Имя Фамилия},
    pdftitle={Название}
}

% Оформление библиографии и подрисуночных записей через точку
\makeatletter
\renewcommand*{\@biblabel}[1]{\hfill#1.}
\renewcommand*\l@section{\@dottedtocline{1}{1em}{1em}}
\renewcommand{\thefigure}{\thesection.\arabic{figure}} % Формат рисунка секция.номер
\renewcommand{\thetable}{\thesection.\arabic{table}} % Формат таблицы секция.номер
\def\redeflsection{\def\l@section{\@dottedtocline{1}{0em}{10em}}}
\makeatother

\renewcommand{\baselinestretch}{1.4} % Полуторный межстрочный интервал
\parindent 1.27cm % Абзацный отступ

\sloppy             % Избавляемся от переполнений
\hyphenpenalty=1000 % Частота переносов
\clubpenalty=10000  % Запрещаем разрыв страницы после первой строки абзаца
\widowpenalty=10000 % Запрещаем разрыв страницы после последней строки абзаца

% Отступы у страниц
\usepackage{geometry}
\geometry{left=3cm}
\geometry{right=1cm}
\geometry{top=2cm}
\geometry{bottom=2cm}

% Списки
\usepackage{enumitem}
\setlist[enumerate,itemize]{leftmargin=12.7mm} % Отступы в списках

\makeatletter
    \AddEnumerateCounter{\asbuk}{\@asbuk}{м)}
\makeatother
\setlist{nolistsep} % Нет отступов между пунктами списка
\renewcommand{\labelitemi}{--} % Маркет списка --
\renewcommand{\labelenumi}{\asbuk{enumi})} % Список второго уровня
\renewcommand{\labelenumii}{\arabic{enumii})} % Список третьего уровня

% Содержание
\usepackage{tocloft}
\renewcommand{\cfttoctitlefont}{\hspace{0.38\textwidth}\MakeTextUppercase} % СОДЕРЖАНИЕ
\renewcommand{\cftsecfont}{\hspace{0pt}}            % Имена секций в содержании не жирным шрифтом
\renewcommand\cftsecleader{\cftdotfill{\cftdotsep}} % Точки для секций в содержании
\renewcommand\cftsecpagefont{\mdseries}             % Номера страниц не жирные
\setcounter{tocdepth}{2}                            % Глубина оглавления, до subsection

% Нумерация страниц справа сверху
\usepackage{fancyhdr}
\pagestyle{fancy}
\fancyhf{}
\fancyhead[R]{\textrm{\thepage}}
\fancyheadoffset{0mm}
\fancyfootoffset{0mm}
\setlength{\headheight}{17pt}
\renewcommand{\headrulewidth}{0pt}
\renewcommand{\footrulewidth}{0pt}
\fancypagestyle{plain}{ 
    \fancyhf{}
    \rhead{\thepage}
}

% Формат подрисуночных надписей
\RequirePackage{caption}
\DeclareCaptionLabelSeparator{defffis}{ -- } % Разделитель
\captionsetup[figure]{justification=centering, labelsep=defffis, format=plain} % Подпись рисунка по центру
\captionsetup[table]{justification=raggedright, labelsep=defffis, format=plain, singlelinecheck=false} % Подпись таблицы слева
\addto\captionsrussian{\renewcommand{\figurename}{Рис.}} % Имя фигуры

% Пользовательские функции
\usepackage{flafter} % рисунко появится не раньше первой ссылки на него

\newcommand{\addimg}[4]{ % Добавление одного рисунка
    \begin{figure}
        \centering
        \includegraphics[width=#2\linewidth]{#1}
        \caption{#3} \label{#4}
    \end{figure}
}
\newcommand{\addCodeAsImg}[3]{ % Добавление tikz рисунка
    \begin{figure}[htb]
        \centering
        {#1}
        \caption{#2} \label{#3}
    \end{figure}
}
\newcommand{\addimghere}[4]{ % Добавить рисунок непосредственно в это место
    \begin{figure}[H]
        \centering
        \includegraphics[width=#2\linewidth]{#1}
        \caption{#3} \label{#4}
    \end{figure}
}
\newcommand{\addtwoimghere}[5]{ % Вставка двух рисунков
    \begin{figure}[H]
        \centering
        \includegraphics[width=#2\linewidth]{#1}
        \hfill
        \includegraphics[width=#3\linewidth]{#2}
        \caption{#4} \label{#5}
    \end{figure}
}
\newcommand{\addimgapp}[2]{ % Это костыль для приложения Б
    \begin{figure}[H]
        \centering
        \includegraphics[width=1\linewidth]{#1}
        \caption*{#2}
    \end{figure}
}

% Заголовки секций в оглавлении в верхнем регистре
\usepackage{textcase}
\makeatletter
\let\oldcontentsline\contentsline
\def\contentsline#1#2{
    \expandafter\ifx\csname l@#1\endcsname\l@section
        \expandafter\@firstoftwo
    \else
        \expandafter\@secondoftwo
    \fi
    {\oldcontentsline{#1}{\MakeTextUppercase{#2}}}
    {\oldcontentsline{#1}{#2}}
}
\makeatother

% Оформление заголовков
\usepackage[compact,explicit]{titlesec}
\titleformat{\section}{}{}{12.5mm}{\centering{\thesection\quad\MakeTextUppercase{#1}}}
\titleformat{\subsection}[block]{\vspace{1em}}{}{12.5mm}{\thesubsection\quad#1}
\titleformat{\subsubsection}[block]{\vspace{1em}\normalsize}{}{12.5mm}{\thesubsubsection\quad#1}
\titleformat{\paragraph}[block]{\normalsize}{}{12.5mm}{\MakeTextUppercase{#1}}

\pdfstringdefDisableCommands{\let\uppercase\relax} % \uppercase не поддерживается в закладках pdf

% Секции без номеров (введение, заключение...), вместо section*{}
\newcommand{\anonsection}[1]{
    \phantomsection % Корректный переход по ссылкам в содержании
    \paragraph{\centerline{{#1}}}
    \addcontentsline{toc}{section}{\uppercase{#1}}
}

% Секции для приложений
\newcommand{\appsection}[1]{
    \phantomsection
    \paragraph{\centerline{{#1}}}
    \addcontentsline{toc}{section}{\uppercase{#1}}
}

% Библиография: отступы и межстрочный интервал
\makeatletter
\renewenvironment{thebibliography}[1]
    {\section*{\refname}
        \list{\@biblabel{\@arabic\c@enumiv}}
           {\settowidth\labelwidth{\@biblabel{#1}}
            \leftmargin\labelsep
            \itemindent 16.7mm
            \@openbib@code
            \usecounter{enumiv}
            \let\p@enumiv\@empty
            \renewcommand\theenumiv{\@arabic\c@enumiv}
        }
        \setlength{\itemsep}{0pt}
    }
\makeatother

\newenvironment{umlstyle}
{\hyphenpenalty 10000000
\footnotesize
\tikzumlset{font=\footnotesize}
\renewcommand{\baselinestretch}{1}
\begin{center}
\begin{tikzpicture}}
{\end{tikzpicture}
\end{center}}

\setcounter{page}{4} % Начало нумерации страниц
 % Подключаем преамбулу

\hypersetup{
    colorlinks, urlcolor={black}, % Все ссылки черного цвета, кликабельные
    linkcolor={black}, citecolor={black}, filecolor={black},
    pdfauthor={Сенчишен Денис Олександрович},
    pdftitle={Проектування та розробка мікросервісів для веб-додатку редагування розкладу}
}

\addbibresource{bibliography.bib} % Библиографический справочник


%%% Начало документа
\begin{document}
\thispagestyle{empty}

{\centering
МІНІСТЕРСТВО ОСВІТИ І НАУКИ УКРАЇНИ

ХЕРСОНСЬКИЙ ДЕРЖАВНИЙ УНІВЕРСИТЕТ

ФАКУЛЬТЕТ КОМП'ЮТЕРНИХ НАУК, ФІЗИКИ ТА ІНФОРМАТИКИ

КАФЕДРА ІНФОРМАТИКИ, ПРОГРАМНОЇ ІНЖЕНЕРІЇ ТА 

ЕКОНОМІЧНОЇ КІБЕРНЕТИКИ

\vfill

ПРОЕКТУВАННЯ ТА РОЗРОБКА UI ВЕБ-ДОДАТКУ РЕДАГУВАННЯ РОЗКЛАДУ

Дипломна робота

на здобуття ступеня вищої освіти бакалавр

}

\vfill

\hfill\begin{minipage}[t]{0.6\textwidth}
Виконав: 

студент 4 курсу 431 групи \\ спеціальності 6.040302  Інформатика

Воробйов Євгеній Андрійович

Керівник:

кандидат педагогічних наук, доцент

Круглик Владислав Сергійович

\end{minipage}

\vfill

{\centering
Херсон --- 2019

} % Титульная страница

\tableofcontents % Содержание 
\clearpage

\anonsection{ВСТУП}

Якість підготовки спеціалістів у закладах освіти і особливо ефективність використання науково-педагогічного потенціалу залежать певною мірою від рівня організації навчального процесу.

Одна з головних складових цього процесу -- розклад занять -- регламентує трудовий ритм, впливає на творчу віддачу викладачів, тому його можна вважати фактором оптимізації використання обмежених ресурсів -- викладацького складу і аудиторного фонду.

Проблему складання розкладу слід розглядати не тільки як трудомісткий процес, об'єкт автоматизації з використанням комп’ютера, але і як проблему оптимального керування. 

Оскільки всі фактори, що впливають на розклад, практично неможливо врахувати, а інтереси учасників навчального процесу різноманітні, задача складання розкладу є багатокритеріальною з нечіткою множиною факторів.

Незалежно від алгоритму побудови розкладу, виникає прикладна проблема з інструментів різних рівнів, що використовуються в процесі. Саме ним і буде присвячено проведену роботу.

\textbf{Актуальність дослідження} полягає в необхідності забезпечення всіх учасників освітнього процесу доступом до актуальної версії розкладу занять у будь-який час, а також можливості спрощення процесу формування розкладу та подальшої інформатизації освітнього процесу.

\textbf{Об’єкт дослідження}~--- системи для планування та підтримки планування розкладу. \textbf{Предмет дослідження}~--- система для підтримки планування розкладу в закладах освіти з поділом учнів (вихованців, здобувачів освіти тощо) на стабільні академічні групи.

\textbf{Метою роботи} є проектування та розробка розширюваної системи підтримки редагування розкладу в закладах освіти з можливістю використання всіма учасниками освітнього процесу та реалізація відкритого API для взаємодії з системою.

Для реалізації мети поставлено наступні \textbf{завдання роботи}:
\begin{enumerate}
	\item Проаналізувати характеристики існуючих систем планування, зокрема обсяг їх можливостей.
	\item Проаналізувати окремі частини процесу підготовки розкладу на прикладі факультету комп'ютерних наук, фізики та математики ХДУ.
	\item На основі проведеного аналізу розробити вимоги щодо можливостей системи.
	\item Відповідно до створених вимог розробити серверну частину, зокрема реалізувати структуру бази даних та API.
	\item Розробити документацію до публічного API.
	\item Обґрунтувати використані технології при проектуванні серверної частини.
\end{enumerate}

Очікується, що спроектований продукт буде придатний до використання всіма учасниками освітнього процесу в ЗВО.

Робота складається з 2 розділів, містить \totalfigures\ рисунків. 
 % Введение

\section{АНАЛІЗ СИСТЕМ ПЛАНУВАННЯ ТА ІНФОРМУВАННЯ}
\subsection{Задачі систем планування} \subsection{Задачі систем планування}

Ідея планування робіт існує стільки, скільки існує людська цивілізація, адже ще в неоліті, з переходом до тваринництва і землеробства, постають задачі з контролем циклічних процесів, що і викликало у подальшому створення календаря і писемності для фіксування задач.


\subsection{Порівняння сучасних сервісів планування та інформування} На сьогодні, кожна людина так чи інакше стикається у повсякденному житті з системами, пов'язаними з контролем часу та завдань. Крім цього, такі технології знаходять застосування в освітньому процесі~\cite{ліщина2014проблеми}.

\subsubsection{Microsoft Outlook}

Microsoft Outlook — додаток-органайзер, входить до пакету офісних програм Microsoft Office. Дозволяє працювати з електронною поштою, надає функції календаря, планувальника завдань, записника і менеджера контактів. Крім того, Outlook дозволяє відстежувати роботу з документами пакету Microsoft Office для автоматичного складання щоденника роботи~\cite{франчук2016використання}.

Outlook може використовуватися  і як окремий додаток, так і виступати в ролі клієнта для Microsoft Exchange Server, що надає додаткові функції для спільної роботи всіх користувачів організації: загальні поштові скриньки, папки завдань, календарі, планування часу загальних зустрічей, узгодження документів тощо.

Крім цього, дозволяє підключати через протоколи POP3/IMAP інші поштові сервіси та додатки, що надаються ними. Зокрема, нижче буде розглянуто синхронізацію MS Outlook з сервісами Google.

\subsubsection{Lightning}

Lightning — проект Mozilla Foundation, що додає функції календаря і планувальника в Mozilla Thunderbird — безкоштовну кросплатформну програму для роботи з електронною поштою і новинами, що може вважатися відкритим аналогом для відповідних продуктів з пакету Microsoft Office.


\subsubsection{Google Calendar}

Google Calendar — безкоштовний веб-додаток для тайм-менеджменту розроблений Google. Інтерфейс подібний до аналогічних календарних додатків, таких як Microsoft Outlook. Має різні режими перегляду, зокрема денний, тижневий та місячний. Події зберігаються онлайн, а тому календар можна переглядати з будь-якого пристрою, обладнаного доступом до мережі Інтернет. Додаток може імпортувати та експортувати файли календаря різних форматів, а для існуючих — задавати різні права доступу~\cite{олексюк2013деякі}. 

Слід зазначити, що Google Calendar, як і інші сервіси Google, має відкрите API, що дозволяє взаємодіяти з ним через власні додатки після відповідних налаштувань.

Окремо слід звернути увагу на розвинені технології вбудовування документів Google (зокрема календарів Google Calendar) у власні веб додатки. 

Одним з прикладів такого використання в контексті розвитку інформаційної інфраструктури університету можна навести інтеграцію календаря подій факультету комп'ютерних наук, фізики та математики ХДУ в відповідну сторінку (kspu.edu/About/Faculty/FPhysMathemInformatics.aspx) на офіційному веб-сайті (рис.~\ref{fig:CalendarKspuEdu}).

\addimg{CalendarKspuEdu.png}{1}{Календар подій факультету}{fig:CalendarKspuEdu}

В наведеному прикладі події різних календарів, об'єднаних для відображення відображаються різними кольорами, в назву події включено час початку, а при натисканні на неї - відображаються деталі, зокрема опис, місце та посилання на подію в Google Calendar, де, крім іншого, можливо додати її для відслідковування та нагадування у власний календар, за умови, якщо користувач попередньо авторизувався в свій акаунт.

Документи, для яких встановлені публічні права для перегляду, можна включати в вихідний код сайту у вигляді фрейму. Фрейм — окремий HTML-документ, який сам чи разом з іншими документами відображений у вікні веб-переглядача. При цьому, всю відповідальність за відображуване в фреймі несе сервіс-власник, тобто Google Calendar в наведеному прикладі, а в місці відображення знаходиться лише код інтеграції з посиланням та супутніми параметрами~\cite{ліщина2014проблеми}.

Інший приклад використання мікросервісів (детальніше про мікросервіси в розділі~\ref{subsec:microservices}) та фреймів у контексті розвитку інформаційної інфраструктури  -- сервіс замовлення довідки про навчання в університеті (працює для студентів факультету комп'ютерних наук, фізики та математики, рис.~\ref{fig:RequestCertificateFromDeansOffice}).

\addimg{RequestCertificateFromDeansOffice.png}{0.75}{Форма замовлення довідки}{fig:RequestCertificateFromDeansOffice}

Після заповнення відповідної форми (не є справжньою формою на основі тегу <form> в розумінні HTML у зв'язку з обмеженнями сайту, проте реалізована з використанням його компонентів; поведінку цілком відтворено з допомогою javascript) відбувається запит до сервісу EmailJS.com, котрий, у свою чергу, надсилає лист за шаблоном на основі запиту з сайту. Сервіс має певні обмеження щодо кількості листів в проміжок часу, проце цілком задовільняє поставлені вимоги. Після отримання відповідного листа на корпоративну робочу пошту, відповідальний за підготовку довідок працівник деканату формує її в інформаційно-аналітичній системі університету  (детальніше в розділі~\ref{subsubs:KIS}) та виконує інші необхідні операції. В результаті зменшується навантаження на працівника та виключається необхідність  для студента у попередньому зверненні для запису; зменшується кількість <<паперових>> процесів що сприятливо впливає на подальшій інформатизції освітнього процесу.

При генерації коду фрейму для інтеграції у адміністратора є можливість налаштувати колірне оформлення фрейму, його розміри, регіональні стандарти (мову відображення, день початку тиждня, часовий пояс), обсяг за замовчуванням (тиждень, місяць), додати або приховати елементи керування. В процесі редагування налаштувань отримується невеликий за обсягом код (HTML тег <iframe>, рис.~\ref{fig:CalendarIframe}) для розміщення в коді власної веб-сторінки. 

\addCodeAsImg{\lstinputlisting[numbers=left]{code/CalendarIframe.tex}}{Код інтеграції календаря факультету}{fig:CalendarIframe}

Аналогічним чином інтегруються інші сервіси Google, що вже знайшло використання при розміщенні матеріалів на сайті, як то презентації, текстові документи, таблиці, карти, що одночасно підтверджує, по-перше, перспективність використання хмарних сервісів для поступового осучаснення інформаційної інфраструктури та, по друге, можливість переходу до використання їх замість звичних офісних пакетів (Microsoft Office, Open Office, Libre Office тощо).



\subsubsection{Microsoft Outlook}

Microsoft Outlook — додаток-органайзер, входить в пакет офісних програм Microsoft Office. Дозволяє працювати з електронною поштою, надає функції календаря, планувальника завдань, записника і менеджера контактів. Крім того, Outlook дозволяє відстежувати роботу з документами пакету Microsoft Office для автоматичного складання щоденника роботи.

Outlook може використовуватися  і як окремий додаток, так і виступати в ролі клієнта для Microsoft Exchange Server, що надає додаткові функції для спільної роботи всіх користувачів організації: загальні поштові скриньки, папки завдань, календарі, планування часу загальних зустрічей, узгодження документів тощо.

Крім цього, дозволяє підключати через протоколи POP3/IMAP інші поштові сервіси та додатки, що надаються ними. Зокрема, нижче буде розглянуто синхронізацію MS Outlook з сервісами Google.

\subsubsection{Lightning}

Lightning — проект Mozilla Foundation, що додає функції календаря і планувальника в Mozilla Thunderbird — безкоштовну кросплатформну програму для роботи з електронною поштою і новинами, що може вважатися відкритим аналогом для відповідних продуктів з пакету Microsoft Office.

\subsubsection{Google Calendar}

Google Calendar — безкоштовний веб-додаток для тайм-менеджменту розроблений Google. Інтерфейс подібний до аналогічних календарних додатків, таких як Microsoft Outlook. Має різні режими перегляду, зокрема денний, тижневий та місячний. Події зберігаються онлайн, а тому календар можна переглядати з будь-якого пристрою, обладнаного доступом до мережі Інтернет. Додаток може імпортувати та експортувати файли календаря різних форматів, а для існуючих — задавати різні права доступу. 

Слід зазначити, що Google Calendar, як і інші сервіси Google, має відкрите API, що дозволяє взаємодіяти з ним через власні додатки після відповідних налаштувань.

Окремо слід звернути увагу на розвинені технології вбудовування документів Google (зокрема календарів Google Calendar) в власні веб додатки. 

Одним з прикладів такого використання в контексті розвитку інформаційної інфраструктури університету можна навести інтеграцію календаря подій факультету комп'ютерних наук, фізики та математики ХДУ в відповідну сторінку (kspu.edu/About/Faculty/FPhysMathemInformatics.aspx) на офіційному веб-сайті (рис.~\ref{fig:CalendarKspuEdu}).

\addimg{CalendarKspuEdu.png}{1}{Календар подій факультету}{fig:CalendarKspuEdu}

В наведеному прикладі події різних календарів, об'єднаних для відображення відображаються різними кольорами, в назву події включено час початку, а при натисканні на неї - відображаються деталі, зокрема опис, місце та посилання на подію в Google Calendar, де, крім іншого, можливо додати її для відслідковування та нагадування у власний календар, за умови, якщо користувач попередньо аворизувався в свій акаунт.

Документи, для яких встановлені публічні права для перегляду, можна включати в вихідний код сайту у вигляді фрейму. Фрейм — окремий HTML-документ, який сам чи разом з іншими документами відображений у вікні веб-переглядача. При цьому, всю відповідальність за відображуване в фреймі несе сервіс-власник, тобто Google Calendar в наведеному прикладі, а в місці відображення знаходиться лише код інтеграції з посиланням та супутніми параметрами.

Інший приклад використання мікросервісів (детальніше про мікросервіси в розділі~\ref{subsec:microservices}) та фреймів у контексті розвитку інформаційної інфраструктури  -- сервіс замовлення довідки про навчання в університеті (працює для студентів факультету комп'ютерних наук, фізики та математики, рис.~\ref{fig:RequestCertificateFromDeansOffice}).

\addimg{RequestCertificateFromDeansOffice.png}{0.75}{Форма замовлення довідки}{fig:RequestCertificateFromDeansOffice}

Після заповнення відповідної форми (не є справжньою формою на основі тегу <form> в розумінні HTML у зв'язку з обмеженнями сайту, проте реалізована з використанням його компонентів; поведінку цілком відтворено з допомогою javascript) відбувається запит до сервісу EmailJS.com, котрий, у свою чергу, надсилає лист за шаблоном на основі запиту з сайту. Сервіс має певні обмеження щодо кількості листів в проміжок часу, проце цілком задовільняє поставлені вимоги. Після отримання відповідного листа на корпоративну робочу пошту, відповідальний за підготовку довідок працівник деканату формує її в інформаційно-аналітичній системі університету  (детальніше в розділі~\ref{subsubs:KIS}) та виконує інші необхідні операції. В результаті зменшується навантаження на працівника та виключається необхідність  для студента у попередньому зверненні для запису; зменшується кількість <<паперових>> процесів що сприятливо впливає на подальшій інформатизції освітнього процесу.

При генерації коду фрейму для інтеграції у адміністратора є можливість налаштувати колірне оформлення фрейму, його розміри, регіональні стандарти (мову відображення, день початку тиждня, часовий пояс), обсяг за замовчуванням (тиждень, місяць), додати або приховати елементи керування. В процесі редагування налаштувань отримується невеликий за обсягом код (HTML тег <iframe>, рис.~\ref{fig:CalendarIframe}) для розміщення в коді власної веб-сторінки. 

\addCodeAsImg{\lstinputlisting[numbers=left]{code/CalendarIframe.html}}{Код інтеграції календаря факультету}{fig:CalendarIframe}

Аналогічним чином інтегруються інші сервіси Google, що вже знайшло використання при розміщенні матеріалів на сайті, як то презентації, текстові документи, таблиці, карти, що одночасно підтверджує, по перше, перспективність використання хмарних сервісів для поступового осучаснення інформаційної інфраструктури та, по друге, можливість переходу до використання їх замість звичних офісних пакетів (Microsoft Office, Open Office, Libre Office тощо).



\subsection{Корпоративні інформаційні системи} \label{subsubs:KIS}

Корпоративна інформаційна система — це інформаційна система, яка підтримує автоматизацію функцій управління на підприємстві і постачає інформацію для прийняття управлінських рішень. У ній реалізована управлінська ідеологія, яка об'єднує бізнес-стратегію підприємства і прогресивні інформаційні технології~\cite{hansvanderhoeven2011}.

У загальному визначенні «автоматизована система» — сукупність керованого об'єкта й автоматичних керувальних пристроїв, у якій частину функцій керування виконує людина. Вона представляє собою організаційно-технічну систему, що забезпечує вироблення рішень на основі автоматизації інформаційних процесів у різних сферах діяльності. 

Сучасні автоматизовані системи управління навчальним процесом у  закладах вищої освіти здатні вирішувати велику кількість функцій~\cite{співаковський2014побудова}, а саме:
\begin{itemize}
	\item планування, контроль та аналіз навчальної діяльності;
	\item оперативний доступ до інформації про навчальний процес;
	\item єдину систему звітів, як внутрішніх, так і за вимогами МОН України;
	\item системи безпеки даних з урахуванням вимог законодавства;
	\item облік контингенту студентів та співробітників;
	\item проведення вступної кампанії;
	\item формування пакетів даних з метою виготовлення тих чи інших документів.
\end{itemize}

Функціонування будь-якої автоматизованої системи можна швидко адаптувати до особливостей навчального процесу конкретного навчального закладу, до локальних мереж різного рівня, що допомагає розширити коло користувачів (адміністрації, викладачів і студентів) для оперативного забезпечення їх необхідною інформацією. 

Отже, використання таких систем дає змогу не тільки удосконалити якість планування навчального процесу, а й оперативність управління ним.

Не зважаючи на всі переваги, які надає використання автоматизованих систем, досі далеко не в кожному закладі вони впроваджені чи використовуються в повній мірі з тих чи інших причин — інерційності поглядів адміністрації, супротив працівників або «саботаж» на місцях, відсутність фінансової або організаційної можливості.

\subsubsection{Інформаційно-аналітична система} \label{subs:ias}

В ХДУ використовується корпоративна інтегрована система «Інформаційно-аналітична система (IAS)». Вона дозволяє вести облік працівників і студентів, бухгалтерський облік, контроль за матеріальними цінностями тощо (рис.~\ref{fig:IasSubsustem}). 

\addimg{IasSubsustem.png}{0.7}{Структура ІАС}{fig:IasSubsustem}
		
Система дозволяє вносити і ефективно стежити за будь-якими змінами. В основі системи лежить ядро, на основі ядра виконується розширення системи до будь-якої кількості компонентів. При цьому основна функціональність може бути розширена за рахунок додаткових компонентів~\cite{львов2007інформаційна}. 

Програма IAS орієнтована на платформу Windows з використанням MS SQL Server. Вона має багаторівневу архітектуру, що складається з бази даних, бізнес-логіки та клієнтського інтерфейсу. Внутрішній журнал реєстрації подій дозволяє вести та слідкувати за записами, що стосуються усіх подій.

Відсутність компонентів, пов’язаних з формуванням розкладу занять, та відсутність у використанні сторонніх рішень ставить задачу з проектування власного додатку для забезпечення всіх учасників освітнього процесу доступом до актуальної версії розкладу занять у будь-який час, а також можливості спрощення процесу формування розкладу та подальшої інформатизації освітнього процесу.
\subsubsection{Інформаційно-аналітична система} \label{subs:ias}

В ХДУ використовується корпоративна інтегрована система «Інформаційно-аналітична система (IAS)». Вона дозволяє вести облік працівників і студентів, бухгалтерський облік, контроль за матеріальними цінностями тощо (рис.~\ref{fig:IasSubsustem}). 

\addimg{IasSubsustem.png}{0.7}{Структура ІАС}{fig:IasSubsustem}
		
Система дозволяє вносити і ефективно стежити за будь-якими змінами. В основі системи лежить ядро, на основі ядра виконується розширення системи до будь-якої кількості компонентів. При цьому основна функціональність може бути розширена за рахунок додаткових компонентів. 

Програма IAS орієнтована на платформу Windows з використанням MS SQL Server. Вона має багаторівневу архітектуру, що складається з бази даних, бізнес-логіки та клієнтського інтерфейсу. Внутрішній журнал реєстрації подій дозволяє вести та слідкувати за записами, що стосуються усіх подій.

Відсутність компонентів, пов’язаних з формуванням розкладу занять, та відсутність у використанні сторонніх рішень ставить задачу з проектування власного додатку для забезпечення всіх учасників освітнього процесу доступом до актуальної версії розкладу занять у будь-який час, а також можливості спрощення процесу формування розкладу та подальшої інформатизації освітнього процесу.
\subsubsection{Єдине інформаційно-освітнє серидовище ХДУ}

Разом з осучасненням освітнього процесу у цілому актуальною є задача з об'єднання існуючих інформаційних систем університету.

В результаті проведеної інтеграції планується отримати так званий <<Особистий кабінет студента>>, в котрому студент, як основний учасник освітнього процесу, матиме доступ до всієї необхідної інформації. Наразі інформаційне середовище включає в себе низку в цілому незалежних один від одного проектів, а саме:

\begin{itemize}
	\item web-портал університету \cite{KspuEdu};
	\item система дистанційного навчання <<KSU Online>> \cite{KsuOnline};
	\item система дистанційної освіти <<Херсонський Віртуальний Університет>> \cite{KsuDis}.
	\item програмний комплекс <<ST-Абітурієнт>>, що використовується для підтримки процесу прийому документів абітурієнтів та обробки заяв про зарахування, результатів вступних іспитів тощо;
	\item програмний комплекс <<Інформаційно-аналітична система (ІАС)>>, що дозволяє вести облік співробітників і студентів, бухгалтерський облік, контроль за матеріальними цінностями (розділ~\ref{subs:ias}), проте лише частково охоплює навчальний процес;
	\item сервіс <<KSU Feedback>> призначений для проведення анонімного або звичайного голосування за визначеними критеріями серед строго визначеного кола респондентів \cite{KsuFeedback};
	\item сервіс  <<Пошук книг в електронному каталозі бібліотеки>> надає доступ до каталогу в будь який момент \cite{eLibrary};
	\item web-портал <<Збірник наукових праць <<Інформаційні технології в освіті>> (ІТО)>> є каналом поширення та передачі знань, де вчені, практики та дослідники можуть обговорювати, аналізувати, критикувати, синтезувати, спілкуватися та підтримувати розробку та впровадження ІТ і пов'язаних з ними наслідків у всіх аспектах їх використання у сфері освіти \cite{ITO};
	\item web-портал <<Чорноморський ботанiчний журнал>>, на котрому у відкритому доступі знаходяться електронні версії всіх статей у форматі pdf, опублікованих у журналі з 2005 року.
\end{itemize}

При цьому наразі відсутня будь-яка інтеграція між сервісами та сайтами, крім посилань на певну частину з них на головній сторінці web-порталу університету.


\section{ПРОЕКТУВАННЯ ТА РОЗРОБКА СЕРВЕРНОЇ ЧАСТИНИ}
\subsection{Клієнт-серверна архітектура веб-додатків} Архітектура клієнт-сервер є одним із архітектурних шаблонів програмного забезпечення та є домінуючою концепцією у створенні розподілених мережних застосунків і передбачає взаємодію та обмін даними між ними. Вона передбачає такі основні компоненти:
\begin{itemize}
	\item набір серверів, які надають інформацію або інші послуги програмам, які звертаються до них;
	\item набір клієнтів, які використовують сервіси, що надаються серверами;
	\item мережа, яка забезпечує взаємодію між клієнтами та серверами.
\end{itemize}

Сервери є незалежними один від одного. Клієнти також функціонують паралельно і незалежно один від одного. Немає жорсткої прив'язки клієнтів до серверів. Більш ніж типовою є ситуація, коли один сервер одночасно обробляє запити від різних клієнтів; з іншого боку, клієнт може звертатися то до одного сервера, то до іншого. Клієнти мають знати про доступні сервери, але можуть не мати жодного уявлення про існування інших клієнтів~\cite{douglowe1997}.

Загальноприйнятим є положення, що клієнти та сервери~--- це перш за все програмні модулі. Найчастіше вони знаходяться на різних комп'ютерах, але бувають ситуації, коли обидві програми~--- і клієнтська, і серверна, фізично розміщуються на одній машині; в такій ситуації сервер часто називається локальним.

\subsection{Мікросервісна архітектура додатків} \subsection{Мікросервісна архітектура}

Монолітна архітектура передбачає реалізацію всіх сервісів ресурсу як єдиної програмної системи. Тобто всі сервіси реалізовані за допомогою одного набору технологій (і мови програмування) і використовують загальні бібліотеки коду. Всі сервіси працюють з одним сервером баз даних.

Мікросервіси є сучасною концепцією реалізації сервісів для систем, що розвиваються. Мікросервісна архітектура полягає в створенні для кожного з логічно відокремлених компонентів системи окремого модулю, пов'язаного з рештою. Часто сервіси групують, якщо вони реалізують схожий, або тісно пов'язаний функціонал.  

Один з принципів проектування мікросервісних додатків додатків визначає, що розмір одного сервісу повинен бути таким, щоб повністю «вміщуватися» в голову програміста \cite{приходченко2018обґрунтування}.

В рамках системи закладено низку модулів, частина з яких використовує у своїй роботі доступ до сервісів Google, зокрема Google Sheets та Google Calendar. При цьому для взаємодії посередництвом Google API потрібно пройти процедуру аутентифікації (рис.~\ref{fig:GoogleServicesAuth}), закладену в методи бібліотек для основних платформ, в тому числі Node.js. Всі пакети мають відкритий вихідний код та поширюються разом з документацією.

\addCodeAsImg{% Auth sequence uml diagram

\documentclass[a4paper,10pt]{article}
\usepackage[english]{babel}
\usepackage[left=3cm, right=1cm, top=1cm, bottom=1cm]{geometry}

\usepackage{tikz-uml}

\sloppy
\hyphenpenalty 10000000

\begin{document}
\thispagestyle{empty}
\begin{center}
\begin{tikzpicture}
\begin{umlseqdiag}
	\umlactor[no ddots, x=1]{User}
	\umlboundary[no ddots, x=5]{App}
	\umlcontrol[no ddots, x=8]{oAuth2}
	\umldatabase[no ddots, x=11.5, fill=blue!20]{FileSystem}
	\umlboundary[no ddots, x=13.5]{AuthAPI}
	
	\begin{umlcall}[op=Auth required action, type=synchron, return=Response, padding=3]{User}{App}
	
		\begin{umlfragment}[type=Auth]
			\begin{umlcall}[op=Read credentials, type=synchron, return=Credentials]{App}{FileSystem}\end{umlcall}
			\begin{umlcall}[op=Authorise, type=synchron, return=oAuth2]{App}{AuthAPI}\end{umlcall}
			
			\begin{umlfragment}[type=oAuth2, label=Error, fill=green!10]
				\begin{umlcall}[op=Get new token, type=synchron, return=oAuth2]{App}{oAuth2}
					\begin{umlcall}[op=Generate URL, type=synchron, return=URL]{oAuth2}{AuthAPI}\end{umlcall}
					\begin{umlcall}[op=Prompt interface, type=synchron, return=Code]{oAuth2}{App}
						\begin{umlcall}[op=Visit URL request, type=synchron, return=Code]{App}{User}\end{umlcall}
						\begin{umlcall}[op=Write token, type=synchron]{App}{FileSystem}
							\begin{umlcall}[op=Set credentials, type=synchron, return=]{oAuth2}{oAuth2}\end{umlcall}
						\end{umlcall}
					\end{umlcall}
				\end{umlcall}
				
				\umlfpart[OK]
				\begin{umlcall}[op=Create, type=synchron, return=oAuth2]{App}{oAuth2}
					\begin{umlcall}[op=Set credentials, type=synchron, return=]{oAuth2}{oAuth2}\end{umlcall}
				\end{umlcall}
				
			\end{umlfragment}
			
		\end{umlfragment}
		
	\end{umlcall}
	
\end{umlseqdiag}
\end{tikzpicture}
\end{center}

\end{document}
}{Авторизація з сервісами Google}{fig:GoogleServicesAuth}

Для компонентів додатку, спроектованого з дотриманням мікросервісної архітектури, справедливі наступні твердження: модулі можна легко замінити, зроблено акцент на незалежність розгортання та оновлення кожного з мікросервісів; модулі організовані навколо функцій, мікросервіс виконує одну елементарну функцію; модулі можуть бути реалізовані з використанням різних мов програмування, виконуватися в різних середовищах, під управлінням різних операційних систем на різних апаратних платформах \cite[159]{кучер2018мікросервісна}.

Загалом, пріоритет віддається на користь найефективнішого для кожної конкретної функції способу розробки і виконання.


\subsection{Проектування бази даних}
\subsubsection{Розвиток комп’ютерних баз даних}

База даних – сукупність даних, організованих відповідно до певної прийнятої концепції, яка описує характеристику цих даних і взаємозв'язки між їхніми елементами. Дані у базі організовують відповідно до моделі організації даних. 

В загальному випадку базою даних можна вважати будь-який впорядкований набір даних, наприклад, паперову картотеку бібліотеки. Але все частіше термін «база даних» використовуєтьсяу контексті використання баз даних в інформаційних системах, як і самі бази даних переносяться в електронні системи в процесі інформатизації. На даний час додатки для роботи з базами даних є одними з найпоширеніших прикладних програм \cite{ситник2004проектування}.

Через тісний зв'язок баз даних з системами керування базами даних (СКБД) під терміном «база даних» нерідко неточно мається на увазі система керування базами даних. Але варто розрізняти базу даних — сховище даних, та СКБД — засоби для роботи з базою даних. Надалі, в роботі під терміном «база даних», в залежності від контексту, може матися на увазі як сукупність даних чи певні її параметри, так і СКБД, крім випадків де це не очевидно.

Розроблення перших баз даних розпочинається в 1960-ті роки. Переважно, дослідницькі роботи ведуться в проектах IBM та найбільших університетів. Пізніше, на початку 1970-х років Едгар Ф. Кодд обґрунтовує основи реляційної моделі \cite{codd1970relational}. Уперше цю модель було використано у бази даних Ingres та System R, що були лише дослідними прототипами. Проте вже в 1980-ті рр. з’являються перші комерційних версій реляційних БД Oracle та DB2. Реляційні бази даних починають успішно витісняти мережні та ієрархічні. Починаються дослідження розподілених (децентралізованих) баз даних.

\subsection{Реляційні бази даних}\label{subsection:relationModel}

Реляційна модель даних — логічна модель даних, вперше описана Едгаром Ф. Коддом \cite{codd1970relational}. В даний час ця модель є фактичним стандартом, на який орієнтуються більшість сучасних СКБД.

У реляційній моделі досягається більш високий рівень абстракції даних, ніж в ієрархічній або мережевій. Стверджується, що «реляційна модель надає засоби опису даних на основі тільки їх природної структури, тобто без потреби введення якоїсь додаткової структури для цілей машинного представлення» \cite{codd1970relational}. А це означає, що подання даних не залежить від способу їх фізичної організації, що забезпечується за рахунок використання математичного поняття відношення.

До складу реляційної моделі даних зазвичай включається теорія нормалізації. Дейт визначив наступні частини реляційної моделі даних \cite{дейт2008введение}:
\begin{itemize}
	\item структурна;
	\item маніпуляційна;
	\item цілісна.
\end{itemize}

Структурна частина моделі визначає, що єдиною структурою даних є нормалізоване n-арне відношення.

\subsubsection{Нормалізація бази даних}

Нормалізація схеми бази даних — процес розбиття одного відношення (таблиці в поняттях СУБД) відповідно до алгоритму нормалізації на кілька відношень на основі функціональних залежностей.

Нормальна форма визначається як сукупність вимог, яким має задовольняти відношення, з точки зору надмірності, яка потенційно може призвести до логічно помилкових результатів вибірки.

Таким чином, схема реляційної бази даних покроково, у процесі виконання відповідного алгоритму, переходить у першу, другу, третю і так далі нормальні форми. Якщо відношення відповідає критеріям n-ої нормальної форми та всіх попередніх нормальних форм, тоді вважається, що це відношення знаходиться у нормальній формі n-ого рівня.

\subsubsection{СКБД PostgreSQL}

PostgreSQL — широко розповсюджена система керування базами даних з відкритим вихідним кодом. Прототип був розроблений в Каліфорнійському університеті Берклі в 1987 році, пізніше проект Берклі було зупинено, а реалізацію було викладено в Інтернет під назвою Postgres95 після вдосконалення вихідного коду. Наразі підтримкою й розробкою займається група спеціалістів, які добровільно приєднались до проекту.

Сервер PostgreSQL написаний на мові C. Розповсюджується у вигляді вихідного коду, який необхідно відкомпілювати. Разом з кодом розповсюджується детальна документація.

\subsubsection{Технологія ORM}

ORM — технологія програмування, яка зв'язує бази даних з концепціями об'єктно-орієнтованих мов програмування, створюючи «віртуальну об'єктну базу даних». В об'єктно-орієнтованому програмуванні об'єкти в програмі представляють об'єкти з реального світу. 

Суть проблеми полягає в перетворенні таких об'єктів у форму, в якій вони можуть бути збережені у файлах або базах даних, і які легко можуть бути витягнуті в подальшому, зі збереженням властивостей об'єктів і відношень між ними. Ці об'єкти називають «постійними». Існує кілька підходів до розв'язання цієї задачі. Деякі пакети вирішують цю проблему, надаючи бібліотеки класів, здатних виконувати такі перетворення автоматично. Маючи список таблиць в базі даних і об'єктів в програмі, вони автоматично перетворять запити з одного вигляду в інший.

В проекті використано ORM Sequelize. Спроектовано на реалізовано у вигляді моделей та відповідним їм таблиць структуру бази даних (рис.~\ref{fig:DbScheme}).

\addCodeAsImg{\begin{umlstyle}

\umlclass[x=0, y=-3, fill=blue!30]{Department}{
	+ Name \\
	}{}
	
\umlclass[x=4, y=-3, fill=blue!30]{Faculty}{
	+ Name \\
	}{}
	
\umlclass[x=4, y=0, fill=green!30]{Teacher}{
	+ Name \\
	+ Department \\
	+ Post \\
	}{}
	
\umlclass[x=8, y=0, fill=green!30]{Subject}{
	+ Discipline \\
	+ Type \\
	+ Teacher \\
	}{}
	
\umlclass[x=12, y=-3]{Lesson}{
	+ Subject \\
	+ Subgroup \\
	+ Teacher \\
	+ Audience \\
	+ Time \\
	}{}
	
\umlclass[x=8, y=-3]{Schedule}{
	+ Name \\
	+ Using time \\
	+ Faculty \\
	+ Worker \\
	}{}
	
\umlclass[x=4, y=-6]{Worker}{
	+ Name \\
	+ Faculty \\
	+ Auth data \\
	}{}
	
\umlclass[x=4, y=-9, fill=red!30]{Speciality}{
	+ Name \\
	+ Chair \\
	}{}
	
\umlclass[x=8, y=-9, fill=red!30]{Group}{
	+ Name \\
	+ Speciality \\
	+ Cource \\
	+ Number \\
	}{}
	
\umlclass[x=12, y=-9, fill=red!30]{Subgroup}{
	+ Group \\
	+ Specialization \\
	}{}

\umlaggreg[geometry=|-,mult1=1, mult2=n, pos1=0.2, pos2=1.9]{Department}{Teacher}
\umlcompo[geometry=--,mult1=1, mult2=n, pos1=0.2, pos2=1.9]{Faculty}{Department}
\umlassoc[geometry=--,mult1=1, mult2=1, pos1=0.2, pos2=0.9]{Subject}{Teacher}
\umlcompo[geometry=|-,mult1=1, mult2=n, pos1=0.2, pos2=1.9]{Department}{Speciality}
\umlassoc[geometry=--,mult1=1, mult2=1, pos1=0.2, pos2=0.9]{Speciality}{Group}
\umlcompo[geometry=--,mult1=1, mult2=n, pos1=0.2, pos2=0.9]{Group}{Subgroup}
\umlassoc[geometry=--,mult1=n, mult2=1, pos1=0.2, pos2=0.9]{Schedule}{Faculty}
\umlcompo[geometry=--,mult1=1, mult2=n, pos1=0.2, pos2=1.9]{Schedule}{Lesson}
\umlassoc[geometry=|-,mult1=n, mult2=1, pos1=0.2, pos2=1.9]{Schedule}{Worker}
\umlassoc[geometry=--,mult1=, mult2=, pos1=0.2, pos2=1.9]{Faculty}{Worker}
\umlassoc[geometry=--,mult1=1, mult2=1, pos1=0.2, pos2=0.9]{Lesson}{Subgroup}
\umlassoc[geometry=|-,mult1=1, mult2=1, pos1=0.2, pos2=1.9]{Lesson}{Subject}

\end{umlstyle}}{Структура бази даних}{fig:DbScheme}

З погляду програміста система повинна виглядати як постійне сховище об'єктів. Він може просто створювати об'єкти і працювати з ними, а вони автоматично зберігатимуться в реляційній базі даних.


\subsection{Розробка прикладного програмного інтерфейсу} Прикладний програмний інтерфейс~--- набір визначень підпрограм, протоколів взаємодії та засобів для створення програмного забезпечення. Спрощено -- це набір чітко визначених методів для взаємодії різних компонентів. API надає розробнику засоби для швидкої розробки програмного забезпечення. API може бути для веб-базованих систем, операційних систем, баз даних, апаратного забезпечення, програмних бібліотек тощо.

\subsubsection{REST API}

REST~--- підхід до архітектури мережевих протоколів, які забезпечують доступ до інформаційних ресурсів. Був описаний і популяризований 2000 року Роєм Філдінгом, одним із творців протоколу HTTP. В основі REST закладено принципи функціонування Всесвітньої павутини і, зокрема, можливості HTTP. Філдінг розробив REST паралельно з HTTP 1.1, при цьому базувався на попередньому протоколі HTTP 1.0.

Дані можуть передаватися у вигляді певних стандартних форматів (наприклад, HTML, XML, JSON). Будь-який REST протокол (HTTP в тому числі) має підтримувати кешування, не повинен залежати від мережевого прошарку, не повинен зберігати стан системи між парами «запит-відповідь». Такий підхід забезпечує масштабовність системи і дозволяє їй розвиватись з отриманням нових вимог. Ці особливості сприяють використанню REST API при проектуванні мікросервісних додатків \cite{кучер2018мікросервісна}.

REST, як і кожен архітектурний стиль відповідає ряду обмежень архітектури додатку. Це гібридний стиль який успадковує обмеження з інших архітектурних стилів.

\paragraph{Клієнт-сервер}

Основна архітектура від якої він успадковує обмеження~--- це клієнт\,--\,сервер. Вона вимагає розділення відповідальності між частинами системи, які займаються зберіганням та обробкою даних (сервером), і тими компонентами, які займаються відображенням даних на стороні користувача та дії з цим інтерфейсом (клієнтом). Таке розділення дозволяє компонентам розвиватись незалежно один від одного та спрощує внесення змін у систему.

\paragraph{Відсутність стану}

Ще одним обмеженням є те, що акти взаємодії між сервером та клієнтом не мають стану, а отже кожен окремий запит містить всю необхідну інформацію для його обробки, і не використовує при цьому те, що сервер знає ту чи іншу інформацію з попереднього запиту.

Проте відсутність стану не означає що стану немає. Відсутність стану означає, що сервер не знає про стан клієнта. Коли клієнт, наприклад, запитує головну сторінку сайту, сервер відповідає на запитання і забуває про клієнта. Клієнт може залишити сторінку відкритою протягом кількох років, перш ніж натиснути посилання, і тоді сервер відповість на наступний запит. Тим часом сервер може відповідати на запити інших клієнтів, або нічого не робити~--- для клієнта це не має значення.

Таким чином, наприклад дані про стан сесії (користувача, який виконав вхід до системи) зберігаються на клієнті, і передаються окремо разом з кожним запитом. Це покращує масштабовність, бо сервер після закінчення обробки запиту може звільнити всі ресурси, задіяні для цієї операції, для використання їх в обробці інших запитів, без жодного ризику втратити цінну інформацію. Також спрощується контроль і пошук помилок, бо для аналізу того, що відбувається в певному запиті, досить переглянути лише на той один запит. Збільшується надійність, адже помилка в одному запиті не зачіпає інші.

\subsubsection{SOAP API}

SOAP~--- протокол для обміну повідомленнями в розподілених системах, що має певну структуру і базується на форматі XML.

Спочатку SOAP призначався, в основному, для реалізації віддаленого виклику процедур (RPC), а назва була абревіатурою: Simple Object Access Protocol~--- простий протокол доступу до об'єктів. Наразі протокол може використовуватися для обміну будь-якими повідомленнями в форматі XML, а не лише для виклику функцій процедур. SOAP є розширенням мови XML-RPC.

SOAP використовується з різними протоколами прикладного рівня: HTTP, SMTP, FTP та інші. Проте його взаємодія з кожним із цих протоколів має свої особливості, які потрібно відзначити окремо. Найчастіше SOAP використовується разом з HTTP.

SOAP є одним зі стандартів, на яких ґрунтується технологія веб-сервісів.

\subsubsection{JSON Web Token} \label{subsubsection:jwt}

\addCodeAsImg{\begin{umlstyle}

\umlactor[x=0, y=0, fill=blue!1]{unreg}
\umlactor[x=0, y=-3, fill=green!30]{user}
\umlactor[x=14, y=-3, fill=red!30]{admin}

\begin{umlsystem}[x=3, y=0]{Schedule API access}
\umlusecase[x=8, y=0, name=uc1, fill=red!30]{Sign up}
\umlusecase[x=6, y=0, name=uc2, fill=green!30]{Sign in}

\umlusecase[x=0, y=-2, name=uc3, fill=blue!1]{GET object}
\umlusecase[x=2.4, y=-2, name=uc31, fill=blue!1]{Dict}
\umlusecase[x=4, y=-2, name=uc32, fill=green!30]{User}
\umlusecase[x=6, y=-2, name=uc33, fill=red!30]{Admin}
\umlusecase[x=8, y=-2, name=uc34, fill=blue!1]{Schedule}

\umlusecase[x=0, y=-3, name=uc4, fill=green!30]{POST object}
\umlusecase[x=2.4, y=-3, name=uc41, fill=red!30]{Dict}
\umlusecase[x=4, y=-3, name=uc42, fill=red!30]{User}
\umlusecase[x=6, y=-3, name=uc43, fill=red!30]{Admin}
\umlusecase[x=8, y=-3, name=uc44, fill=green!30]{Schedule}

\umlusecase[x=0, y=-4, name=uc5, fill=green!30]{PUT object}
\umlusecase[x=2.4, y=-4, name=uc51, fill=red!30]{Dict}
\umlusecase[x=4, y=-4, name=uc52, fill=green!30]{User}
\umlusecase[x=6, y=-4, name=uc53, fill=red!30]{Admin}
\umlusecase[x=8, y=-4, name=uc54, fill=green!30]{Schedule}


\umlusecase[x=0, y=-5, name=uc6, fill=green!30]{DEL object}
\umlusecase[x=2.4, y=-5, name=uc61, fill=red!30]{Dict}
\umlusecase[x=4, y=-5, name=uc62, fill=red!30]{User}
\umlusecase[x=6, y=-5, name=uc63, fill=red!30]{Admin}
\umlusecase[x=8, y=-5, name=uc64, fill=green!30]{Schedule}


\end{umlsystem}

\umlinherit{uc33}{uc3}
\umlinherit{uc43}{uc4}
\umlinherit{uc53}{uc5}
\umlinherit{uc64}{uc6}

\umlassoc{admin}{uc1}
\umlassoc{user}{uc2}
\umlassoc{admin}{uc2}
\umlassoc{unreg}{uc3}
\umlassoc{user}{uc3}
\umlassoc{admin}{uc34}
\umlassoc{user}{uc4}
\umlassoc{admin}{uc44}
\umlassoc{user}{uc5}
\umlassoc{admin}{uc54}
\umlassoc{user}{uc6}
\umlassoc{admin}{uc64}

\end{umlstyle}
}{Доступ на виконання запитів до системи}{fig:ApiAccess}

Для забезпечення конфіденційності при обміні даними використовується JSON Web Token. Роути, що обробляють реєстраційні та авторизаційні запити, представлено на рис.~\ref{fig:ApiAccess}.

JSON Web Token це стандарт токена доступу на основі JSON, стандартизованого в RFC 7519~\cite{jones2015json}. Використовується для верифікації тверджень. JSON Web Token складається з трьох частин: заголовка, вмісту і підпису.

В корисному навантаженні зберігається будь-яка інформація, яку потрібно перевірити. Кожен ключ в корисному навантаженні відомий як «заява». Як і заголовок, корисне навантаження кодується в base64. Після отримання заголовку і корисного навантаження, обчислюється підпис.

У несеріалізованном вигляді JWT складається з заголовка і корисного навантаження, які є звичайними JSON-об'єктами.

Тема (заголовок JOSE) в основному використовується для опису криптографічних функцій, які застосовуються для підпису або шифрування токена. Тут також можна вказати додаткові властивості, наприклад, тип вмісту, хоча це рідко потрібно.

Якщо JWT підписаний або зашифрований, в заголовку вказується ім'я алгоритму шифрування. Для цього призначена заявка $alg$.

Cлово «заявка» в специфікації позначає просто частина інформації і аналогічна ключу JSON-об'єкта. Вона представлена у вигляді пари $name: value$, де $name$ завжди є рядком. Значним заявки може бути будь-який серіалізуємий тип даних. Наприклад, об'єкт JSON на рис.~\ref{fig:JsonSample1} складається з трьох заявок: $iss$, $exp$ і $http:\/\/example.com\/is\_admin$.

\addCodeAsImg{\lstinputlisting[numbers=left]{code/JWT2.tex}}{Приклад об'єкту JSON}{fig:JsonSample1}

Заявки бувають службовими і призначеними для користувача. Перші зазвичай є частиною будь-якого стандарту, наприклад, реєстру JSON Web Token Claims, і мають певні значення.

Токен можна підписати, щоб перевірити, чи не були змінені дані, що містяться в ньому. Підписаний веб-токен відомий як JWS (JSON Web Signature). У компактній серіалізовані формі у нього з'являється третій сегмент - підпис.

На відміну від підпису, який є засобом встановлення автентичності токена, шифрування забезпечує його нечитабельність.

Зашифрований JWT відомий як JWE (JSON Web Encryption). На відміну від JWS, його компактна форма має 5 сегментів, між якими ставиться крапка. Додатково до зашифрованого заголовку і корисного навантаження, він включає в себе зашифрований ключ, вектор ініціалізації і тег аутентифікації.

\subsubsection{Публічне API}

В процесі проектування створено структуру роутів, котра може використовуватися сторонніми сервісами, у тому числі~--- і без авторизації в системі, що дозволяє отримувати інформацію про розклади власними силами для подальшого використання тим чи іншим чином. 

\subsection{Розробка мікросервісів} Для обміну інформацією між користувацьким додатком та back-end частиною використовується протокол передачі гіпертекстових даних HTTP. Передачу даних забезпечує стек транспортних протоколів TCP/IP.
Одним з способів побудови мережевих HTTP-додатків є використання  асинхронного подієвого JavaScript–оточення Node. Для кожного з’єднання викликається функція зворотнього виклику, проте коли з’єднань немає Node засинає.

У Node не має функцій, що працюють напряму з I/O, тому процес не блокується ніколи. Як результат, на Node легко розробляти масштабовані системи \cite{zeiss2015node}.
Node широко використовує подієву модель, він приймає цикл подій за основу оточення, замість того, щоб використовувати його в якості бібліотеки. В інших системах відбувається блокування виклику для запуску циклу подій.

\subsubsection{Специфікація вимог до програмного забезпечення}

Одним із завдань, поставлених для реалізації мети є розроблення вимог щодо можливостей серверної частини па публічного API зокрема.

Після проведення аналізу предметної області та технологій було сформульовано наступні вимоги:

\begin{enumerate}
    \item Можливість використання API не залежить від клієнтської платформи.
    \item Розділення прав доступу.
    \item Створення нових користувачів у системі адміністратором.
    \item Авторизація користувачів у системі.
    \item Редагування даних користувачами системи у відповідності до їх прав доступу.
    \item Редагування даних адміністраторами системи у відповідності до їх прав доступу.
    \item Інтеграція з Google сервісами, зокрема Google calendar та Google Sheets.
    \item Можливість експорту та імпорту даних між системою та сервісами Google або специфікованими форматами даних.
    \item Забезпечення цілісності даних.
\end{enumerate}


\subsubsection{Проміжні обробники} \label{subs:middleware}

Express~--- це веб-фреймворк маршрутизації і тимчасової роботи з мінімальною власною функціональністю: додаток Express, по суті, являє собою серію викликів функцій тимчасової роботи.

Функції тимчасової роботи (middleware)~--- це функції, які мають доступ до об'єкта запиту (req), об'єкту відповіді (res) і до наступній функції тимчасової роботи в циклі "запит-відповідь" додатку. Наступна функція тимчасової роботи, як правило, позначається змінної next.

Функції тимчасової роботи можуть виконувати такі завдання:

\begin{enumerate}
	\item Виконання будь-якого коду.
	\item Внесення змін до об'єкти запитів і відповідей.
	\item Завершення циклу "запит-відповідь".
	\item Виклик наступної функції тимчасової роботи з стека.
	\item Якщо поточна функція тимчасової роботи завершує цикл "запит-відповідь", вона повинна викликати next () для передачі управління наступної функції тимчасової роботи. В іншому випадку запит зависне.
\end{enumerate}	

Додаток Express може використовувати такі типи проміжних оброблювачів:

\begin{enumerate}
	\item Проміжний обробник рівня додатки
	\item Проміжний обробник рівня маршрутизатора
	\item Проміжний обробник для обробки помилок
	\item Вбудовані проміжні обробники
	\item Проміжні обробники сторонніх постачальників ПО
\end{enumerate}	


\subsubsection{Маршрути}

\addCodeAsImg{\lstinputlisting[numbers=left]{code/AppMethod.tex}}{Структура визначення маршрутів}{fig:Route}

В процесі роботи використовується протокол прикладного рівня HTTP. Обмін повідомленнями йде за схемою «запит-відповідь». Для ідентифікації ресурсів HTTP використовує URL. 

В додатку обробляються основні  HTTP методи для взаємодії об’єктами (рис.~\ref{fig:ApiAccess}). 

Протокол HTTP не зберігає свого стану між парами «запит-відповідь». Компоненти, що використовують HTTP, можуть самостійно здійснювати збереження інформації про стан, пов'язаний з останніми запитами та відповідями. 

Одним з розповсюджених способів реалізації цього можна назвати так звані cookies~--- невеликі записи, що зберігаються браузером. Зазвичай, вони встановлюються при виконанні користувачем певних дій та надсилаються серверу разом з наступними запитами. 

На рис.~\ref{fig:AppSignUp} зображено процес створення адміністратором нового користувача системи. Після отримання зазначеного POST запиту, сервер перевірить, чи має користувач відповідні права (блок Auth рис.~\ref{fig:CreateOperation}), створить об’єкт користувача з відповідними правами та збереже його в базі даних.

\addCodeAsImg{% Auth sequence uml diagram

\documentclass[a4paper,10pt]{article}
\usepackage[english]{babel}
\usepackage[left=3cm, right=1cm, top=1cm, bottom=1cm]{geometry}

\usepackage{tikz-uml}

\sloppy
\hyphenpenalty 10000000

\begin{document}
\thispagestyle{empty}
\begin{center}
\begin{tikzpicture}
\begin{umlseqdiag}
	\umlactor[no ddots, x=1]{User}
	\umlboundary[no ddots, x=5]{App}
	%\umlentity[no ddots, x=8]{DbUser}
	%\umlentity[no ddots, x=11]{DbRole}
	\umldatabase[no ddots, x=14, fill=blue!20]{DB}
	
	\begin{umlcall}[op=SignUp request, type=synchron, return=Response, padding=3]{User}{App}
		\begin{umlfragment}[type=SignUp]
			\umlcreatecall[no ddots, x=8]{App}{DbUser}
			
				\begin{umlcall}[op=Init, type=synchron, return=Response]{App}{DbUser}
				
					\begin{umlcall}[op=Set data, type=synchron, return=]{DbUser}{DbUser}
							
						\umlcreatecall[no ddots, x=11]{DbUser}{DbRole}
						\begin{umlcall}[op=Set data, type=synchron, return=Role]{DbUser}{DbRole}
							\begin{umlcall}[op=Get all, type=synchron, return=List]{DbRole}{DB}
							\end{umlcall}
						\end{umlcall}
					
					\end{umlcall}
					
					\begin{umlfragment}[type=Creating, label=OK, fill=green!10]				
						\begin{umlcall}[op=Store, type=synchron]{DbUser}{DB}
						\end{umlcall}
						\begin{umlcall}[op=Success, type=synchron]{DbUser}{DbUser}
						\end{umlcall}
						\umlfpart[Error]				
						\begin{umlcall}[op=Error, type=synchron]{DbUser}{DbUser}
						\end{umlcall}
					\end{umlfragment}
					
				\end{umlcall}
				
				
			
			
			
			
			
		\end{umlfragment}
	\end{umlcall}
\end{umlseqdiag}
\end{tikzpicture}
\end{center}

\end{document}
}{Процес створення адміністратором нового користувача системи}{fig:AppSignUp}

Створений користувач може входити до системи (рис.~\ref{fig:AppSignIn}) для виконання певних задач з користування системою.

\addCodeAsImg{% Auth sequence uml diagram

\documentclass[a4paper,10pt]{article}
\usepackage[english]{babel}
\usepackage[left=3cm, right=1cm, top=1cm, bottom=1cm]{geometry}

\usepackage{tikz-uml}

\sloppy
\hyphenpenalty 10000000

\begin{document}
\thispagestyle{empty}
\begin{center}
\begin{tikzpicture}
\begin{umlseqdiag}
	\umlactor[no ddots, x=1]{User}
	\umlboundary[no ddots, x=5]{App}
	\umldatabase[no ddots, x=14, fill=blue!20]{DB}
	
	\begin{umlcall}[op=SignIn request, type=synchron, return=Response, padding=3]{User}{App}
	
		\begin{umlfragment}[type=SignIn]
			\umlcreatecall[no ddots, x=8]{App}{DbUser}
				\begin{umlcall}[op=Init, type=synchron, return=Response]{App}{DbUser}
					\begin{umlcall}[op=Find one, type=synchron, return=User]{DbUser}{DB}\end{umlcall}	
					
					\umlcreatecall[no ddots, x=11]{DbUser}{JWT}
					\begin{umlcall}[op=Check password, type=synchron, return=Response]{DbUser}{JWT}\end{umlcall}
					
					\begin{umlfragment}[type=Validate, label=OK, fill=green!10]
						\begin{umlcall}[op=Create token, type=synchron, return=Token]{DbUser}{JWT}\end{umlcall}		
						\begin{umlcall}[op=Success, type=synchron]{DbUser}{DbUser}\end{umlcall}
						\umlfpart[Error]				
						\begin{umlcall}[op=Error, type=synchron]{DbUser}{DbUser}\end{umlcall}
					\end{umlfragment}
					
				\end{umlcall}	
		\end{umlfragment}
		
	\end{umlcall}
	
\end{umlseqdiag}
\end{tikzpicture}
\end{center}

\end{document}
}{Вхід користувача до системи}{fig:AppSignIn}

Можна зазначити, що адміністратор теж є користувачем, проте з особливими правами (диференціація на рис.~\ref{fig:Route}).


\subsubsection{Маршрути}

\addCodeAsImg{\lstinputlisting[numbers=left]{code/AppMethod.tex}}{Структура визначення маршрутів}{fig:Route}

В процесі роботи використовується протокол прикладного рівня HTTP. Обмін повідомленнями йде за схемою «запит-відповідь». Для ідентифікації ресурсів HTTP використовує URL. 

В додатку обробляються основні  HTTP методи для взаємодії об’єктами (рис.~\ref{fig:ApiAccess}). 

Протокол HTTP не зберігає свого стану між парами «запит-відповідь». Компоненти, що використовують HTTP, можуть самостійно здійснювати збереження інформації про стан, пов'язаний з останніми запитами та відповідями. 

Одним з розповсюджених способів реалізації цього можна назвати так звані cookies — невеликі записи, що зберігаються браузером. Зазвичай, вони встановлюються при виконанні користувачем певних дій та надсилаються серверу разом з наступними запитами. 

На рис.~\ref{fig:AppSignUp} зображено процес створення адміністратором нового користувача системи. Після отримання зазначеного POST запиту, сервер перевірить, чи має користувач відповідні права (блок Auth рис.~\ref{fig:CreateOperation}), створить об’єкт користувача з відповідними правами та збереже його в базі даних.

\addCodeAsImg{% Auth sequence uml diagram

\documentclass[a4paper,10pt]{article}
\usepackage[english]{babel}
\usepackage[left=3cm, right=1cm, top=1cm, bottom=1cm]{geometry}

\usepackage{tikz-uml}

\sloppy
\hyphenpenalty 10000000

\begin{document}
\thispagestyle{empty}
\begin{center}
\begin{tikzpicture}
\begin{umlseqdiag}
	\umlactor[no ddots, x=1]{User}
	\umlboundary[no ddots, x=5]{App}
	%\umlentity[no ddots, x=8]{DbUser}
	%\umlentity[no ddots, x=11]{DbRole}
	\umldatabase[no ddots, x=14, fill=blue!20]{DB}
	
	\begin{umlcall}[op=SignUp request, type=synchron, return=Response, padding=3]{User}{App}
		\begin{umlfragment}[type=SignUp]
			\umlcreatecall[no ddots, x=8]{App}{DbUser}
			
				\begin{umlcall}[op=Init, type=synchron, return=Response]{App}{DbUser}
				
					\begin{umlcall}[op=Set data, type=synchron, return=]{DbUser}{DbUser}
							
						\umlcreatecall[no ddots, x=11]{DbUser}{DbRole}
						\begin{umlcall}[op=Set data, type=synchron, return=Role]{DbUser}{DbRole}
							\begin{umlcall}[op=Get all, type=synchron, return=List]{DbRole}{DB}
							\end{umlcall}
						\end{umlcall}
					
					\end{umlcall}
					
					\begin{umlfragment}[type=Creating, label=OK, fill=green!10]				
						\begin{umlcall}[op=Store, type=synchron]{DbUser}{DB}
						\end{umlcall}
						\begin{umlcall}[op=Success, type=synchron]{DbUser}{DbUser}
						\end{umlcall}
						\umlfpart[Error]				
						\begin{umlcall}[op=Error, type=synchron]{DbUser}{DbUser}
						\end{umlcall}
					\end{umlfragment}
					
				\end{umlcall}
				
				
			
			
			
			
			
		\end{umlfragment}
	\end{umlcall}
\end{umlseqdiag}
\end{tikzpicture}
\end{center}

\end{document}
}{Процес створення адміністратором нового користувача системи}{fig:AppSignUp}

Створений користувач може входити до системи (рис.~\ref{fig:AppSignIn}) для виконання певних задач з користування системою.

\addCodeAsImg{% Auth sequence uml diagram

\documentclass[a4paper,10pt]{article}
\usepackage[english]{babel}
\usepackage[left=3cm, right=1cm, top=1cm, bottom=1cm]{geometry}

\usepackage{tikz-uml}

\sloppy
\hyphenpenalty 10000000

\begin{document}
\thispagestyle{empty}
\begin{center}
\begin{tikzpicture}
\begin{umlseqdiag}
	\umlactor[no ddots, x=1]{User}
	\umlboundary[no ddots, x=5]{App}
	\umldatabase[no ddots, x=14, fill=blue!20]{DB}
	
	\begin{umlcall}[op=SignIn request, type=synchron, return=Response, padding=3]{User}{App}
	
		\begin{umlfragment}[type=SignIn]
			\umlcreatecall[no ddots, x=8]{App}{DbUser}
				\begin{umlcall}[op=Init, type=synchron, return=Response]{App}{DbUser}
					\begin{umlcall}[op=Find one, type=synchron, return=User]{DbUser}{DB}\end{umlcall}	
					
					\umlcreatecall[no ddots, x=11]{DbUser}{JWT}
					\begin{umlcall}[op=Check password, type=synchron, return=Response]{DbUser}{JWT}\end{umlcall}
					
					\begin{umlfragment}[type=Validate, label=OK, fill=green!10]
						\begin{umlcall}[op=Create token, type=synchron, return=Token]{DbUser}{JWT}\end{umlcall}		
						\begin{umlcall}[op=Success, type=synchron]{DbUser}{DbUser}\end{umlcall}
						\umlfpart[Error]				
						\begin{umlcall}[op=Error, type=synchron]{DbUser}{DbUser}\end{umlcall}
					\end{umlfragment}
					
				\end{umlcall}	
		\end{umlfragment}
		
	\end{umlcall}
	
\end{umlseqdiag}
\end{tikzpicture}
\end{center}

\end{document}
}{Вхід користувача до системи}{fig:AppSignIn}

Можна зазначити, що адміністратор теж є користувачем, проте з особливими правами (диференціація на рис.~\ref{fig:Route}).

\subsubsection{Моделі та CRUD-операції} \label{subs:crud}

В процесі проектування закладено серію моделей, що відповідають об’єктам предметної області. В рамках системи зберігаються в базі даних у вигляді таблиць з певними взаємозвязками (реляційну модель описано в підрозділі \ref{subsection:relationModel}). Для доступу до даних використовуються основні HTTP методи, що відповідають операціям CRUD (від Create, Read, Update, Delete), їх перелічено нижче.

\paragraph{GET}
\addCodeAsImg{% Auth sequence uml diagram

\documentclass[a4paper,10pt]{article}
\usepackage[english]{babel}
\usepackage[left=3cm, right=1cm, top=1cm, bottom=1cm]{geometry}

\usepackage{tikz-uml}

\sloppy
\hyphenpenalty 10000000

\begin{document}
\thispagestyle{empty}
\begin{center}
\begin{tikzpicture}
\begin{umlseqdiag}
	\umlactor[no ddots, x=1]{User}
	\umlboundary[no ddots, x=5]{App}
	\umldatabase[no ddots, x=14, fill=blue!20]{DB}
	
	\begin{umlcall}[op=Post request, type=synchron, return=Response, padding=3]{User}{App}
		\umlcreatecall[no ddots, x=8]{App}{JWT}
		\begin{umlcall}[op=Init, type=synchron, return=Response]{App}{JWT}
			\begin{umlcall}[op=Verify JWT, type=synchron]{JWT}{JWT}\end{umlcall}
		\end{umlcall}
		
		\begin{umlfragment}[type=Main, label=OK]
	
			\umlcreatecall[no ddots, x=11]{App}{Object}
			\begin{umlcall}[op=Parameters, type=synchron, return=Object]{App}{Object}
				\begin{umlcall}[op=Select query, type=synchron, return=Rows]{Object}{DB}\end{umlcall}
					
			\end{umlcall}	
			
			\umlfpart[Error]
			
			\begin{umlcall}[op=Error, type=synchron]{App}{App}\end{umlcall}
		
		\end{umlfragment}
	\end{umlcall}
		
	
\end{umlseqdiag}
\end{tikzpicture}
\end{center}

\end{document}
}{Виконання запиту на отримання об’єкта}{fig:ReadOperation}

Запитує вміст вказаного ресурсу, який може приймати параметри, що передаються в URI (рис.~\ref{fig:ReadOperation}). Згідно зі стандартом, ці запити є ідемпотентними — багатократне повторення одного і того ж запиту GET приводить до однакових результатів (за умови, що сам ресурс не змінився за час між запитами).

В запропонованій реалізації запит GET має дві версії — з параметром (ID) та без нього. Останній виконує дію (надає користувачу) не до конкретного об’єкту, а до всієї множини, що є необхідним в певних ситуаціях (наприклад, відображення списку всіх викладачів за певним критерієм).

\paragraph{HEAD}

Аналогічний GET, за винятком того, що у відповіді сервера відсутнє тіло. Це може бути необхідно для отримання мета-інформації.

\paragraph{POST}
\addCodeAsImg{% Auth sequence uml diagram

\documentclass[a4paper,10pt]{article}
\usepackage[english]{babel}
\usepackage[left=3cm, right=1cm, top=1cm, bottom=1cm]{geometry}

\usepackage{tikz-uml}

\sloppy
\hyphenpenalty 10000000

\begin{document}
\thispagestyle{empty}
\begin{center}
\begin{tikzpicture}
\begin{umlseqdiag}
	\umlactor[no ddots, x=1]{User}
	\umlboundary[no ddots, x=5]{App}
	\umldatabase[no ddots, x=14, fill=blue!20]{DB}
	
	\begin{umlcall}[op=Post request, type=synchron, return=Response, padding=3]{User}{App}
		\umlcreatecall[no ddots, x=8]{App}{JWT}
		\begin{umlcall}[op=Init, type=synchron, return=Response]{App}{JWT}
			\begin{umlcall}[op=Verify JWT, type=synchron]{JWT}{JWT}\end{umlcall}
		\end{umlcall}
		
		\begin{umlfragment}[type=Main, label=OK]
	
			\begin{umlfragment}[type=Create, fill=green!20]
				\umlcreatecall[no ddots, x=11]{App}{Object}
				\begin{umlcall}[op=Init, type=synchron, return=Object]{App}{Object}
					\begin{umlcall}[op=Store, type=synchron]{Object}{DB}\end{umlcall}
						
				\end{umlcall}	
			\end{umlfragment}
			
			\umlfpart[Error]
			
			\begin{umlcall}[op=Undo creation, type=synchron, return=Error]{App}{Object}\end{umlcall}
		
		\end{umlfragment}
	\end{umlcall}
		
	
\end{umlseqdiag}
\end{tikzpicture}
\end{center}

\end{document}



, label=OK, fill=green!10]
						\begin{umlcall}[op=Create token, type=synchron, return=Token]{DbUser}{JWT}\end{umlcall}		
						\begin{umlcall}[op=Success, type=synchron]{DbUser}{DbUser}\end{umlcall}
						\umlfpart[Error]		
						
						
\begin{umlcall}[op=Find one, type=synchron, return=User]{Object}{DB}\end{umlcall}	
				
				\begin{umlcall}[op=Check password, type=synchron, return=Response]{Object}{JWT}\end{umlcall}
				
				\begin{umlfragment}[type=Validate, label=OK, fill=green!10]
					\begin{umlcall}[op=Create token, type=synchron, return=Token]{Object}{JWT}\end{umlcall}		
					\begin{umlcall}[op=Success, type=synchron]{Object}{Object}\end{umlcall}
					\umlfpart[Error]				
					\begin{umlcall}[op=Error, type=synchron]{Object}{Object}\end{umlcall}
				\end{umlfragment}}{Виконання запиту на створення з аутентифікацією}{fig:CreateOperation}

Передає дані (наприклад, з форми на веб-сторінці) заданому ресурсу. При цьому передані дані включаються в тіло запиту. На відміну від методу GET, метод POST не є ідемпотентним, тобто багатократне повторення одних і тих же запитів POST може повертати різні результати (рис.~\ref{fig:CreateOperation}).

На першому етапі відбувається перевірка доступу користувача это створення об’єкту цього типу (авторизація), відповідно до прав доступу (рис.~\ref{fig:ApiAccess}).

\paragraph{PUT}
\addCodeAsImg{\begin{umlstyle}

\begin{umlseqdiag}
	\umlactor[no ddots, x=1]{User}
	\umlboundary[no ddots, x=5]{App}
	\umldatabase[no ddots, x=14, fill=blue!20]{DB}
	
	\begin{umlcall}[op=path request, type=synchron, return=sesponse, padding=3]{User}{App}
		\begin{umlcall}[op=auth procedure, type=synchron]{App}{App}\end{umlcall}
		
		\begin{umlfragment}[type=Update, label=OK, fill=green!20]
				\umlcreatecall[no ddots, x=11]{App}{Object}
				\begin{umlcall}[op=parameters, type=synchron, return=object]{App}{Object}
					\begin{umlcall}[op=select query, type=synchron, return=rows]{Object}{DB}\end{umlcall}
					\begin{umlcall}[op=change, type=synchron]{Object}{Object}\end{umlcall}
					\begin{umlcall}[op=store, type=synchron, return=result]{Object}{DB}\end{umlcall}
				\end{umlcall}	
			
			\umlfpart[Error]
			
			\begin{umlcall}[op=error, type=synchron]{App}{App}\end{umlcall}
		
		\end{umlfragment}
	\end{umlcall}
		
	
\end{umlseqdiag}

\end{umlstyle}
}{Виконання запиту на модифікацію існуючого об’єкту}{fig:UpdateOperation}

Завантажує вказаний ресурс на сервер. В розроблюваній системі використовується для редагування існуючих даних (рис.~\ref{fig:UpdateOperation}). 

В процесі виконання, спочатку з бази даних силами ORM вибирається конкретний об’єкт, в нього вносяться зміни, після чого він записується до сховища на заміну попередньої версії.

\paragraph{PATCH}

Завантажує частину ресурсу на сервер. При розробці необхідності у використанні не знайдено.

\paragraph{DELETE}
\addCodeAsImg{\begin{umlstyle}

\begin{umlseqdiag}
	\umlactor[no ddots, x=1]{User}
	\umlboundary[no ddots, x=5]{App}
	\umldatabase[no ddots, x=14, fill=blue!20]{DB}
	
	\begin{umlcall}[op=delete request, type=synchron, return=sesponse, padding=3]{User}{App}
		\begin{umlcall}[op=auth procedure, type=synchron]{App}{App}\end{umlcall}
		
		\begin{umlfragment}[type=Delete, label=OK, fill=green!20]
				\umlcreatecall[no ddots, x=11]{App}{Object}
				\begin{umlcall}[op=parameters, type=synchron, return=object]{App}{Object}
					\begin{umlcall}[op=select query, type=synchron, return=rows]{Object}{DB}\end{umlcall}
					\begin{umlcall}[op=set timestamp, type=synchron]{Object}{Object}\end{umlcall}
					\begin{umlcall}[op=store, type=synchron, return=result]{Object}{DB}\end{umlcall}
				\end{umlcall}	
				
				
			\umlfpart[Error]
			
			\begin{umlcall}[op=error, type=synchron]{App}{App}\end{umlcall}
		
		\end{umlfragment}
		
	\end{umlcall}
		
	\umlnote[x=8, y=-5.25, fill=cyan!20]{Object}{Record don't deletes really, but deleting timestamp sets to current}
	
\end{umlseqdiag}

\end{umlstyle}
}{Виконання запиту на видалення об’єкту}{fig:DeleteOperation}

Видаляє вказаний ресурс.
Слід звернути увагу, що в процесі виконання запиту на видалення об’єкту в системі, видалення як такого не відбувається. Замість цього в окреме поле таблиці вноситься інформація про час виконання цієї процедури (рис.~\ref{fig:DeleteOperation}).

Такий спосіб реалізації дозволяє з однієї сторони приховати дані, відмічені як видалені від подальшого використання, а з іншої — зберегти їх там, де вони вже використовуються. В іншому випадку, у зв’язку з реляційністю бази потрібно було б вирішувати дилему — або проводити циклічне видалення для збереження цілісності даних, втрачаючи всі об’єкти, що посилаються на той, що видаляється; або ускладнювати структури даних, що потенційно призведе до дублювання даних.


\subsection{Подальша робота}
\subsubsection{Використання QR-кодів}
\addimg{QRcode.png}{0.25}{Приклад QR-коду з посиланням}{fig:QRcode}

Було проаналізовано перспективи при використанні QR-кодів (рис.~\ref{fig:QRcode}) з метою супроводження традиційного паперового розкладу (та інших документів), що публікується на стендах.

Хоча термін «QR code» є зареєстрованим товарним знаком японської корпорації «DENSO Corporation», їх використання не обкладається ніякими ліцензійними відрахуваннями, коди описані та опубліковані як стандарти ISO \cite{воронкін2014можливості}. Основна перевага QR-коду – легке розпізнавання скануючим обладнанням (за допомогою мобільного телефону, планшета або ноутбука з камерою, на яких встановлена програма для зчитування кодів, тощо).

Одним з способів використання QR-кодів в навчальному процесі, крім запропонованих (зокрема, задля забезпечення швидкого доступу до навчально-методичного забезпечення, довідкової літератури, веб-сервісів навчального закладу) \cite[146]{воронкін2014можливості},  можна назвати надання доступу до електронної версії розкладу.


\anonsection{ВИСНОВКИ}

Для виконання поставлених завдань було проведено аналіз характеристик існуючих систем планування, зокрема обсяг їх можливостей.

При підготовці до проектування було приділено увагу окремим частини процесу підготовки розкладу на прикладі факультету комп’ютерних наук, фізики та математики ХДУ.

На основі проведеного аналізу розроблено базові вимоги щодо можливостей додатку та його інтерфейсу.

Суттєву частку роботи приділено аналізу існуючих технологій всіх рівнів для створення веб-додатків. Детально досліджено роботу клієнт-серверних додатків та взаємодію через API (прикладний програмний інтерфейс). 

Відповідно до створених вимог розроблено проект додатку у відповідності до загальноприйнятих принципів побудови веб-додатків.

При реалізації веб-додатку використано бібліотеку React, завдяки якій було розроблено набір компонентів, що використовувались як в веб-додатку, так і в мобільному додатку. Останнє було досягнуто завдяки вибору платформи React Native~--- фреймворку для розробки мобільних додатків з відкритим кодом.

Розроблені додатки використовують публічний API для реалізації бізнес-логіки та взамодії з серверною частиною.


При розробці проекту використовується система контролю версій git з публічним репозиторієм на сервісі GitHub (github.com/ Rembut/gCalShedule), що дозволяє використовувати сучасні методи сумісної роботи та, одночасно з тим, дозволяє використовувати результати проведеного дослідження всім охочим під ліцензією MIT.
 % Заключение
\begingroup 
\renewcommand{\section}[2]{\anonsection{БІБЛІОГРАФІЯ}}
\begin{thebibliography}{00}

\printbibliography[heading=none]

\end{thebibliography}
\endgroup
 % Библиографический список

\end{document}
%%% Конец документа
