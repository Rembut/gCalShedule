% За основу взят github.com/Amet13/bachelor-diploma

\documentclass[a4paper,14pt]{extarticle} % 14й шрифт
\input{PREAMBLE.tex} % Подключаем преамбулу

\hypersetup{
    colorlinks, urlcolor={black}, % Все ссылки черного цвета, кликабельные
    linkcolor={black}, citecolor={black}, filecolor={black},
    pdfauthor={Сенчишен Денис Олександрович},
    pdftitle={Проектування та розробка мікросервісів для веб-додатку редагування розкладу}
}

\addbibresource{bibliography.bib} % Библиографический справочник

%%% Начало документа
\begin{document}
\thispagestyle{empty}

{\centering{\bfseries
МІНІСТЕРСТВО ОСВІТИ І НАУКИ УКРАЇНИ

ХЕРСОНСЬКИЙ ДЕРЖАВНИЙ УНІВЕРСИТЕТ

Факультет комп'ютерних наук, фізики та інформатики

Кафедра інформатики, програмної інженерії та економічної кібернетики

\vfill

ПРОЕКТУВАННЯ ТА РОЗРОБКА UI ВЕБ-ДОДАТКУ РЕДАГУВАННЯ РОЗКЛАДУ

\vfill

Дипломна робота

}

на здобуття ступеня вищої освіти бакалавр

}

\vfill

\hfill\begin{minipage}[t]{0.65\textwidth}
Виконав студент 4 курсу 431 групи 

напряму підготовки 6.040302 Інформатика

Воробйов Євгеній Андрійович

Керівники: доктор педагогічних наук, доцент

Круглик Владислав Сергійович,

кандидат фізико-математичних наук, доцент

Єрмолаєв Вадим Анатолійович 

Рецензент: 

кандидат фізико-математичних наук, професор

Кузьмич Валерій Іванович


\end{minipage}

\vfill

{\centering
Херсон --- 2019

} % Титульная страница

\tableofcontents % Содержание 
\clearpage

\anonsection{ВСТУП}

Якість підготовки спеціалістів у закладах освіти і особливо ефективність використання науково-педагогічного потенціалу залежать певною мірою від рівня організації навчального процесу.

Одна з основних складових цього процесу — розклад занять — регламентує трудовий ритм, впливає на творчу віддачу викладачів, тому його можна вважати фактором оптимізації використання обмежених ресурсів — викладацького складу і аудиторного фонду.

Проблему складання розкладу слід розглядати не тільки як трудомісткий процес, об'єкт автоматизації з використанням комп’ютера, але і як проблему оптимального керування. 

Оскільки всі фактори, що впливають на розклад, практично неможливо врахувати, а інтереси учасників навчального процесу різноманітні, задача складання розкладу є багатокритеріальною з нечіткою множиною факторів.

Незалежно від алгоритму побудови розкладу, виникає прикладна проблема з інструментів різних рівнів, що використовуються в процесі. Саме ним і буде присвячено проведену роботу.

Актуальність дослідження полягає в необхідності забезпечення всіх учасників освітнього процесу доступом до актуальної версії розкладу занять у будь-який час, а також можливості спрощення процесу формування розкладу та подальшої інформатизації освітнього процесу.

Об’єкт дослідження — системи для планування розкладу. Предмет дослідження —  веб-додаток для планування розкладу в закладах освіти з поділом учнів (вихованців, здобувачів освіти тощо) на стабільні академічні групи.

Метою роботи є проектування розширюваного веб-додатку редагування розкладу та мобільного додатку для перегляду розкладу в закладах освіти з можливістю використання всіма учасниками освітнього процесу та розробка їх робочих прототипу.

Для реалізації мети поставлено наступні завдання:
\begin{enumerate}
	\item проаналізувати характеристики існуючих систем планування, зокрема обсяг їх можливостей;
	\item проаналізувати окремі частини процесу підготовки розкладу на прикладі факультету комп’ютерних наук, фізики та математики ХДУ;
	\item на основі проведеного аналізу розробити вимоги щодо можливостей додатків та їх інтерфейсів;
	\item відповідно до створених вимог розробити проект додатку;
	\item розробити робочий прототип і інтерфейс додатку;
	\item використовувати публічне API розробленого сервісу системи пыдтримки редагування розкладу для збережання та отримання даних;
	\item обґрунтувати використані технології при проектуванні клієнтської частини.
\end{enumerate}

Очікується, що спроектований продукт буде придатний до використання всіма учасниками освітнього процесу в ЗВО.
Робота складається з 2 розділів, містить \totalfigures\ рисунків.
 % Введение

\section{Аналіз систем планування та інформування}
\subsection{Задачі систем планування} Ідея планування робіт існує стільки, скільки існує людська цивілізація, адже ще в неоліті, з переходом до тваринництва і землеробства, постають задачі з контролем циклічних процесів, що і викликало у подальшому створення календаря і писемності для фіксування задач.

З розвитком та індустріалізацією суспільства класи задач, що вимагають бути покритими детальним плануванням, суттєво розширилися. Черговим етапом розвитку таких технологій стало виникнення електронно-обчислювальних машин і впровадження їх у використання в промисловості.

В подальшому використання таких систем виходить за межі корпоративних систем підприємств і все частіше ними починають користуватися люди для планування власного часу і вирішення особистих задач.


\subsection{Порівняння сучасних сервісів планування та інформування} 
На сьогодні, кожна людина так чи інакше стикається у повсякденному житті з системами, пов'язаними з контролем часу та завдань (рис.~\ref{fig:LightningOutlook}).

\addtwoimghere{MozillaLightning.png}{MSOutlook.png}{0.45}{Mozilla Lightning та MicroSoft Outlook}{fig:LightningOutlook}


\subsection{Корпоративні інформаційні системи} 
\label{subsubs:KIS}

Корпоративна інформаційна система — це інформаційна система, яка підтримує автоматизацію функцій управління на підприємстві і постачає інформацію для прийняття управлінських рішень. У ній реалізована управлінська ідеологія, яка об'єднує бізнес-стратегію підприємства і прогресивні інформаційні технології.

У загальному визначенні «автоматизована система» — сукупність керованого об’єкта й автоматичних керувальних пристроїв, у якій частину функцій керування виконує людина. Вона представляє собою організаційно-технічну систему, що забезпечує вироблення рішень на основі автоматизації інформаційних процесів у різних сферах діяльності. 

Сучасні автоматизовані системи управління навчальним процесом у  закладах вищої освіти здатні вирішувати велику кількість функцій, а саме:
\begin{itemize}
	\item планування, контроль та аналіз навчальної діяльності;
	\item оперативний доступ до інформації про навчальний процес;
	\item єдину систему звітів, як внутрішніх, так і за вимогами МОН України;
	\item системи безпеки даних з урахуванням вимог законодавства;
	\item облік контингенту студентів та співробітників;
	\item проведення вступної кампанії;
	\item формування пакетів даних з метою виготовлення тих чи інших документів.
\end{itemize}

Функціонування будь-якої автоматизованої системи можна швидко адаптувати до особливостей навчального процесу конкретного навчального закладу, до локальних мереж різного рівня, що допомагає розширити коло користувачів (адміністрації, викладачів і студентів) для оперативного забезпечення їх необхідною інформацією. 

Отже, використання таких систем дає змогу не тільки удосконалити якість планування навчального процесу, а й оперативність управління ним.

Не зважаючи на всі переваги, які надає використання автоматизованих систем, досі далеко не в кожному закладі вони впроваджені чи використовуються в повній мірі з тих чи інших причин — інерційності поглядів адміністрації, супротив працівників або «саботаж» на місцях, відсутність фінансової або організаційної можливості.

\subsubsection{Єдине інформаційно-освітнє серидовище ХДУ}

Разом з осучасненням освітнього процесу у цілому актуальною є заадча з об'єднання існуючих інформаційних систем університету.

В результаті проведеної інтеграції планується отримати так званий <<Особистий кабінет студента>>, в котрому студент, як основний учасник освітнього процесу, матиме доступ до всієї необхідної інформації. Наразі інформаційне серидовище включає в себе низку в цілому незалежних один від одного проектів, а саме:
\begin{itemize}
	\item web-портал університету \cite{KspuEdu};
	\item система дистанційного навчання <<KSU Online>> \cite{KsuOnline};
	\item система дистанційної освіти <<Херсонський Віртуальний Університет>> \cite{KsuDis}.
	\item програмний комплекс <<ST-Абітурієнт>>, що використовується для підтримки процесу прийому документів абітурієнтів та обробки заяв про зарахування, результатів вступних іспитів тощо;
	\item програмний комплекс <<Інформаційно-аналітична система (ІАС)>>, що дозволяє вести облік співробітників і студентів, бухгалтерський облік, контроль за матеріальними цінностями (розділ~\ref{subs:ias}), проте лише частково охоплює навчальний процес;
	\item сервіс <<KSU Feedback>> призначений для проведення анонімного або звичайного голосування за визначеними критеріями серед строго респондентів \cite{KsuFeedback};
	\item сервіс  <<Пошук книг в електронному каталозі бібліотеки>> надає доступ до каталогу в будь який момент \cite{eLibrary};
	\item web-портал <<Збірник наукових праць <<Інформаційні технології в освіті>> (ІТО)>> є каналом поширення та передачі знань, де вчені, практики та дослідники можуть обговорювати, аналізувати, критикувати, синтезувати, спілкуватися та підтримувати розробку та впровадження ІТ і пов'язаних з ними наслідків у всіх аспектах їх використання у сфері освіти \cite{ITO};
	\item web-портал <<Чорноморський ботанiчний журнал>>, но котрому у відкритому доступі знаходяться електронні версії всіх статей у форматі pdf, опублікованих у журналі з 2005 року.
\end{itemize}

При цьому наразі відсутня будь-яка інтеграція між сервісами та сайтами, крім посилань на певну частину з нах на головній сторінці web-порталу університету.


\subsection{Обгрунтування використаних технологій} 
У розробці веб-додатків на сьогодняшній день використувується велика кількість мов програмування, таких як: C\#, Java, JavaScript, Python, PHP. Кожен розробник обирає для себе ту мову, яка йому здається найбільш відповідною для певної задачі. В загалом усі вище перераховані мови, окрім, може JavaScript, традиційно використовуються для розробки бекенд-частини веб-ресурсів, або для генерації фронтенду на серверній стороні та відправки сгенерованої сторінки клієнту. Для кожної конкретної мови є певні фреймворки, що спрощують створення веб-додатків.

Фреймворк - це інфраструктура програмних рішень, що полегшує розробку складних систем.

Взагалом фреймворки можуть включати в себе бібліотеки, що створені для вирішення певних задач та найкращі сталі практики з вирішення тих чи інших питань. Головне завдання фреймворків - зменшити об'єм однотипної праці, що розробник додатку виконує у кожному своєму проекті. Хоча кожна мова програмування й має свої фреймворки, їх сутність у цілому залишається для певних задач доволі близькою: використання певних шаблонів проектування, що допомагають зробити програмний код більш зрозумілим та простим для подальшої підтримки та ускладнення.

Найбільш поширеними фреймворками є: для C\# є .NET, для Java - Spring, Hibernate, для JavaScript - Node, React, Vue, Angular, для Python - Django, для PHP - Laravel. Кожен фреймворк та мова програмування мають свої переваги та недоліки. Тому, коли перед нами постало завдання вибрати, на якому саме стеку технологій ми будемо розробляти веб-додаток, нами було проаналізовано кожну із вищепредставлених мов.

Наш вібір зупинився на JavaScript як для серверної, так і для клієнтської частини. Адже, якщо серверна та клієнтська частина написани з використанням однієї мови програмування, це сильно спрощує розробку та підримку веб-додатку~\cite{vedantani2017}. Для серверної частини нами було вирішено використовувати фреймворк Node.js, тому що він добре поєднується із будь-яким фронтенд фреймворком, має велике коло шанувальників, постійно оновлюється, що підвищує його безпечність, та хорошу документацію.  

\subsubsection{Семантичне версіювання}

В процесі розробки програмного забезпечення можливе виникнення проблеми під назвою <<пекло залежностей>>. 

Суть полягає в тому, що при збільшенні розмірів програмної системи, збільшується кількість бібліотек та пакетів, що використовуються в ній. При цьому, кожен з них, зазвичай, вимагає для своєї роботи деякі інші бібліотеки певних версій. У разі, якщо документація програмного забезпечення надто вільна, то рано чи піздно виникає проблема невідповідності між фактично необхідною версією, вказаною в документації та встановленою, що негативно позначається на всьому процесі розробки програмного забезпечення.

Для вирішення цієї проблеми пропонується простий набір правил і вимог, що визначають як встановлюються і збільнуються номери версій. Для роботи системи необхідно створити і описати публічне API програмного продукту. Після цього будь-які зміни в версії визначаються певною зміної її номера.

Розглянемо формат версій X.Y.Z (мажорна, мінорна, патч).

Зміни, що не впливають на API, збільнують патч-версію. Зворотньо-сумістні зміни та розширення API збільшують мінорну версію. І, нарешті, несумістні зміни API збільшують мажорну версію.

Ця система називатиметься <<Семантичне версіювання>>.

Мажорна версія <<0>> (0.Y.Z) призначена для початкової розробки, публічний API не має розглядатися як стабільний. Версія 1.0.0 визначає публічний API, після цього релізу вона змінюватиметься відповідно до змін в API. Після чергової зміни мінорної версії патч-версія змінюється на <<0>>, аналогічні зміни відбуваються зі зміною мажорної версії.

Крім зазначених правил, специфікація семантичного версіонування~\cite{semver} визначає додатково певні деталі та поради щодо його практичного використання, зокрема для продуктів, що мають складну систему релізів та передрелізних версій.

\subsubsection{Latex} \label{subsub:latex}

\TeX -- це створена чудовим американським математиком і програмістом Дональдом Кнутом система для верстки текстів з формулами. Сам по собі TEX є спеціалізованою мовою програмування (Кнут не тільки придумав мову, а й написав для нього транслятор, причому таким чином, що він працює абсолютно однаково на самих різних комп'ютерах), на якому пишуться видавничі системи, що використовуються на практиці. Точніше кажучи, кожна видавнича система на базі TEXа є пакетом макросів (макропакет) цієї мови. LATEX -- це створена Леслі Лампортом видавнича система на базі TEXа~\cite{львовский2003latex}.

Всі видавничі системи на базі TEXа володіють перевагами, закладеними в самому TEXе. Для новачка їх можна описати однією фразою: надрукований текст виглядає «зовсім як у книзі». LATEX, як видавнича система, надає зручні і гнучкі засоби досягти цього книжкового якості. Зокрема, вказавши за допомогою простих засобів структуру тексту, автор може не вникати в деталі оформлення, причому ці деталі при необхідності неважко змінити (щоб, скажімо, змінити шрифт, яким друкуються заголовки, не треба нишпорити по всьому тексту, змінюючи все заголовки , а досить замінити одну сходинку в «стильовому файлі»). Такі речі, як нумерація розділів, посилання, зміст і т. П. Виходять майже що «самі собою». Величезним плюсом систем на базі TEXа є висока якість та гнучкість форматування абзаців і математичних формул (в останньому відношенні краще TEXа цю задачу не вирішує жодна програма).

TEX (і всі видавничі системи на його базі) невибагливий до використовуваної техніки. З іншого сторони, TEXовські файли мають високий ступінь переносимості: Ви можете підготувати LATEXовський вихідний текст на своєму IBM PC, переслати його до видавництва, і бути впевненими, що там Ваш текст буде правильно оброблений і на друку вийде в точності те ж, що вийшло у Вас при пробному друку на Вашому улюбленому матричному принтері (з тією єдиною різницею, що фотоскладальний автомат дасть текст більш високої якості). Завдяки цій обставині TEX став дуже популярний як мова міжнародного обміну статтями з математики та фізики.

LaTeX - це високоякісна набірна система; він включає функції, призначені для виготовлення технічної та наукової документації. LaTeX є фактичним стандартом для комунікації та публікації наукових документів \cite{lamport1994latex}. LaTeX доступний як вільне програмне забезпечення.

При роботі над звітом також використано сервіс Overleaf -- сучасний інструмент, розроблений у 2012 році. Він був створений щоб допомогти редагувати свої наукові статті, технічні звіти, тези, презентації, блок-схеми та інші документи, написані на мові розмітки LaTeX. 

При цьому, було використано всі переваги хмарних технологій, в тому числі можливість миттєвого початку роботи на практично будь-якому комп'ютері, збереження версій та одночасної роботи над проектом кількох користувачів.

Також нівелюється необхідність у встановленні на комп'ютері додаткового програмного забезпечення, що може бути названим одним із недоліків використання окремої системи, як LaTeX.

\subsubsection{Система контролю версій Git}

Активну популярність мають розподілені системи контролю версій (SCM).

Найбільш поширеними з таких є Subversion (SVN), Microsoft Visual Source Safe (VSS), Revision Control System (RCS), Concurrent Versions System (CVS), Gіt та Mercurіal. Знання подібних систем підвищує затребуваність ІT фахівців на ринку праці, покращує продуктивність розробників та полегшує рішення щоденних завдань. Саме передача знань є вирішальною у процесі експорту-імпорту технологій~\cite{киричек2012модель}.

В процесі роботи використано систему контролю версій Git з віддаленим репозиторієм на сервісі GitHub~\cite{gCalShedule}.

Система контролю дозволяє зберігати попередні версії файлів та завантажувати їх за потребою. Вона зберігає повну інформацію про версію кожного з файлів, а також повну структуру проекту на всіх стадіях розробки. Місце зберігання даних файлів називають репозиторієм. В середині кожного з репозиторіїв можуть бути створені паралельні лінії розробки — гілки.

Git підтримує швидке розділення та злиття версій, містить можливості для візуалізації і навігації за нелінійною історією розробки. 

\subsubsection{Sublime Text}

Sublime Text - прорієтарний текстовий редактор. Підтримує плагіни на мові програмування Python.

Розробник дає можливість безкоштовно і без обмежень ознайомитися з редактором, однак програма періодично буде повідомляти про необхідність придбання ліцензії.

Редактор містить різні візуальні теми, а також можливість завантаження додаткових тем.

Коли користувач набере код, Sublime Text, в залежності від використовуваного мови, буде пропонувати різні варіанти для завершення запису. Також редактор може автоматично додавати розділові знаки (<<\{>>, <<\}>>, <<;>>).

Sublime Text дозволяє збирати програми у готовий проект і запускати їх без необхідності використання зовнішньої командної строки. Користувач також може налаштувати свою систему компіляції і включити автоматичну збірку програм кожного разу при збереженні коду. Ця система схожа з відповідним плагіном для зйомки тексту від віддаленого LaTeX (розділ~\ref{subsub:latex}).

Використовується плагін LaTeXTools. Плагін LaTeXTools надає кілька функцій, які спрощують роботу з файлами LaTeX.

Команда ST збирає компіляцію джерела LaTeX у PDF за допомогою texify (Windows / MikTeX) або latexmk (OSX / MacTeX, Windows / TeXlive, Linux / TeXlive). Потім він розбирає файл журналу і перераховує помилки та попередження. Нарешті, він запускає програму перегляду PDF і, на підтримуваних переглядачах (Sumatra PDF на Windows, Skim на OSX і Evince на Linux за замовчуванням) переходить до поточної позиції курсора.

Додатково реалізована функція автозбереження, що допомагає користувачам не втратити пророблену роботу.


\subsubsection{Бібліотеки JS}

\paragraph{Express}

При розробці використано бібліотеку Express — гнучкий фреймворк для веб-застосунків, побудованих на Node.js, що надає широкий набір функціональності, полегшуючи створення надійних API.

Express забезпечує тонкий прошарок базової функціональності для веб-застосунків, що не спотворює звичну та зручну функціональність Node.js., при отриманні запиту він оброблюватиметься відповідно до визначення маршруту (рис.~\ref{fig:Route}), де app є екземпляром express, METHOD є методом HTTP-запиту, PATH є шляхом на сервері, HANDLER є функцією-обробником, що спрацьовує, коли даний маршрут затверджено як співпадаючий \cite{hahn2016express}.

Маршрутизація визначає, як додаток відповідає на клієнтський запит до конкретної адреси (кінцевій точці), яким є URI (або шлях), і певного методу запиту HTTP (GET, POST і т.д.).

Кожен маршрут може мати одну або кілька функцій обробки, які виконуються при зіставленні маршруту.

Express підтримує перераховані далі методи маршрутизації, які відповідають методам HTTP: get, post, put, head, delete, options, trace, copy, lock, mkcol, move, purge, propfind, proppatch, unlock, report, mkactivity, checkout, merge, m -search, notify, subscribe, unsubscribe, patch, search і connect.

Шляхи маршрутів, в поєднанні з методом запиту, визначають конкретні адреси (кінцеві точки), в яких можуть бути створені запити. Шляхи маршрутів можуть являти собою рядки, шаблони рядків або регулярні вирази.

Для обробки запиту можна вказати кілька функцій зворотного виклику, подібних middleware (детальніше про них в пункті~\ref{subs:middleware}). Єдиним винятком є те, що ці зворотні виклики можуть ініціювати next ('route') для обходу інших зворотних викликів маршруту. За допомогою цього механізму можна включити в маршрут попередні умови, а потім передати управління подальшим маршрутами, якщо продовжувати роботу з поточним маршрутом не потрібно.

В Express немає засобів для роботи з базою даних \cite{simon2015nodeexpress}. Їх надають модулі та бібліотеки Node.js, що дозволяють взаємодіяти з будь-якою базою даних. В роботі використовується ORM Sequalize (детальніше в пункті \ref{subs:orm})


\paragraph{JSON Web Token}

Для забезпечення конфіденційності при обміні даними використовується JSON Web Token. Деталі роботи з ним розглянуто в розділі \ref{subsubsection:jwt}. Для роботи з JSON Web Token використовується бібліотека jsonwebtoken.

JWT визначає особливу структуру інформації, яка відправляється по мережі. Вона представлена в двох формах - серіалізовані і десеріалізованной. Перша використовується безпосередньо для передачі даних із запитами і відповідями. З іншого боку, щоб читати і записувати інформацію в токен, потрібна його десеріалізація.



\paragraph{bcrypt.js}

Оптимізовано bcrypt в JavaScript з нульовими залежностями. Сумісний з C ++ bcrypt прив'язка на node.js і також працює в браузері.

Міркування безпеки
Крім включення солі для захисту від атак аеродинамічних таблиць, bcrypt є адаптивною функцією: з плином часу кількість ітерацій може бути збільшена, щоб зробити її більш повільною, тому вона залишається стійкою до атаки з використанням грубої сили навіть при збільшенні потужності обчислення. (подивитися)

Хоча bcrypt.js є сумісним з C ++ bcrypt прив'язка, він написаний на чистому JavaScript і, таким чином, повільніше (близько 30\%), ефективно зменшуючи кількість ітерацій, які можуть бути оброблені в рівний проміжок часу.

Максимальна вхідна довжина становить 72 байти (зауважимо, що символи UTF8 кодуються до 4 байт), а довжина згенерованих хешей - 60 символів.


\clearpage
\section{Проектування та розробка серверної частини}

\subsection{Клієнт-серверна архітектура веб-додатків} 
\input{ClientServer.tex}

\subsection{Мікросервісна архітектура додатків} 
\subsection{Мікросервісна архітектура}

Мікросервісна архітектура полягає в створенні для кожного з логічно відокремлених компонентів системи окремого модулю, пов'язаного з рештою.

Один з принципів проектування мікросервісних додатків додатків визначає, що розмір одного сервісу повинен бути таким, щоб повністю «вміщуватися» в голову програміста.

\addCodeAsImg{% Auth sequence uml diagram

\documentclass[a4paper,10pt]{article}
\usepackage[english]{babel}
\usepackage[left=3cm, right=1cm, top=1cm, bottom=1cm]{geometry}

\usepackage{tikz-uml}

\sloppy
\hyphenpenalty 10000000

\begin{document}
\thispagestyle{empty}
\begin{center}
\begin{tikzpicture}
\begin{umlseqdiag}
	\umlactor[no ddots, x=1]{User}
	\umlboundary[no ddots, x=5]{App}
	\umlcontrol[no ddots, x=8]{oAuth2}
	\umldatabase[no ddots, x=11.5, fill=blue!20]{FileSystem}
	\umlboundary[no ddots, x=13.5]{AuthAPI}
	
	\begin{umlcall}[op=Auth required action, type=synchron, return=Response, padding=3]{User}{App}
		\begin{umlfragment}[type=Auth]
		
			\begin{umlcall}[op=Read credentials, type=synchron, return=Credentials]{App}{FileSystem}
			\end{umlcall}
			\begin{umlcall}[op=Authorise, type=synchron, return=oAuth2]{App}{AuthAPI}
			\end{umlcall}
			
			\begin{umlfragment}[type=oAuth2, label=Error, fill=green!10]
				\begin{umlcall}[op=Get new token, type=synchron, return=oAuth2]{App}{oAuth2}
					\begin{umlcall}[op=Generate URL, type=synchron, return=URL]{oAuth2}{AuthAPI}
					\end{umlcall}
					\begin{umlcall}[op=Prompt interface, type=synchron, return=Code]{oAuth2}{App}
						\begin{umlcall}[op=Visit URL request, type=synchron, return=Code]{App}{User}
						\end{umlcall}
						\begin{umlcall}[op=Write token, type=synchron]{App}{FileSystem}
							\begin{umlcall}[op=Set credentials, type=synchron, return=]{oAuth2}{oAuth2}
							\end{umlcall}
						\end{umlcall}
					\end{umlcall}
				\end{umlcall}
				
				\umlfpart[OK]
				\begin{umlcall}[op=Create, type=synchron, return=oAuth2]{App}{oAuth2}
					\begin{umlcall}[op=Set credentials, type=synchron, return=]{oAuth2}{oAuth2}
					\end{umlcall}
				\end{umlcall}
				
			\end{umlfragment}
			
		\end{umlfragment}
		
	\end{umlcall}
\end{umlseqdiag}
\end{tikzpicture}
\end{center}

\end{document}
}{Авторизація з сервісами Google}{fig:GoogleServicesAuth}

В рамках системи закладено низку модулів, частина з яких використовує у своїй роботі доступ до сервісів Google, зокрема Google Sheets та Google Calendar. При цьому для взаємодії посередництвом Google API потрібно пройти процедуру аутентифікації (рис.~\ref{fig:GoogleServicesAuth}), закладену в методи бібліотек для основних платформ, в тому числі Node.js. Всі пакети мають відкритий вихідний код та поширюються разом з документацією.


\subsection{Розробка мікросервісів} 
Для обміну інформацією між користувацьким додатком та back-end частиною використовується протокол передачі гіпертекстових даних HTTP. Передачу даних забезпечує стек транспортних протоколів TCP/IP.
Одним з способів побудови мережевих HTTP-додатків є використання  асинхронного подієвого JavaScript–оточення Node. Для кожного з’єднання викликається функція зворотнього виклику, проте коли з’єднань немає Node засинає.

У Node не має функцій, що працюють напряму з I/O, тому процес не блокується ніколи. Як результат, на Node легко розробляти масштабовані системи \cite{zeiss2015node}.
Node широко використовує подієву модель, він приймає цикл подій за основу оточення, замість того, щоб використовувати його в якості бібліотеки. В інших системах відбувається блокування виклику для запуску циклу подій.

\subsubsection{Моделі та CRUD-операції}

В процесі проектування закладено серію моделей, що відповідають об’єктам предметної області. В рамках системи зберігаються в базі даних у вигляді таблиць з певними взаємозвязками (реляційну модель описано в підрозділі \ref{subsection:relationModel}). Для доступу до даних використовуються основні HTTP методи, що відповідають операціям CRUD (від Create, Read, Update, Delete), їх перелічено нижче.

\paragraph{GET}
\addCodeAsImg{\input{uml/ReadOperation}}{Виконання запиту на отримання об’єкта}{fig:ReadOperation}

Запитує вміст вказаного ресурсу, який може приймати параметри, що передаються в URI (рис.~\ref{fig:ReadOperation}). Згідно зі стандартом, ці запити є ідемпотентними — багатократне повторення одного і того ж запиту GET приводить до однакових результатів (за умови, що сам ресурс не змінився за час між запитами).

В запропонованій реалізації запит GET має дві версії — з параметром (ID) та без нього. Останній виконує дію (надає користувачу) не до конкретного об’єкту, а до всієї множини, що є необхідним в певних ситуаціях (наприклад, відображення списку всіх викладачів за певним критерієм).

\paragraph{HEAD}

Аналогічний GET, за винятком того, що у відповіді сервера відсутнє тіло. Це може бути необхідно для отримання мета-інформації.

\paragraph{POST}
\addCodeAsImg{% Auth sequence uml diagram

\documentclass[a4paper,10pt]{article}
\usepackage[english]{babel}
\usepackage[left=3cm, right=1cm, top=1cm, bottom=1cm]{geometry}

\usepackage{tikz-uml}

\sloppy
\hyphenpenalty 10000000

\begin{document}
\thispagestyle{empty}
\begin{center}
\begin{tikzpicture}
\begin{umlseqdiag}
	\umlactor[no ddots, x=1]{User}
	\umlboundary[no ddots, x=5]{App}
	\umldatabase[no ddots, x=14, fill=blue!20]{DB}
	
	\begin{umlcall}[op=Post request, type=synchron, return=Response, padding=3]{User}{App}
		\umlcreatecall[no ddots, x=8]{App}{JWT}
		\begin{umlcall}[op=Init, type=synchron, return=Response]{App}{JWT}
			\begin{umlcall}[op=Verify JWT, type=synchron]{JWT}{JWT}\end{umlcall}
		\end{umlcall}
		
		\begin{umlfragment}[type=Main, label=OK]
	
			\begin{umlfragment}[type=Create, fill=green!20]
				\umlcreatecall[no ddots, x=11]{App}{Object}
				\begin{umlcall}[op=Init, type=synchron, return=Object]{App}{Object}
					\begin{umlcall}[op=Store, type=synchron]{Object}{DB}\end{umlcall}
						
				\end{umlcall}	
			\end{umlfragment}
			
			\umlfpart[Error]
			
			\begin{umlcall}[op=Undo creation, type=synchron, return=Error]{App}{Object}\end{umlcall}
		
		\end{umlfragment}
	\end{umlcall}
		
	
\end{umlseqdiag}
\end{tikzpicture}
\end{center}

\end{document}



, label=OK, fill=green!10]
						\begin{umlcall}[op=Create token, type=synchron, return=Token]{DbUser}{JWT}\end{umlcall}		
						\begin{umlcall}[op=Success, type=synchron]{DbUser}{DbUser}\end{umlcall}
						\umlfpart[Error]		
						
						
\begin{umlcall}[op=Find one, type=synchron, return=User]{Object}{DB}\end{umlcall}	
				
				\begin{umlcall}[op=Check password, type=synchron, return=Response]{Object}{JWT}\end{umlcall}
				
				\begin{umlfragment}[type=Validate, label=OK, fill=green!10]
					\begin{umlcall}[op=Create token, type=synchron, return=Token]{Object}{JWT}\end{umlcall}		
					\begin{umlcall}[op=Success, type=synchron]{Object}{Object}\end{umlcall}
					\umlfpart[Error]				
					\begin{umlcall}[op=Error, type=synchron]{Object}{Object}\end{umlcall}
				\end{umlfragment}}{Виконання запиту на створення з аутентифікацією}{fig:CreateOperation}

Передає дані (наприклад, з форми на веб-сторінці) заданому ресурсу. При цьому передані дані включаються в тіло запиту. На відміну від методу GET, метод POST не є ідемпотентним, тобто багатократне повторення одних і тих же запитів POST може повертати різні результати (рис.~\ref{fig:CreateOperation}).

На першому етапі відбувається перевірка доступу користувача это створення об’єкту цього типу (авторизація), відповідно до прав доступу (рис.~\ref{fig:ApiAccess}).

\paragraph{PUT}
\addCodeAsImg{\input{uml/UpdateOperation}}{Виконання запиту на модифікацію існуючого об’єкту}{fig:UpdateOperation}

Завантажує вказаний ресурс на сервер. В розроблюваній системі використовується для редагування існуючих даних (рис.~\ref{fig:UpdateOperation}). 

В процесі виконання, спочатку з бази даних силами ORM вибирається конкретний об’єкт, в нього вносяться зміни, після чого він записується до сховища на заміну попередньої версії.

\paragraph{PATCH}

Завантажує частину ресурсу на сервер. При розробці необхідності у використанні не знайдено.

\paragraph{DELETE}
\addCodeAsImg{\input{uml/DeleteOperation}}{Виконання запиту на видалення об’єкту}{fig:DeleteOperation}

Видаляє вказаний ресурс.
Слід звернути увагу, що в процесі виконання запиту на видалення об’єкту в системі, видалення як такого не відбувається. Замість цього в окреме поле таблиці вноситься інформація про час виконання цієї процедури (рис.~\ref{fig:DeleteOperation}).

Такий спосіб реалізації дозволяє з однієї сторони приховати дані, відмічені як видалені від подальшого використання, а з іншої — зберегти їх там, де вони вже використовуються. В іншому випадку, у зв’язку з реляційністю бази потрібно було б вирішувати дилему — або проводити циклічне видалення для збереження цілісності даних, втрачаючи всі об’єкти, що посилаються на той, що видаляється; або ускладнювати структури даних, що потенційно призведе до дублювання даних.


\subsection{Проектування бази даних}
\subsubsection{Розвиток комп’ютерних баз даних}

База даних~--- сукупність даних, організованих відповідно до певної прийнятої концепції, яка описує характеристику цих даних і взаємозв'язки між їхніми елементами. Дані у базі організовують відповідно до моделі організації даних. 

В загальному випадку базою даних можна вважати будь-який впорядкований набір даних, наприклад, паперову картотеку бібліотеки. Але все частіше термін «база даних» використовуєтьсяу контексті використання баз даних в інформаційних системах, як і самі бази даних переносяться в електронні системи в процесі інформатизації. На даний час додатки для роботи з базами даних є одними з найпоширеніших прикладних програм \cite{ситник2004проектування}.

Через тісний зв'язок баз даних з системами керування базами даних (СКБД) під терміном «база даних» нерідко неточно мається на увазі система керування базами даних. Але варто розрізняти базу даних — сховище даних, та СКБД — засоби для роботи з базою даних. Надалі, в роботі під терміном «база даних», в залежності від контексту, може матися на увазі як сукупність даних чи певні її параметри, так і СКБД, крім випадків де це не очевидно.

Розроблення перших баз даних розпочинається в 1960-ті роки. Переважно, дослідницькі роботи ведуться в проектах IBM та найбільших університетів. Пізніше, на початку 1970-х років Едгар Ф. Кодд обґрунтовує основи реляційної моделі \cite{codd1970relational}. Уперше цю модель було використано у бази даних Ingres та System R, що були лише дослідними прототипами. Проте вже в 1980-ті рр. з’являються перші комерційних версій реляційних БД Oracle та DB2. Реляційні бази даних починають успішно витісняти мережні та ієрархічні. Починаються дослідження розподілених (децентралізованих) баз даних.

\subsubsection{Реляційна модель даних}\label{subsection:relationModel}

Реляційна модель даних~--- логічна модель даних, вперше описана Едгаром Ф. Коддом \cite{codd1970relational}. В даний час ця модель є фактичним стандартом, на який орієнтуються більшість сучасних СКБД.

У реляційній моделі досягається більш високий рівень абстракції даних, ніж в ієрархічній або мережевій. Стверджується, що «реляційна модель надає засоби опису даних на основі тільки їх природної структури, тобто без потреби введення якоїсь додаткової структури для цілей машинного представлення»~ \cite{codd1970relational}. А це означає, що подання даних не залежить від способу їх фізичної організації, що забезпечується за рахунок використання математичного поняття відношення.

До складу реляційної моделі даних зазвичай включається теорія нормалізації. Дейт визначив наступні частини реляційної моделі даних~\cite{дейт2008введение}:
\begin{enumerate}
	\item структурна;
	\item маніпуляційна;
	\item цілісна.
\end{enumerate}

Структурна частина моделі визначає, що єдиною структурою даних є нормалізоване n-арне відношення.


\subsubsection{Нормалізація бази даних}

Нормалізація схеми бази даних~--- процес розбиття одного відношення (таблиці в поняттях СУБД) відповідно до алгоритму нормалізації на кілька відношень на основі функціональних залежностей.

Нормальна форма визначається як сукупність вимог, яким має задовольняти відношення, з точки зору надмірності, яка потенційно може призвести до логічно помилкових результатів вибірки.

Таким чином, схема реляційної бази даних покроково, у процесі виконання відповідного алгоритму, переходить у першу, другу, третю і так далі нормальні форми. Якщо відношення відповідає критеріям n-ої нормальної форми та всіх попередніх нормальних форм, тоді вважається, що це відношення знаходиться у нормальній формі n-ого рівня.

\subsubsection{СКБД PostgreSQL}

PostgreSQL~--- широко розповсюджена система керування базами даних з відкритим вихідним кодом. Прототип був розроблений в Каліфорнійському університеті Берклі в 1987 році, пізніше проект Берклі було зупинено, а реалізацію було викладено в Інтернет під назвою Postgres95 після вдосконалення вихідного коду. Наразі підтримкою й розробкою займається група спеціалістів, які добровільно приєднались до проекту.

Сервер PostgreSQL написаний на мові C. Розповсюджується у вигляді вихідного коду, який необхідно відкомпілювати. Разом з кодом розповсюджується детальна документація.

\subsubsection{Шаблон проектування ORM} \label{subs:orm}

ORM~--- шаблон програмування, який зв'язує бази даних з концепціями об'єктно\,--\,орієнтованих мов програмування, створюючи «віртуальну об'єктну базу даних». В об'єктно-орієнтованому програмуванні об'єкти в програмі представляють об'єкти з реального світу. 

Суть проблеми полягає в перетворенні таких об'єктів у форму, в якій вони можуть бути збережені у файлах або базах даних, і які легко можуть бути витягнуті в подальшому, зі збереженням властивостей об'єктів і відношень між ними. Ці об'єкти називають «постійними». Існує кілька підходів до розв'язання цієї задачі. Деякі пакети вирішують цю проблему, надаючи бібліотеки класів, здатних виконувати такі перетворення автоматично. Маючи список таблиць в базі даних і об'єктів в програмі, вони автоматично перетворять запити з одного вигляду в інший.

В проекті використано ORM Sequelize. Спроектовано на реалізовано у вигляді моделей та відповідним їм таблиць структуру бази даних (рис.~\ref{fig:DbScheme}).

\addCodeAsImg{\begin{umlstyle}


\umlclass[x=4, y=0, fill=green!30]{Teacher}{
	+ Surname \\
	+ Name \\
	+ Patronym \\
	+ Department \\
	+ Post \\
	}{}

\umlclass[x=9, y = 12, fill=red!30]{Post}{
	+ Name \\
	+ ShortName \\
}{}

\umlclass[x=8, y=-9, fill=red!30]{Group}{
	+ Name \\
	+ Department \\
	}{}

\umlclass[x=0, y=-3, fill=blue!30]{Department}{
	+ Name \\
	+ ShortName \\
	+ Faculty \\
	}{}
	
\umlclass[x=8, y=0, fill=green!30]{Discipline}{
	+ Name \\
	}{}
	
\umlclass[x=12, y=-3]{Lesson}{
	+ Discipline \\
	+ Teacher \\
	+ Group \\
	+ LessonForm \\
	+ LessonNumber \\
	+ Day \\
	+ Week \\
	+ Schedule \\
	}{}
	
\umlclass[x=12, y=3]{LessonForm}{
	+ Name \\
	+ ShortName \\	
}{}

\umlclass[x=10, y=6]{Faculty}{
	+ Name \\
	+ ShortName \\	
}{}

\umlclass[x=4, y=-9, fill=red!30]{LessonNumber}{
	+ Name \\
	}{}
	
\umlclass[x=8, y=-9, fill=red!30]{Day}{
	+ Name \\
	+ ShortName \\
	}{}

\umlclass[x=8, y=-3]{Week}{
	+ Name \\
	}{}
	
\umlclass[x=4, y=-6]{Worker}{
	+ Surname \\
	+ Patronym \\
	+ Faculty \\
	}{}
	
\umlclass[x=4, y=-9, fill=red!30]{Schedule}{
	+ Name \\
	+ Chair \\
	}{}
	
\umlclass[x=8, y=-9, fill=red!30]{Group}{
	+ Name \\
	+ DateBegin \\
	+ DateEnd \\
	+ Faculty \\
	}{}

\umlaggreg[geometry=|-,mult1=1, mult2=n, pos1=0.2, pos2=1.9]{Department}{Teacher}
\umlcompo[geometry=--,mult1=1, mult2=n, pos1=0.2, pos2=1.9]{Faculty}{Department}
\umlassoc[geometry=--,mult1=1, mult2=1, pos1=0.2, pos2=0.9]{Subject}{Teacher}
\umlcompo[geometry=|-,mult1=1, mult2=n, pos1=0.2, pos2=1.9]{Department}{Speciality}
\umlassoc[geometry=--,mult1=1, mult2=1, pos1=0.2, pos2=0.9]{Speciality}{Group}
\umlcompo[geometry=--,mult1=1, mult2=n, pos1=0.2, pos2=0.9]{Group}{Subgroup}
\umlassoc[geometry=--,mult1=n, mult2=1, pos1=0.2, pos2=0.9]{Schedule}{Faculty}
\umlcompo[geometry=--,mult1=1, mult2=n, pos1=0.2, pos2=1.9]{Schedule}{Lesson}
\umlassoc[geometry=|-,mult1=n, mult2=1, pos1=0.2, pos2=1.9]{Schedule}{Worker}
\umlassoc[geometry=--,mult1=, mult2=, pos1=0.2, pos2=1.9]{Faculty}{Worker}
\umlassoc[geometry=--,mult1=1, mult2=1, pos1=0.2, pos2=0.9]{Lesson}{Subgroup}
\umlassoc[geometry=|-,mult1=1, mult2=1, pos1=0.2, pos2=1.9]{Lesson}{Subject}

\end{umlstyle}
}{Схематична структура бази даних}{fig:DbScheme}

З погляду програміста система повинна виглядати як постійне сховище об'єктів. Він може просто створювати об'єкти і працювати з ними, а вони автоматично зберігатимуться в реляційній базі даних.


\subsection{Розробка прикладного програмного інтерфейсу} Прикладний програмний інтерфейс~--- набір визначень підпрограм, протоколів взаємодії та засобів для створення програмного забезпечення. Спрощено -- це набір чітко визначених методів для взаємодії різних компонентів. API надає розробнику засоби для швидкої розробки програмного забезпечення. API може бути для веб-базованих систем, операційних систем, баз даних, апаратного забезпечення, програмних бібліотек тощо.

\subsubsection{REST API}

REST — підхід до архітектури мережевих протоколів, які забезпечують доступ до інформаційних ресурсів. Був описаний і популяризований 2000 року Роєм Філдінгом, одним із творців протоколу HTTP. В основі REST закладено принципи функціонування Всесвітньої павутини і, зокрема, можливості HTTP. Філдінг розробив REST паралельно з HTTP 1.1 базуючись на попередньому протоколі HTTP 1.0.

Дані повинні передаватися у вигляді невеликої кількості стандартних форматів (наприклад, HTML, XML, JSON). Будь-який REST протокол (HTTP в тому числі) повинен підтримувати кешування, не повинен залежати від мережевого прошарку, не повинен зберігати інформації про стан між парами «запит-відповідь». Стверджується, що такий підхід забезпечує масштабовність системи і дозволяє їй еволюціонувати з новими вимогами. Ці особливості сприяють використанню REST API при проектуванні мікросервісних додатків \cite[158]{кучер2018мікросервісна}.

REST, як і кожен архітектурний стиль відповідає ряду архітектурних обмежень (англ. architectural constraints). Це гібридний стиль який успадковує обмеження з інших архітектурних стилів.

\paragraph{Клієнт-сервер}

Перша архітектура від якої він успадковує обмеження — це клієнт-серверна архітектура. Її обмеження вимагає розділення відповідальності між компонентами, які займаються зберіганням та оновленням даних (сервером), і тими компонентами, які займаються відображенням даних на інтерфейсі користувача та реагування на дії з цим інтерфейсом (клієнтом). Таке розділення дозволяє компонентам еволюціонувати незалежно.

\paragraph{Відсутність стану}

Наступним обмеженням є те, що взаємодії між сервером та клієнтом не мають стану, тобто кожен запит містить всю необхідну інформацію для його обробки, і не покладається на те, що сервер знає щось з попереднього запиту.

Відсутність стану не означає що стану немає. Відсутність стану означає, що сервер не знає про стан клієнта. Коли клієнт, наприклад, запитує головну сторінку сайту, сервер відповідає на запитання і забуває про клієнта. Клієнт може залишити сторінку відкритою протягом кількох років, перш ніж натиснути посилання, і тоді сервер відповість на інший запит. Тим часом сервер може відповідати на запити інших клієнтів, або нічого не робити — для клієнта це не має значення.

Таким чином, наприклад дані про стан сесії (користувача, який автентифікувався) зберігаються на клієнті, і передаються з кожним запитом. Це покращує масштабовність, бо сервер після закінчення обробки запиту може звільнити всі ресурси, задіяні для цієї операції, без жодного ризику втратити цінну інформацію. Також спрощується моніторинг і зневадження, бо для того аби розібратись, що відбувається в певному запиті, досить подивитись лише на той один запит. Збільшується надійність, бо помилка в одному запиті не зачіпає інші.
s
\subsubsection{SOAP API}

SOAP — протокол обміну структурованими повідомленнями в розподілених обчислювальних системах, базується на форматі XML.

Спочатку SOAP призначався, в основному, для реалізації віддаленого виклику процедур (RPC), а назва була абревіатурою: Simple Object Access Protocol — простий протокол доступу до об'єктів. Зараз протокол використовується для обміну повідомленнями в форматі XML, а не тільки для виклику процедур. SOAP є розширенням мови XML-RPC.

SOAP можна використовувати з будь-яким протоколом прикладного рівня: SMTP, FTP, HTTP та інші. Проте його взаємодія з кожним із цих протоколів має свої особливості, які потрібно відзначити окремо. Найчастіше SOAP використовується разом з HTTP.

SOAP є одним зі стандартів, на яких ґрунтується технологія веб-сервісів.

\subsubsection{JSON Web Token} \label{subsubsection:jwt}

\addCodeAsImg{\input{uml/ApiAccess}}{Доступ на виконання запитів до системи}{fig:ApiAccess}

Для забезпечення конфіденційності при обміні даними використовується JSON Web Token. Роути, що обробляють реєстраційні та авторизаційні запити, представлено на рис.~\ref{fig:ApiAccess}.

JSON Web Token це стандарт токена доступу на основі JSON, стандартизованого в RFC 7519. Використовується для верифікації тверджень. JSON Web Token складається з трьох частин: заголовка, вмісту і підпису.

В корисному навантаженні зберігається будь-яка інформація, яку потрібно перевірити. Кожен ключ в корисному навантаженні відомий як «заява». Як і заголовок, корисне навантаження кодується в base64. Після отримання заголовку і корисного навантаження, обчислюється підпис.

\subsubsection{Публічне API}

В процесі проектування створено структуру роутів, котра може використовуватися сторонніми сервісами, у тому числі — і без авторизації в системі, що дозволяє отримувати інформацію про розклади власними силами для подальшого використання тим чи іншим чином. 
\subsubsection{Документація публічного API}

Однією із поставлених задач було розроблення документації до публічного API. Саме це є необхідною умовою для забезпечення принципів перевикористання коду і можливості використання проекту або окремих його мікросервісів сторонніми учасниками. Документацію опубліковано в репозиторії проекту \cite{gCalShedule}.



\subsection{Контроль якості програмного забезпечення}
\subsubsection{Задачі контролю якості програмного забезпечення}


\subsubsection{Бібліотеки Chai та Mocha}

\addCodeAsImg{\lstinputlisting[numbers=left]{code/ChaiMochaSample.tex}}{Перевірка з використанням Chai та Mocha}{fig:ChaiMochaSample}

MochaJS - це JavaScript фреймворк, який використовується для автоматичного тестування додатків. Він може використовуватися як на стороні сервера Javascript, так і в браузері. 

ChaiJS - це бібліотека для node.js і, як Mocha, Chai може використовуватися на стороні сервера або в браузері. Chai може бути використаний спільно з будь-якою бібліотекою для тестування.

Ми описуємо, що ми хочемо перевірити, використовуючи $describe()$ (на рис.~\ref{fig:ChaiMochaSample}). Функція приймає два параметри: String і callback. Цей рядок може бути будь-яким.

$it()$ використовується для опису того, що буде протестовано в цьому блоці коду. Дозволено писати вкладені описи $describe()$ і $it()$.

В процесі роботи використовувалися засоби як ручного, так і автоматичного тестування. 
Можна окремо виділити тести наступних підсистем:

\begin{enumerate}
    \item запуск системи та відсутність  критичних помилок часу виконання на початковому етапі;
    \item підключення до бази даних;
    \item виконання CRUD-операцій над об'єктами;
    \item підключення до сторонніх сервісів, що використовуються в роботі;
    \item успішна взаємодія з Google API.
\end{enumerate}



\subsection{Подальша робота}
\subsubsection{Використання QR-кодів}
\addimg{QRcode.png}{0.25}{Приклад QR-коду з посиланням}{fig:QRcode}

Було проаналізовано перспективи при використанні QR-кодів (рис.~\ref{fig:QRcode}) з метою супроводження традиційного паперового розкладу (та інших документів), що публікується на стендах.

Хоча термін «QR code» є зареєстрованим товарним знаком японської корпорації «DENSO Corporation», їх використання не обкладається ніякими ліцензійними відрахуваннями, коди описані та опубліковані як стандарти ISO~\cite{воронкін2014можливості}. Основна перевага QR-коду – легке розпізнавання скануючим обладнанням (за допомогою мобільного телефону, планшета або ноутбука з камерою, на яких встановлена програма для зчитування кодів, тощо).

Одним з способів використання QR-кодів в навчальному процесі, крім запропонованих (зокрема, задля забезпечення швидкого доступу до навчально-методичного забезпечення, довідкової літератури, веб-сервісів навчального закладу) \cite{воронкін2014можливості},  можна назвати надання доступу до електронної версії розкладу.

\subsubsection{Автоматична генерація розкладу занять}

TODO: write


\anonsection{ВИСНОВКИ}

Для виконання поставлених завдань було проведено аналіз характеристик існуючих систем планування, зокрема обсяг їх можливостей. При підготовці до проектування було приділено увагу окремим частинам процесу підготовки розкладу на прикладі факультету комп’ютерних наук, фізики та математики ХДУ.

На основі проведеного аналізу розроблено базові вимоги щодо можливостей додатку та його інтерфейсу.
Суттєву частину роботи приділено аналізу існуючих технологій всіх рівнів для створення веб-додатків. Детально досліджено роботу клієнт-серверних додатків та проектуванню API. 

Відповідно до створених вимог розроблено проект додатку та бекенд частини. Розроблено робочий прототип бекенд частини (зокрема реалізовано структуру бази даних засобами PostgreSQL, моделі з використанням ORM Squalize та окремі частини API і інтерфейсу додатку.

Сформовано проект документації до публічного API. При написанні ключових частин використано спеціальну форму коментарів, що забезпечують інтеграцію опису функцій та їх параметрів в підказки популярних IDE (інтегрованих середовищ розробки). Останнє є корисним при подальшій розробці, особливо при використанні існуючої кодової бази сторонніми розробниками, що є цілком можливим, зважаючи на модульність проекту при використанні мікросервісної архітектури.

При розробці проекту використовується система контролю версій git з публічним репозиторієм на сервісі GitHub (github.com/ Rembut/gCalShedule), що дозволяє використовувати сучасні методи сумісної роботи та, одночасно з тим, дозволяє використовувати результати проведеного дослідження всім охочим під ліцензією MIT.
 % Заключение
\include{bibliography} % Библиографический список

\end{document}
%%% Конец документа
