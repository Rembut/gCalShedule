\subsection{Розвиток комп’ютерних баз даних}

База даних – сукупність даних, організованих відповідно до певної прийнятої концепції, яка описує характеристику цих даних і взаємозв'язки між їхніми елементами. Дані у базі організовують відповідно до моделі організації даних. 

В загальному випадку базою даних можна вважати будь-який впорядкований набір даних, наприклад, паперову картотеку бібліотеки. Але все частіше термін «база даних» використовуєтьсяу контексті використання баз даних в інформаційних системах, як і самі бази даних переносяться в електронні системи в процесі інформатизації. На даний час додатки для роботи з базами даних є одними з найпоширеніших прикладних програм \cite{ситник2004проектування}.

Через тісний зв'язок баз даних з системами керування базами даних (СКБД) під терміном «база даних» нерідко неточно мається на увазі система керування базами даних. Але варто розрізняти базу даних — сховище даних, та СКБД — засоби для роботи з базою даних. Надалі, в роботі під терміном «база даних», в залежності від контексту, може матися на увазі як сукупність даних чи певні її параметри, так і СКБД, крім випадків де це не очевидно.

Розроблення перших баз даних розпочинається в 1960-ті роки. Переважно, дослідницькі роботи ведуться в проектах IBM та найбільших університетів. Пізніше, на початку 1970-х років Едгар Ф. Кодд обґрунтовує основи реляційної моделі \cite{codd1970relational}. Уперше цю модель було використано у бази даних Ingres та System R, що були лише дослідними прототипами. Проте вже в 1980-ті рр. з’являються перші комерційних версій реляційних БД Oracle та DB2. Реляційні бази даних починають успішно витісняти мережні та ієрархічні. Починаються дослідження розподілених (децентралізованих) баз даних.
