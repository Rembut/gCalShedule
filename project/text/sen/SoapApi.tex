\subsubsection{SOAP API}

SOAP — протокол для обміну повідомленнями в розподілених системах, що має певну структуру і базується на форматі XML.

Спочатку SOAP призначався, в основному, для реалізації віддаленого виклику процедур (RPC), а назва була абревіатурою: Simple Object Access Protocol — простий протокол доступу до об'єктів. Наразі протокол може використовуватися для обміну будь-якими повідомленнями в форматі XML, а не лише для виклику функцій процедур. SOAP є розширенням мови XML-RPC.

SOAP використовується з різними протоколами прикладного рівня: HTTP, SMTP, FTP та інші. Проте його взаємодія з кожним із цих протоколів має свої особливості, які потрібно відзначити окремо. Найчастіше SOAP використовується разом з HTTP.

SOAP є одним зі стандартів, на яких ґрунтується технологія веб-сервісів.
