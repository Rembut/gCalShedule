\subsubsection{SOAP API}

SOAP — протокол обміну структурованими повідомленнями в розподілених обчислювальних системах, базується на форматі XML.

Спочатку SOAP призначався, в основному, для реалізації віддаленого виклику процедур (RPC), а назва була абревіатурою: Simple Object Access Protocol — простий протокол доступу до об'єктів. Зараз протокол використовується для обміну повідомленнями в форматі XML, а не тільки для виклику процедур. SOAP є розширенням мови XML-RPC.

SOAP можна використовувати з будь-яким протоколом прикладного рівня: SMTP, FTP, HTTP та інші. Проте його взаємодія з кожним із цих протоколів має свої особливості, які потрібно відзначити окремо. Найчастіше SOAP використовується разом з HTTP.

SOAP є одним зі стандартів, на яких ґрунтується технологія веб-сервісів.
