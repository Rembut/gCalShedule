\section{АНАЛІЗ СИСТЕМ ПЛАНУВАННЯ}
\subsection{Історія виникнення комп’ютерних систем планування}
Ідея планування робіт існує стільки, скільки існує людська цивілізація, адже ще в неоліті, з переходом до тваринництва і землеробства, постають задачі з контролем циклічних процесів, що і викликало у подальшому створення календаря і писемності для фіксування задач.

\subsection{Сучасні сервіси для планування завдань}
На сьогодні, кожна людина так чи інакше стикається у повсякденному житті з системами, пов'язаними з контролем часу.

\subsubsection{Microsoft Outlook}
Microsoft Outlook — додаток-органайзер, входить в пакет офісних програм Microsoft Office. Дозволяє працювати з електронною поштою, надає функції календаря, планувальника завдань, записника і менеджера контактів. Крім того, Outlook дозволяє відстежувати роботу з документами пакету Microsoft Office для автоматичного складання щоденника роботи.

Outlook може використовуватися  і як окремий додаток, так і виступати в ролі клієнта для Microsoft Exchange Server, що надає додаткові функції для спільної роботи всіх користувачів організації: загальні поштові скриньки, папки завдань, календарі, планування часу загальних зустрічей, узгодження документів тощо.
Крім цього, дозволяє підключати через протоколи POP3/IMAP інші поштові сервіси та додатки, що надаються ними. Зокрема, нижче буде розглянуто синхронізацію MS Outlook з сервісами Google.

\subsubsection{Lightning}
Lightning — проект Mozilla Foundation, що додає функції календаря і планувальника в Mozilla Thunderbird — безкоштовну кросплатформну програму для роботи з електронною поштою і новинами, що може вважатися відкритим аналогом для відповідних продуктів з пакету Microsoft Office.

\subsubsection{Google Calendar}
Google Calendar — безкоштовний веб-додаток для тайм-менеджменту розроблений Google. Інтерфейс подібний до аналогічних календарних додатків, таких як Microsoft Outlook. Має різні режими перегляду, зокрема денний, тижневий та місячний. Події зберігаються онлайн, а тому календар можна переглядати з будь-якого пристрою, обладнаного доступом до мережі Інтернет. Додаток може імпортувати та експортувати файли календаря різних форматів, а для існуючих — задавати різні права доступу. 

Слід зазначити, що Google Calendar, як і інші сервіси Google, має відкрите API, що дозволяє взаємодіяти з ним через власні додатки після відповідних налаштувань.

\subsubsection{Корпоративні інформаційні системи}
Корпоративна інформаційна система — це інформаційна система, яка підтримує автоматизацію функцій управління на підприємстві і постачає інформацію для прийняття управлінських рішень. У ній реалізована управлінська ідеологія, яка об'єднує бізнес-стратегію підприємства і прогресивні інформаційні технології [2].

У загальному визначенні «автоматизована система» — сукупність керованого об’єкта й автоматичних керувальних пристроїв, у якій частину функцій керування виконує людина. Вона представляє собою організаційно-технічну систему, що забезпечує вироблення рішень на основі автоматизації інформаційних процесів у різних сферах діяльності. Сучасні автоматизовані системи управління навчальним процесом у  закладах вищої освіти здатні вирішувати велику кількість функцій, а саме [3]:
\begin{itemize}
	\item планування, контроль та аналіз навчальної діяльності;
	\item оперативний доступ до інформації про навчальний процес;
	\item єдину систему звітів, як внутрішніх, так і за вимогами МОН України;
	\item системи безпеки даних з урахуванням вимог законодавства;
	\item облік контингенту студентів та співробітників;
	\item проведення вступної кампанії;
	\item формування пакетів даних з метою виготовлення тих чи інших документів.
\end{itemize}

Функціонування будь-якої автоматизованої системи можна швидко адаптувати до особливостей навчального процесу конкретного навчального закладу, до локальних мереж різного рівня, що допомагає розширити коло користувачів (адміністрації, викладачів і студентів) для оперативного забезпечення їх необхідною інформацією. Отже, використання таких систем дає змогу не тільки удосконалити якість планування навчального процесу, а й оперативність управління ним [3].

Не зважаючи на всі переваги, які надає використання автоматизованих систем, досі далеко не в кожному закладі вони впроваджені чи використовуються в повній мірі з тих чи інших причин — інерційності поглядів адміністрації, супротив працівників або «саботаж» на місцях, відсутність фінансової або організаційної можливості.

В ХДУ використовується корпоративна інтегрована система «Інформаційно-аналітична система (IAS)». Вона дозволяє вести облік працівників і студентів, бухгалтерський облік, контроль за матеріальними цінностями. 		
Система дозволяє вносити і ефективно стежити за будь-якими змінами. В основі системи лежить ядро, на основі ядра виконується розширення системи до будь-якої кількості компонентів. При цьому основна функціональність може бути розширена за рахунок додаткових компонентів. 

Програма IAS орієнтована на платформу Windows з використанням MS SQL Server. Вона має багаторівневу архітектуру, що складається з бази даних, бізнес-логіки та клієнтського інтерфейсу. Внутрішній журнал реєстрації подій дозволяє вести та слідкувати за записами, що стосуються усіх подій.

Відсутність компонентів, пов’язаних з формуванням розкладу занять, та відсутність у використанні сторонніх рішень ставить задачу з проектування власного додатку для забезпечення всіх учасників освітнього процесу доступом до актуальної версії розкладу занять у будь-який час, а також можливості спрощення процесу формування розкладу та подальшої інформатизації освітнього процесу.

\clearpage