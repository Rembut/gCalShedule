% Auth sequence uml diagram

\documentclass[a4paper,10pt]{article}
\usepackage[english]{babel}
\usepackage[left=3cm, right=1cm, top=1cm, bottom=1cm]{geometry}

\usepackage{tikz-uml}

\sloppy
\hyphenpenalty 10000000

\begin{document}
\thispagestyle{empty}
\begin{center}
\begin{tikzpicture}
\begin{umlseqdiag}
	\umlactor[no ddots, x=1]{User}
	\umlboundary[no ddots, x=5]{App}
	\umldatabase[no ddots, x=11.5, fill=blue!20]{FileSystem}
	\umlboundary[no ddots, x=13.5]{AuthAPI}
	
	\begin{umlcall}[op=auth required action, type=synchron, return=response, padding=-4]{User}{App}
	
		\begin{umlfragment}[type=Auth]
			\begin{umlcall}[op=read credentials, type=synchron, return=credentials]{App}{FileSystem}\end{umlcall}
			\begin{umlcall}[op=authorise, type=synchron, return=oAuth2]{App}{AuthAPI}\end{umlcall}
			
			\begin{umlfragment}[type=oAuth2, label=Error, fill=green!10]
				\umlcreatecall[no ddots, x=8.25]{App}{oAuth2}
				\begin{umlcall}[op=get new token, type=synchron, return=oAuth2]{App}{oAuth2}
					\begin{umlcall}[op=generate URL, type=synchron, return=URL]{oAuth2}{AuthAPI}\end{umlcall}
					\begin{umlcall}[op=prompt interface, type=synchron, return=code]{oAuth2}{App}
						\begin{umlcall}[op=visit URL request, type=synchron, return=input code]{App}{User}\end{umlcall}					
					\end{umlcall}
					\begin{umlcall}[op=set credentials, type=synchron, return=]{oAuth2}{oAuth2}\end{umlcall}
					\begin{umlcall}[op=write token, type=synchron]{oAuth2}{FileSystem}\end{umlcall}	
				\end{umlcall}
				
				\umlfpart[OK]
				\begin{umlcall}[op=create, type=synchron, return=oAuth2]{App}{oAuth2}
					\begin{umlcall}[op=set credentials, type=synchron, return=]{oAuth2}{oAuth2}\end{umlcall}
				\end{umlcall}
				
			\end{umlfragment}
			
		\end{umlfragment}
		
		\begin{umlcall}[op=call action, type=synchron, return=, dt=-1]{App}{App}\end{umlcall}
		
	\end{umlcall}
	
\end{umlseqdiag}
\end{tikzpicture}
\end{center}

\end{document}
