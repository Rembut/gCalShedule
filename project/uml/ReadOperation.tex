% Auth sequence uml diagram

\documentclass[a4paper,10pt]{article}
\usepackage[english]{babel}
\usepackage[left=3cm, right=1cm, top=1cm, bottom=1cm]{geometry}

\usepackage{tikz-uml}

\sloppy
\hyphenpenalty 10000000

\begin{document}
\thispagestyle{empty}
\begin{center}
\begin{tikzpicture}
\begin{umlseqdiag}
	\umlactor[no ddots, x=1]{User}
	\umlboundary[no ddots, x=5]{App}
	\umldatabase[no ddots, x=14, fill=blue!20]{DB}
	
	\begin{umlcall}[op=Post request, type=synchron, return=Response, padding=3]{User}{App}
		\umlcreatecall[no ddots, x=8]{App}{JWT}
		\begin{umlcall}[op=Init, type=synchron, return=Response]{App}{JWT}
			\begin{umlcall}[op=Verify JWT, type=synchron]{JWT}{JWT}\end{umlcall}
		\end{umlcall}
		
		\begin{umlfragment}[type=Main, label=OK]
	
			\umlcreatecall[no ddots, x=11]{App}{Object}
			\begin{umlcall}[op=Parameters, type=synchron, return=Object]{App}{Object}
				\begin{umlcall}[op=Select query, type=synchron, return=Rows]{Object}{DB}\end{umlcall}
					
			\end{umlcall}	
			
			\umlfpart[Error]
			
			\begin{umlcall}[op=Error, type=synchron]{App}{App}\end{umlcall}
		
		\end{umlfragment}
	\end{umlcall}
		
	
\end{umlseqdiag}
\end{tikzpicture}
\end{center}

\end{document}
